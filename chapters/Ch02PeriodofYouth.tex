\chapter{Community Engagement and the Period of Adolescence} 

\begin{bclogo}[couleur=blue!30, arrondi=0.2, logo=\bclampe]{“And you? When will you begin that long journey into yourself?”}
	 - Rumi.	
\end{bclogo}


The desire for a better society may be one of the most ubiquitous in the world today, and this desire is held equally by youth \citepalias{UnitedNationsEnvironmentProgramme2011}. In general, the vision and methods of this better society may vary significantly, some fulfil this desire through work, some through raising children, some through the products they buy and others through volunteering. The most effect means to create a better society, however, is through systematic, ongoing community development efforts which, ideally, include an increasing number of people from within a population, including youth. This not only increases the human resources available for such efforts, it also demonstrates the importance of the participation to all community members, and simultaneously develops numerous skills and attitudes important for effective community building processes - consultation, coordination, perseverance, etc. To understand the contributions and role of youth to community building processes, we should consider if engaging individuals at different periods of life will bring about particular emotional, intellectual, social and physical benefits, and whether those benefits are limited to the individual or if they extend to the community. To do this, there must be an understanding of emerging capacities at each developmental period of life and the most effective corresponding opportunities for those to fully develop.

Since there is a significant focus on the development of individuals during this period, and the importance of how they interact with their environment cannot be overlooked, this chapter will incorporate both the psychological and sociological perspectives of youth engagement. While extensive research has been done on the benefits of youth engagement from a psychological development perspective, questions remain about why certain youth engage while others do not. And given the number of individuals who are able to rise up out of their surroundings and defy the odds, we can safely say that the environment in which young people exist has a great influence. Yet there are those individuals who are able to 'escape their destiny', or more specifically develop their capacity beyond what most from their social context are able to sustain. Hence, the integration of psychological and sociological insight will bring a richness and understanding to this endeavour. 

This chapter will take a brief look at the biology and youth studies fields which examine this period of life from different perspectives. The remainder of this chapter will explore some of the key areas of development for adolescents, and how this relates to their engagement in community settings. This draws largely from two fields - sociology and developmental psychology. This will cover identity, purpose and meaning in the context of individual development and how this process is affected by engaging in the community. The discussion will continue by looking at emotion, volition and cognitive development, and finally reflect on the ideologies of youth and how this contributes to social evolution. The discussion looks primarily at the period of adolescence, and when available articulates the research on early adolescence: how this period may be a pivotal time to engage individuals and that, compared to other age periods, there are significantly greater and sometimes unique developmental advantages. It is a time in which identity is beginning to be questioned \citep{Verlande2002,Finkenauer2012,Little2016}, moral framework is explored \citep{Fabes1999,Carlo1999} and growing independence means that connection moves beyond familial relations to incorporate the wider community \citep{Eccles2002a}. The early teenage years see changes in behaviour, language, thought processes and interactions which suggests the importance of further research in the context of community building. 


While there are often varying definitions of childhood, adolescence and adulthood, there is a growing consensus about the adolescence as a key period in which to develop their full capacity. Adolescence can be broken down into early (12-15), middle (15-18), and late (18 and above) \citep{Abela2011}. These definitions are rarely straightforward however; some youth literature may consider those over 18 adults and hence no long an adolescent, while numerous sources identify different age ranges for early adolescents. Humanities, for example, define early adolescence from 10 to 14 years of age \citep{JEA}, while reports from the Australian Government indicate a much narrower age range of 13-14 years \citep{McMatamney2009}. Another view of adolescence suggests that specified age ranges become less relevant than development milestones since puberty - the biological indicators of early adolescence - is a multifaceted construct, with genetic, environmental and sex impacting its onset \citep{Pfeifer2012}. Attempts to explore how early adolescents engage in the community have sometimes used these defined age ranges instead of recognising the developmental milestones that occur at different points throughout this age range. One study, for example, purporting to research early adolescents, conducted a self-reported survey with 10-11 year olds, and found the average research participant was only half developed in terms of puberty, with many likely showing little or no signs of pubescent maturation \citep{Lerner2005}. The relevance of pubescent maturation is important when considering the changing emotional, cognitive and social needs. Experiences in which an individual is exposed at different points throughout their life will impact them in different ways according to their level of development and capacity at the time; engaging them prior to puberty might be advantageous in one way, yet engaging them during and after may be advantageous in others. 
This suggests that defined age ranges used in this way may overlook different rates of development between individuals. We can also use other fields such as neuroscience which suggest that by the age of 12 the vast majority of individuals have experienced the onset of puberty, and use the ages of 12-15 as a guide for investigating this period. Since by 12 years old almost all individuals will have experienced the onset of puberty \citep{Pfeifer2012} and consequently are beginning to think deeply about society at large and their place therein, this research will echo the definition of neuroscientific research of early adolescents between the ages of 12–15 years. Additionally, this age is generally associated with the transition to high school for most Australians, and middle school in the United States; transitions such as this have traditionally been associated with new strengths and challenges \citep{Urdan1998}, as well as a period of stress and vulnerability \cite{Eccles2004}.
 
Youth engagement literature, however, has, to date, primarily focused on youth over the age of 15. With an increasing scholarship around engaging children as active citizens \citep[see for example][]{Naughton2008,Jans2004,Golombek2006}, and, with youth engagement research - whether it be democratic citizenship or positive youth development - examining youth primarily from 15 years of age, early adolescence seems to be a clear gap in the research. Hence, little is known about the importance or impact of community engagement during early adolescence, despite a growing body of research demonstrating emerging capacities \citep[see for example][]{Pfeifer2012}. This could be a missed opportunity in two ways - first it is possible that engaging in community development efforts during this age period brings unique advantages, and secondly it may also increase the likelihood of continued engagement throughout one's life, increasing the pool of potential resources for community development efforts. Whilst little is known about the benefits of engaging in community during early adolescence, areas of development ranging from identity to ideology, hormonal to mental capacities, all have the potential to influence both their experiences of service and the roles they can embody. %STEVE SAYS - Subsequent comments on why you want to focus on this age might refer back to this point. %% “little is known about the importance or impact of community engagement during early adolescence, despite a growing body of research demonstrating emerging capacities”


\subsection*{Youth Studies Intro} 

\label{IntroYouthStudies}
While nurturing the internal processes that contribute to development is important, the external environments in which one exists plays an undeniable role. The youth studies field is perhaps one of the most comprehensive, as it attempts to explore youth from an interdisciplinary perspective, incorporating historical, cultural, political and psychological experience of adolescence \citep{Furlong2013}. This perspective allows us to examine the full range of experiences of young people as well as how adults and society view youth, and the concept of youth in general. According to \citet[][p11]{Wyn1997}, the concept of youth needs to be considered a “social process whereby age is socially constructed, institutionalised and controlled in historical and culturally specific ways”. When viewed in this way, the period of youth as a transition to adulthood is only meaningful specific to the relevant social, economic and political conditions \citep[][p15]{Wyn1997}. 

For many in society today, their interactions with young people can be limited to media representations which often present youth as threatening or that the moral and cultural change they aspire to are threatening \citep[][p18]{Wyn1997}. Media representations of Australian youth in the first half of the 20th century presented youth as inherently bad and threatening, yet simultaneously they were a source of hope and optimism coming from vulnerability and intrinsic good \citep[][p18]{Wyn1997}. This period also saw youth as desirable for adults who wanted to control the aging process; it was this aspect which brought about the end of the period of youth as increasingly meaningless.

This period is often clouded with terms such as `early intervention' and `prevention', which have strong implications of inherent negative behaviours: that youth deviancy is inevitable. An extensively cited analysis by \citet{Muuss1975} of the various theories surrounding youth suggest that most youth studies theorists postulate adolescence as an important transition, he believes most no longer agree that there is a universal period of storm and stress, some even attesting to adolescence as relatively stress-free. 

Given the impact of environment on one's development, an examination of environments related to community engagement is relevant here. Positive environments such as those related to community engagement offer an opportunity for young people to experience environments of conscious and self-reflective practice aimed at changing social norms and structures \citep[][p317]{White1998}. While relatively few youth engage in these initiatives, these are opportunities to participate, develop agency and collective identity, discussed further in section \ref{ParticipateAgencyCollectiveIdentity}. Engagement, however, is not without its challenges \citep[][]{White1998}.



\section{Identity}
\label{IdentityPsychSoc}
Questions around identity remain an ongoing issue related to youth as an analytical category. From a psychological perspective, identity progresses through a number of pre-determined stages after which it becomes a fully formed adult identity which is both fixed and stable \citep{Stokes2015}. The contrasting sociological perspective challenges the idea that there is ever a fixed identity as an individual characteristic, rather that identity changes relating to the social context \citep{Stokes2015}. Identity, as explored by sociologists, allows insight into the interaction between individual and society yet is distinct from personal identity as it is concerned with shared status and relationships; it is defined as an individual's emotional, cognitive and moral connection with broader sense of community, practice, institution or category \citep[][p285]{Polletta2001}. “Beyond the question of how identities are defined", sociologists remind us, “is the question of the \textit{meaning} associated with an identity” \citep[][p83, emphasis added]{Deaux1991}. When an individual participates in a program, they feel a sense of connection to the group and develop a sense of belonging. Identifying with the group in this way is a natural expression of connectedness which subsequently informs their own individual identity \citep{Futch2016}. 

Identity, to most developmental psychologists, is an inherent characteristic of the individual, while to sociology it is an outcome of interactions which are informed by social categories and divisions such as ethnicity and race, gender, class, sexuality, place and disability \citep{Stokes2015}. While many argue of the freedom from class, geography and gender constraints, young people are increasingly influenced by discourses which shape their choices. Additionally, disadvantaged young people in particular have even less choice, with structural constraints resulting in a constant decrease in available options \citep{Stokes2015}. Whilst social class is increasingly invisible in the lives of young people - particularly in discourse - and cannot be explicitly generalised, the complexities relating to class continue to underlie understandings around the identities of young people. 

Blurred class divisions as they exist in the form of inequality and privilege continue to be a presence in the lives of young people; disadvantaged young people in particular experience this through limited opportunities in life \citep{Stokes2015}. Sociologists suggest these structural inequalities are overlooked in the psychological exploration of identity. \citet{Beck1992} also picks up the discussion, exploring how increasing individualisation reduces sources for collective identity accompanied by the epistemological fallacy of a “can do” rhetoric related to this individualisation process and the belief that young people can make individualised choices. To better understand these fallacies, and to understand young people's lives more broadly, sociologists draw on Foucault's concept of power as internalised instead of imposed, practiced through seduction instead of repression, resulting in disciplinary practices which shape individual experience, evaluate, understand, judge and conduct themselves \citep{Stokes2015}. While there is still significant debate around identity as “bounded and solidary” categories, they can still be a useful tool for mobilisation, particularly in social change movements \citep{Stokes2015,Brubaker2000}

\subsection{The Sociological Perspective of Identity}

\label{IdentitySoc}
These social identities are a means to belong and are a key component in health and well-being \citep{Futch2016}. Understanding these social processes of identity allows insight into the psychological importance of development opportunities in youth programs, while also recognising that specific activities are likely to nurture this development \citep{Futch2016}. It also assists beyond the youth program itself since engagement influences day-to-day thoughts, behaviours and actions, impacting other contexts of the lives of youth \citep{Futch2016}. In this way, young people are able to see their capacity to not only be moulded by their environment, but mould it. The interaction between the individual and their society becomes more organic when there is an active effort to mould and shape ones environment: “Such work is important to highlight because it suggests ways that we can move from understanding settings as mere crucibles for youth development to exploring how identity development can occur in conjunction with youth actively shaping their settings” \citep[][p677]{Futch2016}. This quote not only demonstrates the impact the environment has on youth, but also the potential for youth to shape their world. The changing and complex social, cultural and historical contexts, including family, community, and institutions, are argued to create a more holistic view of the individual and understand how individuals renegotiates and redevelops identity in response to changing and unpredictable social forces. 


The concept of identity means different things in different contexts. In an article on the complexities of identity and it's uses, \citet{Brubaker2000} detail the varied contexts in which it is used and the corresponding meanings. In the context of the self, for example, psychological understandings of identity refer to it as a fundamental condition of the individual. It is considered abiding, deep, basic or fundamental, as opposed to transitory, and should be valued, recognised, cultivated, supported and preserved. In the context of collective phenomena, however, identity infers a sense of sameness between group members. This gives rise to feelings of solidarity, shared consciousness or dispositions, or collective action. The social action context - whether individual or collective - it is often used as a means to differentiate action from self-interested motives, reflecting on the understanding of oneself as a means for action. It focuses on the development of the self that allows social action to occur, developing through interaction which produces a collective self-understanding. In the context of post-structuralism and post-modernism, identity is influenced by the work of Foucault. Reflecting the nature of the the present-day self, identity elicits the fragmenting, fluctuating and unstable nature of identity. Despite the disparity of definitions and uses of the term identity, \citet{Brubaker2000} suggest that identity can be an effectively took to mobilise individuals in areas such as collective action. 


While nurturing the internal processes that contribute to development is important, the external environments in which one exists plays an undeniable role. Identity in this context, as explored by sociologists, allows insight into the interaction between individual and society. Identity in this context is distinct from personal identity as it is concerned with shared status and relationships; it is defined as an individuals emotional, cognitive and moral connection with broader sense of community, practice, institution or category \citep[][p285]{Polletta2001}. “Beyond the question of how identities are defined", sociologists remind us, “is the question of the \textit{meaning} associated with an identity” \citep[][p83, emphasis added]{Deaux1991}. When an individual participates in a program, they feel a sense of connection to the group and develop a sense of belonging. Identifying with the group in this way is a natural expression of connectedness which subsequently informs their own individual identity \citep{Futch2016}. These social identities are a means to belong and are a key component in health and well-being \citep{Futch2016}. Understanding these social processes of identity allows insight into the psychological importance of development opportunities in youth programs, while also recognising that specific activities are likely to nurture this development \citep{Futch2016}. It also assists beyond the youth program itself since engagement influences day-to-day thoughts, behaviours and actions, impacting other contexts of the lives of youth \citep{Futch2016}. In this way, young people are able to see their capacity to not only be moulded by their environment, but mould it. The interaction between the individual and their society becomes more organic when there is an active effort to mould and shape ones environment: “Such work is important to highlight because it suggests ways that we can move from understanding settings as mere crucibles for youth development to exploring how identity development can occur in conjunction with youth actively shaping their settings” \citep[][p677]{Futch2016}. This quote not only demonstrates the impact the environment has on youth, but also the potential for youth to shape their world.


\subsection{The Psychological Perspective of Identity}

\label{IdentityPsych}
Identity development often requires a period of experimenting with different roles which can lead to an identity crisis. This crisis is often understood as the rejection of particular roles being forced upon them and subsequently, to the outside observer, the road to delinquency, seen by developmental psychologist \citet[][p132]{Erikson1968} as a mental illness, suggesting these choices need to be “diagnosed and treated” to avoid “seemingly psychotic and criminal incidents”. However he maintains that “adolescence is not an affliction but a normative crisis” \citep[][p116]{Erikson1956}. In this period of experimentation, when society identifies youth as “psychotic or criminal”, \citet[][p255]{Erikson1968} suggests that this may solidify them in their investigation of a negative or “delinquent” identity. Hence, the way in which society views youth particularly those who experiments with anti-social behaviours, can define and reinforce negative behaviours. Viewing them as resources to be developed however, opens up pathways \citep{Roth1998}.

The importance of peer perception is, to \citet{Erikson1968}, one element of identity consciousness. This can become identity confusion when an individual experiences dissonance between how they are perceived by others, their increased autonomous self-image, and their self-esteem. As an individual explores this dissonance, they will become increasingly self-conscious which arises alongside the culmination of autonomy, emerging visibility in an adult world, the judgement of peers, and self-certainly (which can often be veiled by a sense of group certainty). Through formations of subcultures, youth explore perceptions of themselves, how they are perceived by others, sexual maturity, identity and social placement \citep[][]{Erikson1968}. Significant individuals in the life of a young person can have a marked impact on identity at the individual level, while media, school and institutions have an impact on the collective level \citep{Juhasz1982}. Consequently, the personal and social interactions within which an individual engages forms identity, and if those are in the context of contributing to the community, a pro-social identity is likely to develop. In other words, the effects of young people engaging in community settings leads to a sense of identity which further stimulates community engagement.

Identity construction is developed by an individual's relationships with others. Romantic involvement during adolescence, for example, is seen as an act of identity formation through “projecting one’s diffused self-image on another and by seeing it thus reflected and gradually clarified” \citep[][p132]{Erikson1968}. In this way, intimacy is the fusion of identity, and as such true intimacy can only be achieved once identity formation is already progressing; individuals not able to form bonds of intimacy make themselves susceptible to more superficial forms of relationships and feelings of isolation \citep[][]{Erikson1968}. This points to the importance of healthy connections with others and the community at large as a contributor to one's sense of self and the reduction of long term mental health problems. On the other hand, cruelty to others - the exclusion and intolerance of others whether by their characteristics or by preferences and choices - facilitates and affirms identity development in youth. Through conflicting social values, this cruelty is understood to aid development of their own values \citep[][]{Erikson1968}. This process - the exploration of different values, testing boundaries, investigating attitudes and beliefs, and ultimately questioning their purpose in life - is increasingly understood as commencing in early adolescence \citep{Verlande2002}. In short, adolescent development - and early adolescent in particular - is a crucial time in which the environment in which youth find themselves, and the connections they make, shape who they are and how they see themselves in relation to their community.

\subsection{Identity Integration}

Following in Erikson's tradition, developmental psychologist \citet{Youniss1997} proposes that engaging during one's adolescent years significantly increases the likelihood of engagement during one's adult years in two ways: it familiarises the individual with the inner workings of community organisations and it provides an opportunity for the individual to incorporate community oriented activities into their lives as they form their identity. Engagement allows the integration of civic character into identity, which in turn forms the foundation for engagement during adulthood, hence, identity is the link which differentiates engaged and disengaged adults. Identities may develop over different contexts and time, however as an individual advances through adolescence, these identities will integrate and create a sense of continuity, what some consider the autonomous identity \citep{Kroger2006,Davidson1991}. In a study with Dutch adolescents between 12 and 20 years of age, developmental psychologists \citet{VanGoethem2012} found that the more identity contexts an adolescent experienced, the more likely they were to volunteer and the frequency of volunteering increased. Interestingly, they found that the more integrated their identity - the ability to integrate ideas of themselves over differing contexts and time - the less frequently they volunteered, suggesting the importance of volunteering opportunities in the early adolescent years when they are still exploring and developing their sense of identity \citep{VanGoethem2012}. 

The researchers identified significant differences in volunteering for older and younger adolescents, attributing this difference to identity integration and their capacity for moral reasoning. They suggested that for younger adolescents, volunteering is a means to explore identity \citep{VanGoethem2012}. However being able to integrate morality and identity, is a means to demonstrate a closer integration between one's ideal moral self and their real autonomous self \citep{Youniss1999a}. Moral development is, in the words of \citet[][p119]{Davidson1991}, “the primary identity's progressive acquisition of facility in entering, or opening up to, and eventually becoming explicitly aware of, the autonomous identity”. 

Adolescents are at the same time searching for a unique identity and a way to be part of a group. The inseparability of these two desires is effectively expressed by \citet[][p626]{Youniss1997}: “Participation adds social meaning to identity by providing specific information about being a civic actor along with like-minded others in the building of society... In this regard, the individual and society are not separate entities but complementary parts of a single relationship.” Understanding this relationship helps us explicate how adolescent identity development becomes an integral aspect of the evolution of social values. \citet{Juhasz1982}, for example, comments on how youth influence the evolution of values in society. In essence, their developmental characteristics allow adolescents to become influential agents in the processes of social change. He also notes the importance of mentors - teachers, parents and adults in the community - on the development of individual identity, while media, school, state and mass media have an impact on the collective level \citep{Juhasz1982}. He argues that in the current social milieu, much of the mass media, religion, science and education is focused on the self, and as such developing a strong civic identity can be challenging \citep{Juhasz1982}. Fundamentally, this demonstrates the importance of developing a strong civic identity through engaging youth in purposeful engagement directed at social change, and creating an environment that is simultaneously supportive and helps them deal meaningfully with prevalent social forces. 

\section{Belonging}
\label{BelongingPsych}
Lastly, an important part of the process of identity formation is belonging: the exploration of people and groups one might trust, and to whom service might be meaningful. Developmental psychologist \citet{Erikson1968} - who many modern psychologists continue to draw on - believes that these commitments are sporadically tested, so as not to seem foolish if they were to adhere themselves to a foolish cause. The meaning that is derived from these associations can be so important that adolescents will, for example, put aside material goals or the security of high paying jobs and choose meaningful occupations in lieu of greater or delayed remuneration \citep{Erikson1968}. As such, the experiences of youth needs to be relevant to them, to the historical era, and to their search for belonging, in order to capitalise on the courage, resourcefulness and devotion of youth \citep{Verlande2002}. In sum, community has profound implications for the development of identity, purpose and meaning which also has significant implications for how that individual develops, their relationship with others and how they see their role in the community. 


\label{BelongingSoc}
Whilst belonging - feeling connected to particular groups or cultures - is explored as an aspect of ones relationship to their community, it is also discussed as an element of human nature \citep{Antonsich2010}. \citet{Miller2003}, for example, draws on the Danish concept of Kierkegaardian to explore the the notion of correct relations - with community, history and locality. \citet[][p217-8]{Miller2003} considers belonging as “fundamental to who and what we are” and the “quintessential mode of being human ... in which all aspects of the self, as human, are perfectly integrated — a mode of being in which we are as we ought to be: fully ourselves.” This is reflected in the work of authors such as \citet{Calhoun2003} and \citet[][p16]{Antonsich2010} who talk about belonging as an “inescapable condition of humanity”. For some, it is more than an aspect of human nature, it is a means to motivate individuals to create social change. \citet[][p497]{Baumeister1995}, for instance, sees belonging as an innate need which drives individuals to action: “...the need to belong is a powerful, fundamental, and extremely pervasive motivation."


The many uses of the concept of belonging - sociology, psychology, geography, etc - conflate it with other concepts such as identity (particularly national or ethnic) and citizenship \citep{Antonsich2010}. Both individual and social identity become synonyms for belonging in discussions around identity politics and politics of belonging \citep{Antonsich2010}. This is particularly so as young people ask questions of “Who am I?", which is intimately linked with questions around “where do I belong?” \citep{Loader2006}. This sense of belonging is understood as a 'thicker' concept than that of citizenship, and lays the foundation for individuals to participate in and subsequently feel empowered to actively shape their community - consolidating that sense of belonging \citep{Antonsich2010,Mee2009}. 



In the context of the discussion at hand, belonging refers to an individuals ability to be “active participants in society” \citep[][p367]{May2011}. In a particularly relevant paper, \citet{May2011} explores how belonging is a key concept for studying the relationship between individuals and social change. She suggests that individuals and society are so inter-related that to study them separately is futile, and it is the relationship - in particular a sense of belonging that not only defines this relationship but serves as a foundation for the creation of social change. She explores how belonging allows an understanding of the relationship between the self and society that depicts social structures as actively lived, leaving open the possibility of individual influence on society and implications of everyday practices as a means to create social change. For \citet[][p375]{May2011} “We do not merely spectate a society or participate in it. We are \textit{in} it, we \textit{are} it.” [emphasis in original]


There is a common understanding of the impact society has on the individuals sense of self, yet often the opposite - the impact individuals have on social change - is overlooked. Sociologists have long discussed how our sense of self arises from our interactions with others, with society, and with our collectively held notions of social norms, customs and values: these are, fundamentally, the “social origins of the self” \citep[][p368]{May2011}. Yet change is not one directional: social change also occurs from idiosyncratic reactions of individuals to new situations. The cyclic “social change” influences “individual (inter)action” which further promotes more social change \citep[][p367]{May2011}. These are fundamentally small incremental changes arising from mundane activities of the layman, which re-frames the possibility of social change from heroic deeds performed by a small number of elite, to something undertaken by all \citep{May2011}. Belonging is a key aspect of this process, and yet, not belonging can also fuel a similar process of social change, exposing individuals to the new and unfamiliar which opens the mind to possibilities \cite{May2011}.

In sum, belonging connects individuals to the social, it is, to many, a key aspect of living a meaningful life, and often serves as a reflection of ones values. In a frequently quoted passage, \citet[][p88]{Weeks1990} highlights the intersection between belonging and ones sense of self: “Identity is about belonging, about what you have in common with some people and what differentiates you from others.” He goes on to situate this relationship as an element of potential conflict and the values we employ to overcome this conflict: 
\begin{quote}At its most basic it gives you a sense of personal location, the stable core to your individuality. But it is also about your social relationships, your complex involvement with others, and in the modern world these have become ever more complex and confusing. Each of us live with a variety of potentially contradictory identities, which battle within us for allegiance: as men or women, black or white, straight or gay, able-bodied or disabled, `British' or `European' ... The list is potentially infinite, and so therefore are our possible belongings. Which of them we focus on, bring to the fore, `identify' with, depends on a host of factors. At the centre, however, are the values we share or wish to share with others. \end{quote}





%RELATIONSHIP btn SOCIETY AND INDIVIDUAL - Possible quote? “self and society are mutually constitutive and therefore cannot be examined separately” p364 \cite{May2011}





\section{Morality and Identity} 
There is an increasing awareness about the reciprocal nature of community service and identity which ties in closely with one's sense of morality - commonplace, non-heroic respect for others which forms part of ones identity \citep{Youniss1999a}. Whilst youth form their identity through engaging with meaningful causes \citep{Youniss1997}, current and future service is strongly influenced by both morality and identity \citep{VanGoethem2012}. For youth in search of identity however, the values and ideologies of groups to which they belong will often dictate the options for identity that an individual would consider meaningful \citep{Phinney2011,Flanagan2011a}. Additionally, while all cultures and communities have the potential to assist in the development of civic dispositions, skills and identities \citep{Flanagan2011a}, identity may be critical in understanding continuity around engaging in one's community irrespective of varying social contexts throughout one's life \citep{Youniss1997}. Longitudinal research has also demonstrated the importance of community involvement for identity maturation \citep{Hardy2011}. Age, for example, is increasingly understood as influential in commencing volunteering, and a number of other factors, including identity integration, determine the frequency of volunteering \citep{VanGoethem2012}.

Important in the process of identity development is the ideal self; when this is closely connected to the current self, individuals are more likely to engage in pro-social behaviour \citep{Hart1995}. Identity is closely connected to behaviour; the more youth are able to develop their moral identity, the more involved they are likely to be in the community \citep{Porter2013}. Interestingly, a survey of students from the US about their civic activities as well as moral and political identities, and identified that as an individual's moral identity grows, there is a corresponding decrease in traditional political involvement \citep{Porter2013}. This points to community engagement - as distinct from political engagement - as a key means to develop identity, and more specifically moral identity.


The relationship between morality and identity is interesting, for as morality becomes integral to one's sense of self, individuals are less likely to attribute their actions to a sense of morals, but instead it becomes a norm. This integration represents an individual's moral identity and commonly refers to “the degree to which being a moral person is important to a person's identity” \citep[][p212]{Hardy2011a}. It is noteworthy however, to acknowledge that different interpretations of being a moral person may not be fully articulated in this analysis. For example, one could argue that few people, if any, believe they are immoral people. Many who undertake acts which are perceived as immoral are often able to justify those actions, and consequently are unlikely to view them as immoral \citep[see][]{Bandura1999}. 

The distinction between judgement and action with regards to morality can also be examined to aid understanding in this area. As an individual develops their moral identity, they increasingly understand morality as flowing from one's relationship to others. According to \citet[][p372]{Youniss1999a} “To understand the breadth of a person's moral orientation, then, one needs to know how that individual conceives of the self in relationship to personal others and to society”. Hence, actualising morality through relationships centres one's moral actions in identity, instead of a specialised psychological function. They continue, “...moral choice and action come from the individual's sense of self in relationship to others... moral actions lead to a moral identity which, in turn, leads to further moral actions and solidification of moral identity”. However moral judgements - the mental state or expression of belief - must be assessed through one's sense of responsibility before one acts, and it is increasingly believed that identity may hold the key to overcoming the moral judgement-action gap \citep[][see also \citealp{Hart1995}]{Hardy2011a}. This demonstrates the importance of opportunities for service as youth seek meaning, which can often be found through engaging in organisations with particular ideologies which resonate with an individual's moral framework. These opportunities provide an avenue for expression and identity consolidation \citep{Youniss1999a}. 

Viewing actions as dichotomous - as either self-interested or self-sacrificing overlooks how actions to help others enhance one's own moral self. When actions are understood as having the dual nature of enhancing one's own self while simultaneously being beneficial for others, altruistic deeds become merely an expression of their moral selves \citep[][see also \citealp{Colby1999}]{Youniss1999a}. How youth perceive themselves and their understanding of social norms are then transformed such that individuals who act morally do so believing they are reinforcing social norms, instead of seeing themselves as moralistic. This integration is effectively summed up in the words of one research participant who helped Jews in Nazi Germany: “I don't think I did anything special... What I did is what everybody normally should be doing” \citep[][p213]{Monroe1994}. Altruistic deeds as habitual action, then, result in everyday morality becoming an integral aspect of one's identity \citep{Youniss1999a}. 

Underlying much of the process of moral identity and social responsibility integration is a sense of civic efficacy - the belief that one can impact their community \citep{Crocetti2012}. When adolescents have a higher level of moral reasoning, they experienced increased feelings of moral responsibility and understandings of real life issues \citep{VanGoethem2012}; this can be linked to one's sense of moral identity and judgement which are founded on the same thought structures \citep{Davidson1991}. Understanding underlying social forces, such as workplace situations or greater social inequalities, provides motivation to engage which solidifies a young person's sense of moral responsibility and identity, which ultimately increases civic efficacy. 

The implications of a moral framework - the set of morals which an individual identifies with and uses as a basis to guide their behaviour - integrated into one's identity extend beyond engaging in community. Much of the literature on youth's engagement, for example, focuses on the value of engagement to the future time in the workforce, encouraging individuals to gain skills to improve their employability \citep{Shukra2012}. Yet rarely is this explicit in what attitudes and behaviours are required for the workforce. For a potential business owner for instance, there are obvious benefits to youth learning the importance of book keeping or sales, but is it not also important, perhaps more so, that they learn about honesty in a corrupt office environment and treating co-workers with respect and dignity, to be stalwart and persevere in the face of challenges? Or are we only encouraging them to make money no matter the consequences? Arguably, attitudes and behaviours such as honesty and service apply not only to the workplace, but also to community interactions more generally.


\section{Participation, Agency and Collective Identity}
\label{ParticipateAgencyCollectiveIdentity}
An exploration of the social dynamics of young peoples experience, Australian scholars \citet{Harris2010} and \citet{White2008} explore some of the challenges facing our younger generation who wish to participate in their communities. \citet{Harris2010} discusses potential barriers that young people face prior to any participation - political processes almost exclusively framed around adult interests and needs, and depicted as apathetic by the media and society - leaving feelings of marginalisation. Perceptions of apathy may be linked to the tendency for younger generations to engage in alternate forms of social action, often rejecting traditional forms of politics. If considering youth participation as does \citet[][p539]{Edwards2007}, among others, there needs to be a paradigm shift from “the deficiencies of individual youth” to “barriers that can precipitate young people’s disenfranchisement.” To do so would allow a more careful interpretation of the actions of young people in the context of barriers, providing a more nuanced understanding of the relationship between formal participation and apathy \citep{Harris2010}. Understanding these changes, according to \citet{Harris2010}, leads to new understandings around participation by young Australians, with subsequent changes in definitions, such as the Melbourne Declaration including a wider array of participation beyond formal politics. It also informs more broader definitions of participation to include any act that contributes to social change; in response to the tension around political and non-political engagement, a proposed definition of participation was broadened to include “acts that can occur, either individually or collectively, that are intrinsically concerned with shaping the society that we want to live in.” \citep[][p82-83]{Vromen2003}. Despite this shift, there continues to be a focus on such opportunities as a means to inform students of future responsibilities, with little discussion of current engagement \citep[][p109]{White2008}. Ironically, when young people do participate, they are often faced with hostility or disbelief in the authenticity of their choice of engagement: hostility from politicians when protesting or belief that they were manipulated \citep[][p110]{White2008}. 


Young people do engage in social and political issues, however they often do so in a way that defies traditional forms of engagement \citep{Harris2010}. Engagement for young people often comes in the form of social forums on a regional or even global scale, engaging in and organising transnational protests, consumer activism and volunteering in their local communities. This is argued to be a move away from materialistic values and a means to embrace the increasing globalisation \citep[][p3]{Bennett2003}. Because of this change in scope, many concepts around citizenship have also changed. The range of citizenship concepts have been described by \citet{White2008} as ranging from maximal to minimal, and are used to describe a range of ways to describe the role of individuals within society; these concept can limit or include young people in various ways. Minimal citizenship focused on legal status, voting and rights and responsibilities which often exclude youth, whereas maximal citizenship suggests individuals should be conscious of their role in society and accounts for an array of roles that an individual can play in ‘forming, maintaining and changing their communities’ \citep[][p108]{White2008}. Citizenship, according to maximal interpretations, develops reflective practices, critical thinking, self-determination and autonomy. Additionally, they allow individuals to understand their relationship to a shared, democratic culture, the social disadvantage resulting from lack of participation, as well as their own involvement in the community \citep[][p109]{White2008}. Conceptualising citizenship as maximal allows youth to be valued and valuable with their existing capabilities and not only as future contributors \citep[][]{White2008}. Young people can also be considered a resource to engage other young people. For example, in a research project involving young Australians, they identified that young people listen more when they are being engaged by other young people, as opposed to older people \citep{Harris2010}. As such, young people's involvement may need to go beyond mere participation to active be, in the words of Freire, both educand and educator. 

\subsection{Agency}
\label{AgencySoc}


At a conference, a candid UK youth worker commented on the “spray-on word[s]” used in the industry saying: “in the 1970s we had enfranchisement; 1980s participation; 1990s empowerment; and now citizenship” \citep[][p1]{Jeffs2005}. This comment is reflective of the changing ideas to effectively engage young people and develop their capacity, but underlying each of these “spray-on words” is the concept of agency \citep{Sercombe2010}. Despite a pervasive use of the term, there is a lack of clarity around the meaning and usage of agency. In a review article on the use of the concept of agency, \citet{Coffey2014} explored how the term is used in divers and diverse ways. The ambiguity of its use within the sociological field has led the authors to explore the differing interpretations - whether one possesses or acquires agency, whether it is a quality or circumstances prescribe the quantity an individual can possess, whether the concept of agency is meaningful if we consider it as both a possession and a conceptual tool to explain their subjective, day-to-day lives and biographies. This ambiguity continues despite an ongoing use of the term \citep{Coffey2014}. Despite this, there are some common threads in its usage. It is generally used to refer to intentional action and active subjectivity; the term can also refer to forms of self-expression and decision making \citep{Coffey2014}. The exploration of this term is often an exploration of the power imbalances affecting young people \citep{Coffey2014}. Agency is a concept which is often associated with notions of free will, freedom, action, originality, creativity and “the very possibility of change through the actions of free agents” \cite[][p240]{Barker2005}. 

Agency is often pitted against the structural limitations of youth experience; social structures being oppositional forces which limit ones agency \citep{Coffey2014}. There is ongoing debate around which has greater influence, with those believing the overwhelming influence of structure to be the most significant influence drawing primarily from Bourdieu, and those who believe the individualisation of society results in less structural influence and greater individual agency drawing on the work of Beck \citep{Coffey2014}. Emerging from this debate are those such as \citet{Evans2002a} that believe there is a form of dynamism in this relationship - what is commonly referred to as the middle ground concept of agency, termed bounded agency. This concept suggests that individuals have a level of agency which is confined within certain social frameworks, such as gender and class. Bounded agency explores the subjective feelings and beliefs of an individual making it analogous to emotions and attitudes, and which can be increased or decreased with respect to the relevant social frameworks \citep{Evans2002a}. This is an example of the dual impact of structures - enabling and constraining individuals through social forces beyond the control of individual agents, yet allowing them to act in certain ways \citep{Barker2005}. Behaviours which employ agency, in this scenario, are those that go against existing social trends. 



In relation to the area under discussion - the development of young people and their communities - the concept of agency is crucial as it “underpin[s] the possibility of ... social change” \citep[][p243]{Barker2005}. Social change is made possible, according to \citet{Barker2005}, through unique agents participating in conflicting social discourses which constitute our social reality. In this scenario the structural limitations are more obvious: 
\begin{quote}“Agency is a necessary and valuable quality for any society: we need people to take responsibility, to be active. However, it is a two-edged sword. People who are agents are not necessarily compliant, and tend to be more active in making power-holders accountable. The authorities would like people to be active citizens, but on their terms — to still be compliant, obedient, not to cause trouble. Unfortunately, you don't get one without the other. So authorities typically try to foster agency and sabotage it at the same time. Promoting a discourse (such as the empowerment discourse) and then dismantling it is one way to do this” \citep[][p131]{Sercombe2010}. \end{quote}
What this tells us is that agency alone is not responsible for changing social conditions, but that individual agents are important contributors to the process by contributing to the social discourse which shapes our understanding and subsequently our reality, which in turn changes the structural limitations in which we can act with agency. Both structural forces and individual agency shape who we are, in other words they shape our individual identity \citep{Barker2005}. 

This level of engagement, however, will largely depend on the individual's sense of agency. In this perspective, young people as educators who accompany educands, young people have an abundance of agency. Conceptualising young people with significant agency, particularly when accompanied with ideas of youth as social actors, can inflate actual capabilities as powerful social contributors \citep[][p315]{White1998}. On the other end of the spectrum, if agency is believed to be lacking, advocates will often step in, assume agency on behalf of the individual and act on their behalf \citep[][p317]{White1998}. These are merely two of the different theoretical approaches to agency which result in particular constraints. Many offer a deterministic view of agency with fixed stages and characterise development in a universalised way, ultimately suggesting a denial of agency \citep[][p3219]{White1998}. A discussion of these issues by \citet{White1998} highlights the contextual perspective's attempt to discuss agency in terms of historical constructs of lived practice, resulting in a contextualised idea of agency only surfacing in certain situations. Additionally they note the importance of social structures in shaping the behaviour of young people, as noted by the voluntaristic perspective. These various perceptions of engagement account for individual experience, perceptions of young people, as well as social and institutional influences on engagement, demonstrating the multifaceted conditions which inform participation. 


\section{Selflessness as a means to find self}

As youth reflect on broader social circumstance, they also contemplate their own roles. Engaging in community stimulates the formation of civic identity through the development of their role in political processes, their own commitment to moral principles, their own sense of agency and social responsibility \citep{Youniss1997}. This sense of duty to others is closely linked with in-depth identity exploration and commitment; this emerges as a sense of social responsibility and is understood as a potential mechanism to develop higher order identity processes related to community engagement \citep{Crocetti2012}. 

In a self-reporting study conducted in Italy, 14-20 year olds were asked to explore the relationship between identity and civic engagement. The researchers observed that youth who had a strong sense of identity had stronger aspirations to contribute to their community, particularly compared to peers who were still exploring identity \citep{Crocetti2012}. However a deeper exploration into an identity context, moreso than identity commitment, was positively related to social responsibility which, successively, was a predictor for previous and future community engagement. They concluded the importance of active reflection - discussed in more detail in section \ref{Reflection} - as a means to promote future engagement and identified that participation in community can have a reciprocal effect on an individual's civic efficacy and social responsibility, each leading to and resulting from the other \citep{Crocetti2012}. Again, the nexus to community engagement is obvious: youth who engage breeds civic identity to enforce responsible membership of the community; disengagement favours individualism and is not conducive to positive individual development. Active reflection is a means to bridge that identity gap. 

Increased understanding of moral issues also comes to the forefront when youth engage in the community. When an individual's understanding of moral issues is combined with a sense of responsibility, youth are able to apply moral reasoning and are more likely to act, which, in turn, consolidates identity \citep{VanGoethem2012}. In \citet{Youniss1997}'s study on community service projects with the homeless, youth were able to not only see themselves as able to make a positive difference and develop a sense of responsibility for others, but they also became acutely aware of underlying social forces. Through service, young people saw the importance of human actors in the construction of society, each with their own “political and moral goals", instead of “distant, preformed objects". This new-found understanding empowered them and provided new insight into social forces as well as their own capacity to affect change, helping them realise that their agency “gives them responsibility for the way society is and for the well-being of its members.” \citep[][p625, see also \citealp{Youniss1997b}]{Youniss1997}. Increased recognition of social forces allows individuals to act more consciously and in a manner which is conducive for both the individual and community (further discussion of social forces can be found in section \ref{SocialForces}. This quote also touches on the importance of agency; this capacity is, in itself, important, however as explored by \citet[][]{Youniss1999a}, in service to the common good it is integral to larger community development processes. The authors also noted that any analysis of civil society which uses top down approaches can leave a sense of dissatisfaction and frustration, however when a developmental analysis is employed, individuals mature and develop their capacity to make a positive impact in their society, while simultaneously understanding underlying social processes \citep{Youniss1997}. This suggests that as an individual develops their own capacities, a sense of responsibility to one's community coupled with a sense of agency is, fundamentally, a means to, and outcome of, individual and community engagement, bringing with it corresponding benefits.

As the development of identity is increasingly understood in the context of community service, it is clear that selfless and altruistic actions undertaken in the path of service are a powerful means to discover oneself. In the words of one social activist: “You felt like you were part of a historical movement... You were making a history and that you were... utterly selfless and yet found yourself” \citep[][p138]{McAdam1990}. These words demonstrate reciprocal relationship - engagement fuels identity development, while at the same time identity development fuels engagement \citep{Hardy2011}. In an analysis of identity and morality work, \citet{Davidson1991} propose that the development of morality and identity are different aspects of the same construct. Hence, moral judgement is merely an expression of an individual's identity or aspect thereof. In the formation of identity, community involvement contributes by: providing opportunities for the development of agency and self-efficacy; connecting with adults, organisations and ideologies which strengthen social connectedness; engaging in activities infused with, and providing spaces for reflection which nurture, values and ideologies \citep[][see also \citealt{Yates1996a}]{Hardy2011}. Interestingly, in addition to identity formation resulting from engagement, individuals who had more developed identities tended to engage more, reinforcing this reciprocal effect \citep{Hardy2011}. This suggests that in order to capitalise on the courage, resourcefulness and devotion of youth, experiences of community engagement need to be relevant to the youth themselves, to the historical era, and to their search for belonging \citep{Juhasz1982}. 






\section{Purpose and Meaning}

\subsection{Importance and Outcomes of Purpose}

Purpose is increasingly understood as a source of happiness and satisfaction in life \citep{Damon2003}. It provides answers to the big questions in life - `Why?' - and its related questions of `Why is it important?' `Why does it matter to me?' `Why am I doing this?' \citep{Damon2003}. Purpose consists of three aspects - future oriented goal which provides meaning and a path for forward momentum (particularly important for goals unachievable in one's lifetime), motivation to make a positive contribution to the world at large, and activity towards the realisation of that goal which is both current and sustained. If all three are unable to be fulfilled, this is considered a precursory form of purpose. Unfortunately, when an individual possesses a form of precursory purpose, there is little known about the developmental pathways required to fully realised purpose. Hence, those who engage and are future oriented, yet have self-serving motives for example, may not be able to fully realise the integration of values, goals and actions required for true purpose \citep{Malin2015}. Despite the importance of these purposeful goals, many chosen by youth may be unachievable in their lifetime, such as equality or universal education \citep{Damon2003}. 

Lacking purpose in life can impact an individual both physically and mentally. A lack of purpose can lead an individual to a life of self-indulgence, believing that this life will lead them to happiness; they mourn for meaning in their lives which \citet{Damon2003} maintains is a universal desire. A lack of purpose has been identified as a pathway to addiction, depression and self-absorption \citep{Damon2003}. For youth, low levels of purpose has been linked to drug use \citep{Padelford1974}, an inability to sustain relationships, deviance and lack of productivity \citep{Damon2003}. 


When purpose is achieved, particularly during one's adolescent years, it provides meaning, allows the individual to effectively prioritise, as well as motivates and inspires an individual to pursue learning and achievement \citep{Damon2003}. An understanding of purpose provides insight into the likelihood of potential issues; it provides an understanding of levels of determination, grit and morale, all of which allows the individual to overcome challenges and show resilience \citep{Damon2003}. For \citet[][p141]{Erikson1968}, purposefulness was a key principle to build “vital individual strength” for adulthood. He argued that the period of adolescence is an essential time to form “a realistic sense of ambition and purpose”. When an individual is able to develop purpose during adolescence, it leads to high self esteem, personal achievement, moral commitments and prosocial behaviours \citep{Damon2003}. 

Since purpose is understood as a “stable and generalized intention to accomplish something that is at the same time meaningful to the self and consequential for the world beyond the self” \citep[][p33]{Damon2003}, it provides meaning by connecting the individual to a world larger than him/herself, it implies the desire to make a contribution to others or create something anew. Purpose is about engaging in something that is challenging and compelling; it is marked by an individuals valued contributions to the world larger than himself \citep{Damon2003}. Purpose requires a certain level of commitment and progress to achieving a larger aim. To fulfil one's purpose, an individual also needs a structure of social support consistent with the effort exerted \citep{Damon2003}. Again, this all points to the importance of youth's engagement and ongoing support to ensure that engagement results in benefits for the individual and community.


\subsection{Purpose, Meaning and Youth} 
\label{Purpose}
Purpose brings together a sense of meaning and the ability to make a contribution to something greater, provides motivation and increases a sense of well-being. Those who lack purpose, however, are often challenged by low self-esteem and consumed by material desires \citep{Damon2003}. Community engagement is an effective means to fulfil this search for purpose, with a pertinent body of empirical work demonstrating purpose as both a motivator for, and outcome of engagement \citep{VanGoethem2012,Youniss1999a,Youniss1999,Hardy2011a}. 

As explored above, adolescence is a key time in the life of an individual in which the search for meaning and purpose is central to their thoughts, attitudes and behaviours - their inner and outer lives - and ultimately their sense of who they are - their identity. This search for purpose, a meaningful life and the pursuit thereof comes to the forefront during early adolescence and can continue throughout one's life \citep{Malin2015}. Indeed, youth have been found to reflect on their own sense of purpose naturally, without any prompting \citep{Inhelder2013}. Some have attributed this altruistic pondering - from the reconciliation of science and religion to social and political reform - to “egotism” and a “sophisticated game of compensation functions” \citep[][p344-345]{Inhelder2013}, which attributes a level of superficiality and lack of follow through, reinforcing stereotypical understandings of young people. Altruism, however, also has as direct impact on the life plans that youth adopt. From an analysis of youth's personal diaries \citet[][pp344]{Inhelder2013} note the importance of this search for meaning to their path in life, they state that such pondering “has a real influence on the individual’s later growth” and may form the basis for their later life pursuits. This contributes to the development of purpose which is significant because, according to prominent developmental theorists, when unable to be developed during their adolescent years, it is increasingly difficult to do so later in life further pointing to the significance of the period \citep[see][]{Marcia1980,Erikson1968,Damon2003}. 

The period of adolescence is a time of seeming contradictions, it is a time of passion and pursuit of a higher purpose in which youth can test their boundaries and desire to create meaningful social change. According to prominent purpose researcher \citet{Damon2009} these feelings of passion are rooted in the same deep brain systems as biological drives and the primitive elements of emotion. Hence, the neuro-development of the brain during early adolescence can lead an individual to risk taking behaviours while at the same time make them want to ponder world politics, it increases morbidity and mortality rates while also expanding their range of interests \citep{Damon2003}. Yet this passion is also intertwined with the highest levels of human endeavour: passion for ideas and ideals, for beauty, and to create music or art. Some have even suggested that if these emotions are directed towards the passion to succeed, whether it be in sport, business, or politics, or towards a person, activity, object or pursuit, they can engender transcendent feelings \citep{Dahl2004}. Such structures (for instance a non-familial mentor) can have a significant impact on the young person's search for purpose \citep{Damon2003}. Hence, when structures are put in place to support these desires, youth are more likely to think beyond themselves, and these passions can be aligned with more positive pursuits in service of higher goals \citep{Dahl2004,Mariano2012}, such as the development of one's community. 

A key component of purpose is developing values and goals and learning how to apply moral principles to complex situations \citep{Malin2015}. At a time when an individual is starting to plan life goals such as career and family, the development of values plays a key role \citep{Nurmi1991}. Early adolescents however are developing their cognition to grapple with increasingly complex situations, and this process can appear as a decline in moral decision making and loss of purpose. As their cognitive development continues however, they are able to have more nuanced perceptions of abstract moral principles and see links between suffering and the unjust actions of others \citep{Malin2015}. This demonstrates the value of supporting this crucial period of development, exposing youth to ever-more challenging situations and critically exploring with them nuances to maximise their development capacity.

These theoretical propositions have significant practical implications for practitioners working directly with youth. In an analysis of different engagement practices of youth, social scientists \citet[][]{Shukra2012} found that until they identified what sort of society they wanted, youth were not able to identify what sort of services they wanted, reducing engagement to an ideological enterprise. This confirmed the importance of a values based approach and leads to the necessity of asking sometimes difficult and uncomfortable questions: “What sort of society do we want? What are the barriers and opportunities to developing this? Who are the agents of change and their allies? How can we shape a youth service that supports youth in building such a society?” \citep[][p52]{Shukra2012}. This brings to focus the importance of engaging youth with a greater vision of community development and ensuring this vision incorporates supports for the inclusion of youth.



\section{Emotional, Volitional \& Cognitive Development} 

Whilst there is increasing awareness that adolescence is not to be seen as a time of storm and stress, there are significant biological, psychological, cognitive, and social changes occurring in the individual. The teenage brain, for example, is increasingly understood as unique from other stages of one's life, unique to both children and adult brains \citep{Blakemore2012}. Intellectual capacity matches an adult brain, learning is at its peak, and mental tasks including calculations, impulse control and reacting to emotional stimuli use different parts to the adult brain to varying degrees during this time \citep{NationalInstituteofMentalHealth2011}. These differences, scientists are quick to emphasise, do not imply an inferiority of the adolescent brain. To the contrary, their capacity to learn, expanding social life, desire to explore and test limits actually facilitate their transition to independence \citep{NationalInstituteofMentalHealth2011}. Whilst little is known about the impact of those in community building settings or on how they impact the developing brain, it would be difficult to argue that this would have no impact. 

Youth have a heightened response in emotional processing when exposed to emotionally charged images and situations which is also linked to the rewards system of the brain, effecting motivation \citep{NationalInstituteofMentalHealth2011}. A willingness to engage is increasingly important as the young person establishes their independence and asserts their place in society. All these changes suggest a potential window of opportunity specific to adolescence which raises questions around the long-term effects of early adolescent experiences and whether this can be used to create opportunities for positive pathways at particular developmental periods \citep{Dahl2004}. 

The increased freedom and autonomy that accompanies adolescence heightens the impact of experiences which, in the context of community created lasting feelings, both positive and negative. This autonomy also comes with an increased choice of activities which undoubtedly influence the developing adolescent brain \citep{Kuhn2006}. Rewards and emotions which generally follow these new experiences create a level of excitement in the the teenage brain, and when these experiences are enjoyable there is a release of dopamine which further increases the enjoyment of the activity (more so than adults and children). This dopamine creates a desire to seek out that experience again \citep{Galvan2013}. From such experiences, adolescents will create meaning and will use this meaning to create identity. Hence, the activities in which they engage, and the meaning to which those activities are attributed, contribute to their sense of self \citep{Kuhn2006}. 

These changes also affect motivation and emotion \citep{Dahl2004}. Adolescent vulnerabilities, for instance, are increasingly associated to changes in social and affective processing, yet this may also bring advantages such as increased flexibility in their ability to adjust intrinsic motivations and priorities in life in response to changing social contexts \citep{Crone2012}. However, motivations and priorities can often be linked to the desire to change these social contexts. The increased sense of idealism experienced by adolescents is often coupled with a desire to contribute to the world around them \citep{Dahl2004}. Changes in motivation and emotion remain largely understudied, and further studies into this area could provide insight into how actions channelled into a range of activities can impact motivations and passions, and how they are shaped by particular experiences \citep{Dahl2004}. 

Early adolescence is increasingly recognised as unique in studies of cognition, identity, purpose, etc, yet research on their community engagement in general remains scarce. Relevant research includes community engagement as it relates to purpose in 12-18 year olds \citep{Barber2013}, identity in 12-20 year olds \citep{VanGoethem2012} and gratitude in 12-20 year olds \citep{Froh2010} (a more detailed analysis of these studies can be found in section \ref{Purpose}). One study surveyed youth aged 12-19 (and notably was one of the few inter-country studies), yet the early adolescence specific analysis was limited to an observation of an overall decline in civic engagement, despite a continuing endorsement of such initiatives throughout this age \citep{Flanagan1999}. A meta-analysis of literature on prosocial behaviours across adolescence identified a dearth of knowledge on early adolescence, yet suggested “there are general increases in prosocial behavior during this time when compared with early age periods” \citep[][p5]{Fabes1999}. 

The interplay between early adolescent developmental processes creates an important context for participation in community, and in particular the two-fold purpose of individual and community development. In this milieu it is easy to see why adolescence is considered a “critical time for the interplay in development of the brain, the mind, and the person” \citep[][p65]{Kuhn2006}. The cognitive changes occurring during this time are defined by the environments in which early adolescents spends time. These changes result in a more efficient brain, creating stronger, fewer associations to those contexts with which they are exposed \citep{Blakemore2012,Kuhn2006}. These processes are generally complete by the end of early adolescents \citep{Kuhn2006}, and suggests that environments to which this age group is exposed contributes to behaviours both destructive and constructive to themselves and others \citep{NationalInstituteofMentalHealth2011}. This largely determines what an individual most strongly associates with during their lifetime. These cognitive processes allow adolescents to imagine their future selves in a way that brings out positive emotions, and play a part in the development of self and identity in a way that supports those future selves \citep{Dahl2004}.

The development of skills at any age requires careful and systematic attention. In the case of early adolescents, when they are just starting to gain the faculty of critical thinking and awareness of themselves in relation to society, many have limited opportunities to systematically develop these skills \citep{ForrestBank2015}. Certainly, some may have opportunities in school settings or through family interactions, but this is by no means a universal opportunity, and rarely if ever would it be systematically implemented to ensure full development of these abilities. To develop capable, community minded citizens, these early adolescent years, when establishing their independence, may prove critical. 

\subsection{Neurological development}

\label{Neuro} 
These changes are, in part, explained by another perspective in which to understand the development of youth - the field of biology. Youth experience increased levels of hormones which are argued to be the cause of many adolescent behaviours, from increased sexual drive to deviancy. Yet hormones are also understood to impact behaviour, in particular social behaviour \citep{NationalInstituteofMentalHealth2011}. Dopamine, for example, is one such hormone, and its presence influences the level of enjoyment and desire to re-engage with certain environments and experiences in the future, in particular how they value certain social experiences \citep{Crone2012}. To extend this finding, one could reasonably assume that these hormones impact an individual's willingness to engage and re-engage in community building settings. Questions remain, however, about the effects of hormones on early adolescents, specifically how intense feelings impact social valuation \citep{Crone2012}. The cognitive abilities of adolescents allow for increased impulse control, with increasing consistency as they age and their brains mature \citep{Johnson2009}. Studies of adolescent brain development also identify unique mental capacities - such as critical thinking - developing during these early adolescent years. From this perspective, we can see an emerging understanding of the significant of this period that are yet to be fully understood or appreciated. Despite criticisms from other fields based on observations of young children undertaking critical analysis, the importance of early adolescence is becoming increasingly clear in many domains, and the impact of community on their emerging capacities cannot be overlooked. Whilst this demonstrates the unique potential of this age and the importance of environmental considerations during the period of youth, we must be wary however not to use brain development as an explanation for all forms of development experienced in the second decade of an individuals life. Indeed, some have observed how this brain development accompanies cognitive, psychological and social changes \citep{Kuhn2006, Wyn2011a,Cote2002}. 


\section{Realising Identity, Purpose and Meaning through Community Engagement}
Purpose, meaning and identity are all important developmental milestones for every young person. As they begin to ask questions such as “Who am I?”, their sense of purpose, fuelled by emerging ideals, concerns and values inform this journey and consolidate the process of identity formation \citep{Malin2015}. For many, this is evidenced in their participation in community, whether it be through participating in community events, volunteering, or being involved in social action (see table \ref{FormsOfEngagement} for definitions). This process often coincides with an increased awareness of the importance of morals in their life, and as they become central to their identity, individuals seek corresponding actions which reflect this, resulting in prosocial behaviour. To act on one's quest for meaning provides purpose; both understood as being rooted in the same moral development processes - cognitive, emotional and identity formation. Purpose, according to \citet[][p109]{Malin2015}, “is driven by the recognition that by understanding what I can contribute to the world and by acting on that understanding, I can have a meaningful existence”. In some cases, it can appear as if youth have a lack of purpose, however this is often merely an exploration of social roles and the process of identity consolidation \citep{Malin2015}. 

In a longitudinal mixed method study, \citet{Malin2015} explored community service as a means to develop purpose. For many in this study, service was be an effective means to express their emerging purpose and meaning. Individually, youth needed to develop the capacity for meaningful participation which includes the desire for, and commitment to sustain engagement, as well as developing values which serve as a foundation for socially responsible behaviour. Developing this moral structure requires an individual to not only learn about social values, but gives them the opportunity to inquire about the reason those values exist as they do \citep{Malin2015}. 

Purpose is an expression of meaning, is it part of who you are and provides motivation to contribute to the world. Using individual skills to address a need in the community, this study showed, created a sense of enjoyment, was a source of motivation and contributed to feelings of hope. Working alongside other community members also provided a culture and atmosphere which inspired one to continue along their path of purpose \citep{Malin2015}. This demonstrates that by engaging youth, they not only find meaning and purpose, but they contribute to the process of change in their community. 

The relationship between engagement of various kinds and purpose has been extensively researched. In a study looking at Chinese adolescents, when an individual had strong prosocial values, volunteering intention and behaviour, they also had a high sense of purpose \citep{Law2009}. Indeed the importance of purpose extends beyond one's adolescent years. Across the globe in a US study looking at retirees, researchers found a correlation between those who volunteered more than 10 hours a week and those who scored highly on the purpose in life test \citep{Weinstein1995}. Another US study by educational psychologist \citet{Barber2013} looked more deeply into the relationship between purpose and engagement proposed that \textit{how} purpose is manifested, instead of who finds purpose in community engagement, was a stronger predictor of who was likely to continue their engagement long term. This longitudinal study, found that different engagement practices between demographic groups of youth lies less in who is able to find purpose through engagement, and more how purpose is manifested. The researchers conclude that engagement programs should allow youth to consider various forms of community engagement, and not limit their choices, since the process of finding an activity of personal importance is an integral part of the process \citep{Barber2013}. Hence, engaging in community, and in particular the form that engagement takes can shape an individuals purpose, which has fundamental implications for the well-being of the individual. 

While one form of engagement may lead an individual on a path of purpose, for others it may lead to frustration or a more limited focus for ongoing engagement. With any engagement opportunity challenges often arise - this may be external from individuals, organisations or laws which inhibit their efforts, or internal from preconceived expectations of the experience and its results. The stronger one's sense of purpose, the easier it will be to overcome these challenges, and the greater likelihood that civic identity will be formed \citep{Malin2015}. Such challenges often arise from the social context which plays a role in development of civic purpose. For marginalised groups this can often lead to disempowerment or engagement in activities which serve the interests of their own groups instead of wider society \citep{Malin2015}. In these cases practitioners must be mindful of reflecting with these youth, aligning their efforts with underlying values and goals, to ensure that the development of social responsibility “goes beyond concern for others to encompass concern for interdependent society and a personal sense of responsibility for sustaining the common good” \citep[][p110]{Malin2015}. 

For youth who have a strong civic purpose, their motivations are often derived from well developed values. In a US-based longitudinal study looking at civic purpose with youth, those who held strong values and beliefs about social issues were far more likely to engage and discussed their engagement with respect to their values \citep{Malin2015}. This study found that, of the values that motivate an individual with civic purpose, social responsibility was common, and of the social contexts that seem to nurture civic purpose, family, faith and community was the strongest \citep{Malin2015}. Once civic purpose was identified, parents, community members etc. were often needed to assist in the identification of opportunities to further develop those interests \citep{Malin2015}. Hence the opportunities need also provide youth with personal support prior to, and during their engagement. 


While purpose answers questions of why, noble purposes must account for both the \textit{how} and the \textit{why} of the goal, devoting oneself to a worthy goal and undertaking that task in an honourable manner \citep{Damon2003}. Thus, “noble purposes also may be found in the day-to-day fabric of ordinary existence” \citep[][p44]{Damon2003}. Noble purpose provides mundane activities with meaning, and is not limited to heroic tasks or life-endangering adventures. Purpose is taking pride in your work, interacting with siblings in a joyful manner, doing household chores with care, being friendly to new or younger students. Hence, when purpose is imbued with nobility, an attitude of service becomes seamlessly integrated into all aspects of one's life. 











\section{Meaningful Opportunities and Environments}
The environment in which young people engage may temper some of the potential outcomes. An example of this was from a study on two generations of Czech's which showed that more recent generations were less likely to demonstrate a sense of civic identity. The researchers attributed this to an increased level of professionalised volunteering opportunities and less non-professional ones \citep{Serek2016}. Professional volunteering opportunities limited opportunities to develop responsibility and a sense of connection, whereas non-professional organisations allowed volunteers to negotiate roles, demonstrate equality with all members of the organisation and engage in a more supportive environment increasing empathy, understanding of others and friendly relationships \citep{Serek2016}. This is indicative of the shift from traditional forms of meaning and purpose often derived from work, future prospects and youth policies, to modern forms of meaning which are increasing sought from outside the workplace \citep{Abbott-Chapman2000}. Deriving meaning requires deep thought and reflection which is increasingly being sought from non-traditional means. Some have even suggested that mind altering stimuli such as drugs and alcohol is a way to achieve this form of reflection, making time go slower or to relieve themselves of feelings of responsibility \citep{Abbott-Chapman2000}. In the words of \citet{Hall1996}: “Self enquiry, self reflection and self assessment have never been more important than they are today”. He continues by linking this reflection to a continuous form of learning and moral self-improvement and a means to feel successful: “It is essential to be a continuous learner and to generate the excitement and motivation from this process in achieving personal values and goals and hence “psychological success”.” This search for spaces in which to reflect and consider their moral framework is increasingly difficult to find in a society increasingly consumed with the rhetoric of immediacy and engulfed by decreasing levels of social trust \citep{Eisenstadt1995,Abbott-Chapman2000}. This disjuncture can result in apathy towards ideological motives, cultural and spiritual superficiality, and increasingly empty and meaningless relationships \citep{Eisenstadt1995,Abbott-Chapman2000}. Hall points to additional barriers that can exist for those who lack education in this process of self-fulfilment and personal success: “The downside is that people who lack education, basic skills and motivation will be at a disadvantage” (\citealp[][p14]{Hall1996} as quoted in \citealp[][p23]{Abbott-Chapman2000}). 





\section{Ideologies, Evolution and Engaging Young People in Social Change}
In addition to identity, developmental psychologists maintain that adolescents are developmentally programmed to desire two things: an ideology or belief system to which they can adhere, and having an impact on both the present and future from a wider social perspective \citet{Juhasz1982}. As adolescents are most attuned to changing ideology, it is a key time in which to develop purpose and systems of belief \citep{Erikson1968}. This newly found ideology can be effective in creating a sense of enjoyment towards an occupation, and a means to avoid acting out destructive feelings \citep{Erikson1968}. \citet[][p187]{Erikson1968} observed the importance of ideologies in providing youth with 
\begin{enumerate}
 \item a counteraction of “time confusion” through simplified vision of the foreseeable future;
 \item insight into the promises and challenges of the inner and outer world;
 \item a form of social uniformity which hinders identity-consciousness;
 \item collective action which discards inhibition and guilt;
 \item justification for progress coming from an understanding of prevailing technology;
 \item an understanding of self in relation to a particular world perspective; 
 \item guidelines for sexual intimacy concerned with a relatable set of principles; and 
 \item insight into a social order with adults community members who avoid paternalism.
\end{enumerate} 

\citet{Erikson1968} sees a distinction between early and middle adolescence in the development of ideologies. He suggests the ages 16-20 are when youth become cognisant of social needs and awareness, and they develop an ability to think about issues, groups and ideologies beyond themselves and their immediate surroundings. He also notes that the preceding years, 11-15, are when youth start to contemplate hypotheticals in new ways, allowing them to explore a full range of possibilities which would have been previously impossible, and laying the foundation for the development and attachment to new ideologies. This period also sees the burgeoning of understanding around historical relevance; the impacts different social and historical contexts have on current trends and situations, and consequently attempts to align themselves with groups which have a more acceptable historical identity. In this way, we often see the importance of the ideological positions of organisations which sponsor engagement opportunities, and how those organisations are able to influence the ideologies of the young. Erikson's studies also noted the increasing difficultly to acquire motivating belief systems if one is unable to do so during adolescence. 

With the process of forming ideologies, youth will often desire to create social change \citep{Juhasz1982}. Indeed, the processes by which youth are changing likely form the foundation for social change: 
\begin{quote}
“In this respect adolescence is a decisive turning-point - that at which the individual rejects, or at least revises his estimate of, everything that has been inculcated in him, and acquires a personal point of view and a personal place in life... the first duty of the modern adolescent is to revolt against all imposed truth and to build up his intellectual and moral ideals as freely as he can” \citep[][pp3-4]{Piaget1947}. 
\end{quote}
The natural tendency for youth to re-examine themselves and their surroundings can lead to them to reorient their values \citep[][]{Mead1947}. This process, according to \citet{Verlande2002}, influences the evolution of values in society and, ultimately, creates opportunity for youth to be at the forefront of social change. 

\citet{Erikson1968} echoes this sentiment, stating that as youth dedicate their energy and loyalty to causes which they find significant, they thereby develop their social identity and become regenerators of social evolution. He states: “... no longer is it merely for the old to teach the young the meaning of life. It is the young who, by their responses and actions, tell the old whether life as represented to them has some vital promise, and it is the young who carry in them the power to confirm those who confirm them, to renew and regenerate, to disavow what is rotten, to reform and rebel” \citep[][p258]{Erikson1968}. In essence, this process allows youth to express themselves through actions which spring forth from the ethical foundations of human nature, a process which is most effectively expressed between individuals and society, which subsequently provides meaning. The implications of this is that when engaging with youth, workers need to be open about their own ideology, not presenting themselves as impartial and objective. All have a vested interest in the future of their own community, hence youth workers need to also be committed in participating alongside youth in the critical discourses about the society to which we aspire, and how our own services and agency can, and can not, facilitate the realisation thereof \citep[][p52]{Shukra2012}. Understanding this, we can see the importance of further understanding around early adolescent community engagement, the possible advantages of engaging during this potential window, and how it influences lifelong engagement. Their experience of service and the roles they can embody in this domain are discussed in depth in next chapter. 
 

\section{Summary} 

The question of how young people understand their capacity to contribute to their community rests largely on how they see themselves, their capacity, and their role. To answer this question requires an exploration of many facets of the individual. The challenge however is that these areas are in flux during the adolescent years - the significant changes that occur in these years means that children who, for example, have big dreams to make the world a better place do not always become teenagers who believe in the same ideals. Instead, we must understand further the processes involved in development; how each area - identity, purpose, meaning, agency, motivation and ideology - helps young people understand their capacity to contribute to the community. This has also been a significant inquiry of existing academic research. 

As such, this chapter explored the key developmental milestones of adolescence, particularly early adolescence, and how this relates to the engagement in the community. Touching briefly on neuro-development, it explored how the emerging mental capacities give rise to new and critical ways of understanding themselves, their relationships and their world. The chapter then went into more depth into the sociological and psychological areas of development, covering concepts such as agency, identity, purpose, meaning, motivation and ideologies. Drawing on developmental psychologists such as \citet{Youniss1997b,Yates1998}, as well as sociologists such as \citet{Harris2010, White2008}, concepts such as agency can be understood with more nuance demonstrating how different perceptions of young people's can inflate or deflate their capacity, yet when agency is coupled with a sense of responsibility, it spurs young people towards being powerful social actors. Adolescence is a crucial time for the development of identity, purpose and meaning, and, when an individual is unable to develop these adequately, there are often lifelong consequences \citep{Flanagan2008}. 

The challenge here, however, is that engagement may be somewhat of a chicken and egg situation - those who engage in community activities likely develop their identity, agency, etc, in a way that supports the belief in their own capacity, while those who do not can be deprived of this belief. Similarly, as will be explored in the next chapter, those who engage will also derive further benefits such as improved academic performance and mental health as well as reduced anti-social behaviours. Questions remain about how to increase engagement - and subsequently the benefits to both individual and social development - of young people, and whether encouraging engagement earlier may be advantageous in this endeavour, and whether it will bring unique benefits. 


The period of youth is critical in laying the foundations for who they will become, and how they see themselves in the context of their community and society at large. By drawing from both fields, hopefully the reader is able to explore a more nuanced understanding of how internal processes are affected by the environment in which one exists, and also how individuals are able to contribute to their surroundings. Whilst this chapter focused primarily on individual processes, the next chapter will explore more deeply how the exigencies of engagement can contribute to broader processes. This is done to further understand how community members, youth workers and so on, can encourage engagement, and reap the benefits thereof. 


