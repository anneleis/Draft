\chapter{Conclusion}
% INEVITABILITY OF EQUALITY %USE?
Beck ponders on the words of Alexis de Tocqueville as he speaks of the inevitability of equality. With the beauty of hindsight, we can see that equality has come to the forefront of many political and social debates, but what is notable about this discussion is that equality is a force of its own, it has power not because we give it power, but because it eludes human interference. In this way, it has a somewhat religious prowess, a revolution which cannot be denied. Beck discusses democracy in the same manner. \citep{Beck2001} MY THOUGHTS - I would like to believe that community building has a similar force to it, the inevitability of continued development towards material and social prosperity which provides opportunities for both the individual and society. In this sense, we can either choose to be a part of the process (which is what many youth express a desire towards), or we can let it happen and still enjoy of the benefits (yet the benefits in this case may not match with our own personal preferences). Equality brings with a deeper level of questioning around social norms - what does it mean to be human? What is scientific truth? What are the priorities of economic growth? What is my role as a parent? Traditional explanations are losing their validity, and the connection from an increasing number of different communities and cultures, the social discourse around these topics is creating a level of anxiety, but also creates a level of debate which is more deliberate in its selection and establishment of practices. According to \citet[][p174]{Beck2001}, the age of equality "signifies not the end of difference but a universal struggle for its recognition". MY THOUGHTS - with this we see the concept of unity in diversity emerge.... 



%\chapter{1Intro}
%\chapter{2What Does It Take For Young People To Engage?}{Predictors of Engagement, Purpose and Identity}
%\section{Predictors of Engagement} 
%\subsection{Development related to community engagement eg purpose, etc - unsorted}

%\chapter{3Why Do They Engage?}
%\section{Mandatory Community Service}
%\section{Motivation}


%\chapter{4Why Should Young People Engage?}{Approaches to youth engagement}
%\section{Positive Youth Development}
%\section{Development and the Individual}
%\section{Capabilities as Development?}
%\section{Foundations for Development - ???}
%\section{Civic Engagement}
%\section{Fukuyama?}
%\section{Purpose}
%\section{Meaning}
%\section{Identity}


%\chapter{5What Do They Get Out Of Engagement?}{Benefits}
%\section{Benefits of Community Engagement}
%\section{Foundations Developed while Engaged with Community}

%\chapter{6Do Benefits Change at Different Ages?}{Early Adolescence}
%\section{Adolescence}
%\section{Behaviour}
%\section{The Teenage Brain}
%\section{Environment}
%\section{Brain Development, Motivations and Emotions}
%\section{Early Adolescence - unsorted}
%\section{Early adolescence community engagement - unsorted}

%\chapter{7What Are They Engaging In?}{}
%\section{Community Building, Participation and Engagement - whole section}

%\chapter{8How Are They Engaging?}{Participation, Agency}
%\section{Participation and Empowerment}

%\chapter{9How Should They Engage?}{}
%\section{Critical Theory and Paulo Friere}

%\chapter{10What Does Effective Engagement Look Like?}{}
%\section{On Being Critical - Investigating the relationship between Individual and Social Transofrmation}

