
\chapter{Development as Capacity Building and Systematic Learning}
\label{Relationship1}
\label{ConceptualFramework}

\begin{bclogo}[couleur=blue!30, arrondi=0.2, logo=\bclampe]{“Start a huge, foolish project, like Noah…it makes absolutely no difference what people think of you.”}
	- Rumi.	
	Never doubt that a small group of thoughtful, committed citizens can change the world; indeed, it's the only thing that ever has.
Margaret Mead
\end{bclogo}





\section{Objectives}

This framework has three, interrelated aspects: learning, capacity building and service. As many scholars have noted, one of the most effective ways to build capacity and empower populations is through education. As individuals learn, consciousness is raised and they open their eyes to new possibilities. While little is known about the processes involved, many scholars have noted the importance of education for the empowerment of individuals and populations. %INSERT RELEVANT REFERENCE, DETAILS OR QUOTE HERE
Education in this context must go beyond the scientific education of traditional schooling, and include the development of spiritual qualities. Developing ones sense of justice, their compassion and social responsibility, for instance, can only be achieved through interaction with others. Hence, capacity building in this framework is directed towards service to others; it's effectiveness for developing personal qualities is documented in vast amounts of research (see section \ref{Benefits} for a detailed discussion). Building capacity for the purpose of serving others not only develops spiritual qualities within the individual offering that service, but it also develops social connections and promotes a sense of community - both of which can bring significant benefits to all involved. This framework, however, promotes service in the context of broader social change, and for this to occur, service needs to be systematic. While one-off service initiatives may have benefits for the individual, those are limited and have little lasting impact on the community as a whole. Systematic service, however, in the context of creating lasting social change, requires service to methodical, respond to the needs of the community, and be responsive to changing capacities and needs. While the service offered should take into account the broader picture of social change, it should also be commensurate to the capacity of the individual offering. As an increasing number of individuals participate in the process of social change in a region, they also participate in the generation and dissemination of a growing body of knowledge, and subsequently contribute to the learning process for the benefit of themselves and others. In this way, recipients of development programs are not merely the recipients of theory and practice developed elsewhere, they are active contributors to the process. Hence, as individuals develop their capacity, the cycle of social and individual capacity building continues. 



\section{Outline for theoretical framework}

Karlberg's model - capacity building and systematic learning.
Not only did this model arise from one of its tenets - systematic learning - it is an integral aspect of an ongoing process. 
Capacity building in this model goes beyond many other models, to see the individual more holistically. Considering the social, material, physical, mental and spiritual aspects of an individuals existence, this model seeks to develop the entire being. This is done as an integrative approach in which various principles are explored in different ways. In a lesson about XYZ for example, there is a question about the equality of men and women. This allows participants to explore many aspects of their lives, and has been shown to result in deeper understandings around how gender and power influence their lives. 
Capacity building is also directed to service to the common good - building capacity for service. Hence, while they are developing capacity within themselves, the outcomes benefit the community as a whole. 
The other aspect of this framework is systematic learning - despite effectively delivering this program to an increasing number of individuals throughout South America for a number of years - and showing undeniable positive outcomes - this program continues to learn about becoming more effective, refining the process, and ensuring the learning continues beyond its current focus. This encourages students to continue their own capacity building beyond the course itself, building on their capacity for service by continuing their education to the point where they can run the program for others. Creating a systematic means to an ever-increasing pool of individuals in the process. %CHECK THIS LAST POINT'S ACCURACY! 


\section{Intro}

From the outset, I would like to address a number of possible criticisms of the theoretical approach I am about to explore. Firstly, it is easily assumed that community building is a process that is not relevant to developed countries, or perhaps that there are only certain communities within developed countries which it might apply to, and those are generally limited to low socioeconomic communities. This issue will be discussed in more detail in section \ref{FixMe}, \label{FixMe} %FIX THIS REFERENCE!!!
however this thesis assumes that all communities have the potential for further growth and development, and that in fact expanding our mindset to think beyond this will allow mutually beneficial learning between developed and developing countries, and greater gains overall. This point is particularly relevant given the example from which I draw my framework - a developing community in South America - however the principles which will be employed, namely capacity building and systematic learning, have been identified as important for development in all development work, and, as I will argue, should be considered essential in all settings. 

The second is an issue that is explored in some youth community engagement literature to varying degrees - the capacity of youth to meaningfully engage and contribute to community building processes. Given their developing capacity, and current lack of skills and abilities in some areas, it is understandable that there are reservations from the public and community organisations to encourage youth to be involved in community building activities, yet conversely we know that engaging them develops many pro-social capacities. Hence, this thesis is by no means suggesting that the best means to engage young people is by doing so in the same way as adults would engage. Engagement must be done in a way that allows them to gain knowledge and experience in a safe and conducive environment, and allow them to explore ways they see as important in contributing to their community. This may include engaging in other community organisations or not. In this way, young people will not only be able to develop their own capacities to the fullest extent, but it may also allow them to think beyond the possibilities offered to them by adult-centred community organisations. It is also because of this that I am not using frameworks which many other esteemed academics have used, such as developmental psychology or sociological perspectives. %WHAT IS THE WORD I AM LOOKING FOR? SOCIAL DEVELOPMENT? hmm



\section{The West versus the Rest}

There is an interesting relationship in the development field between Western societies and developing ones. I use the term Western here instead of developed since there are a number of countries which could be categorised as developed, it is primarily Western countries that have taken it upon themselves to `help' developing countries through aid and development programs. Additionally, Western countries have positioned themselves in a way that suggests they are the pinnacle of development, that while there is potential for improvement, that is merely a refinement of an already well-functioning society. 

I would argue that there are two fundamental issues with this thinking. The first is argued by an increasing number in the development field: that Western approaches in non-Western contexts do not bring the significant benefits that protagonists claim, and can in fact be detrimental to the focus regions. Ideals of individualism and training for employment, for example, have been inextricably linked with development and education in developing countries as a means to growth and progress, yet this is often taught in abstract, non-practical ways which do not have real-world applications, and paint traditional forms of education and lifestyle as backwards and ignorant. In Zambia for instance, this method resulted in masses of primary school aged children unable to continue to secondary education due to low marks, or in Ladakh where students felt ashamed of their cultural traditions and way of life \citep{Norberg-hodge2016}. The consequence is not only a lack of development, but a decreasing sense of culture alongside increased materialism and exploitation of the environment \citep{Norberg-hodge2016}. 

\index{CF} \index{Developing the West}
The second issue is that it implies there is no further development required in Western countries: these countries are materially prosperous and as such have nothing to learn from developing countries and no opportunity for further advancement. This idea of flawed in a number of ways (see for example, \cite{Karlberg2004} for a detailed analysis), however most relevant to this discussion is that of social solidarity and spiritual development. Social solidarity or social cohesion can also be used as a substitute for the term social capital, however social capital is often associated with concepts of market/economic gain \citep{Honeyman2010}, and as such the term social solidarity will be used here to mean “...a broad term encompassing the norms and networks facilitating collective action for mutual benefit” \citep[][p155]{Woolcock1998}. Spirituality, however, has become a forbidden word in many Western societies. As religion becomes less relevant to day to day life, spirituality has also been lost. In this context I use the word spirituality not to mean abstract or non-worldly beings, but a sense of morals and ethics that may or may not be linked to a higher being (which, for most of the worlds majority and a surprising number of the Western world, continues to be a force in their lives)  \citep{Arbab2000}. Spirituality, in this context, implies the application of principles into one's daily life, such as compassion, justice, unity and empowerment \citep{Arbab2000}.



%INSTEAD OF highlighted use the word underscored (sometimes)
 %IN a section entitled Theories about community, the second paragraph starts “Community can be viewed in different ways.” Eriksson2011
\label{Reflection}
\label{SocialForces}

\section{History of Development}
Alongside an increasingly globalised world there was an awareness about the inequalities between and within countries. This brought about an impetus for development targeting disadvantaged communities, primarily focused on communities with inadequate goods and services - economic wealth. This led many theorists and practitioners alike to take a “reductionist approach” to development which was “stripped to its material dimensions alone” \citep[][p481]{Goulet1980}. Based on a modernisation paradigm, industrialised countries were the model by which developing countries must aspire, creating a focus on entrepreneurship, agricultural development, health services and education - almost exclusively based on Western models: “In the development strategies that are propagated it is always the pursuit of material well-being, it is the socioeconomic component of development which has primacy. Underlying this bias are the European ideologies of social change and the cognitive systems which grew out of the industrial revolution and enthroned the economistic view of society and man.” \citep[][p481]{Goulet1980}. 

The early 1980's saw development experts recognising the limitations of economic-based approaches; determined largely by outside agencies, unfamiliar with the social reality and often excluding local and religious values \citep{Goulet1980}. These outside agencies would undertake development using a scientific approach, blind to the wisdom derived from cultural traditions which accounted for both the body and the soul, united communities, created meaning and sacredness \citep{Goulet1980}. 



\section{Evolution of thought in Development}

Development efforts, on the international front, came to the fore of people's mind after WWII, when the increased disparity was undeniable. Many countries began to designate a portion of their budget to international aid, convinced these efforts would eradicate poverty. Initial approaches used economic based programs based on Adam Smiths work, which in today's terms would be called trickle down economics \citep{Morrison2009}. 
The last few centuries saw a surge of new thinking around society and being human, much of this became central to the plethora of development theories. We begin with the work of Adam Smith, the Scottish moral philosopher and economist who, although not directly discussing issues of community development, paved the way for many of the economic based theories. \citeauthor{Smith1998}'s work assumed economic exchange based on mutual benefit, and his trickle down economics theory played a significant role in how societies generate wealth. In \citeauthor{Smith1998}'s thought emerges the notion that as an individual pursues their own interests, he is also promoting the interests of society, more so than when he intends to promote social good \citeyearpar[][par. IV.2.9]{Smith1998}. His rational, material view of humans was also the basis for the principles of freedom and equality that emerged as part of the Enlightenment which continued to argue for free markets and self-interest \citep{McKernan2013}. Enlightenment ideas form the foundation of human progress and development practices \citep{Schafer2012}. Development as an economic enterprise continued for decades, until the 1970's when it became clear that trickle down approaches rarely benefited the poor. Philosophers such as Karl Marx shed light on the exploitation of the working class, highlighting the social reproduction of inequality \citep{Fukuyama2011b}. Social progress for Marxists consisted of recognition of their own alienation, organisation and revolt against the controlling classes \citep{Vakil2001}. The culmination of these different approaches gives unique insight into effective development practices. The independent, rational thinking of individuals; the ineffectiveness of trickle down economics; the social forces that lead to inequalities - all demonstrate critical issues for consideration. %CAN ALSO move this sentence further down and include issues such as gender inequality, participation and capability development. 
This approach contained a number of assumptions: that economic growth would occur in similar stages to Western nations; the technological and resources assistance from industrialised nations would be a catalyst for this growth; and with the support from Western countries, leaders would sacrifice other values for the purposes of growth and modernisation \citep{Morrison2009}. As these efforts continued, it became increasingly obvious that this approach was ineffective, which subsequently gave rise to doubts about the focus on economics in general; poverty was beginning to be seen as merely a symptom of other systemic issues such as inequality \citep{Morrison2009}. The Basic Human Needs approach was the global response to these ineffective efforts, yet this often meant the mere re-labelling of existing initiatives \citep{Morrison2009}.

As experience and understanding continued to grow, the Human Development Index was considered more holistic, incorporating life expectancy, adult literacy and the ability for individuals to acquire basic needs. To reach these goals required an increase in funding from OECD countries - or as \citet[][p235]{Morrison2009} puts it, “the club of rich-country donors” - yet the reality was disappointing, with actual donations reducing (expected increase from 0.33\% to 0.7\% of GNI instead was met by a reduction to 0.22\%) \citep{Morrison2009}. Perhaps because of the decrease in funding, the United Nations 1994 report suggested the importance of local community members as a potential source of human resources. This became a central aspect of the World Development Goals which, ironically, were not devised with the participation of developing countries and ultimately reinforced the focus on poverty \citep{Morrison2009}. 

This continued inequality is echoed in the unguarded words of the World Bank's John Page when he identified these strategies as “a compulsory process wherein the people with the money tell the people who want the money what they need to do to get the money” \citep[][p147]{IMF2001}. With an equally critical eye on the donors, \citet[][p14]{Smillie2004} states that many such strategies “have been rushed; much of the content has been designed offshore; the participation of civil society has been weak ...; and the commitment of donors ... is patchy”. He continues with a focus on the deficiencies of the model itself: “The approach could... be described as a “babysitter model” rather than one genuinely based on ownership. Donors, usually in the form of the World Bank, hold the government’s hand while it creates a suitable [strategy].” This efforts did little to overcome the underlying issues of poverty, such as powerlessness, isolation and immobility, and for the most disadvantaged - the women, children and elderly for example - changes created during these initiatives could lead to disastrous consequences \citep{Morrison2009}. 

Through this realisation, poverty was recognised as more than mere material deprivation, and, according to \citet{White2013}, needed to consider physical health, education, environment, political and spiritual deprivation as varying forms of poverty. As both the causes - often a combination of systemic inequality, bias, disease, environmental - and effects - material deprivation, inability to convert income into well-being, etc - of poverty were increasingly recognised, the United Nations adopted the Human Development Index as a measure of well-being based on Sen's capability approach. Considered a more holistic approach to development, the capabilities approach takes into consideration social, demographic and medical information in addition to income \citep{Morrison2009}. Despite the Human Development Index not including many factors which Sen considers basic needs - security, justice, human rights - the capabilities approach looks beyond simple increasing in income and considers the substantive freedoms that an individual is capable of, as well as their ability to make choices they value \citep{Morrison2009, Sen2009}. 


This approach, however, has been criticised because it largely overlooks the influence of families and communities, focusing primarily on the capability failure of the individual \citep{Laderchi2003}. To eradicate poverty needs an understanding of more systemic issues and the unequal structures which create and perpetuate poverty, and needs to go beyond simplistic assumptions that “poverty is like dirt that can be sucked up by “the vacuum cleaner” of more aid. While aid is an important catalyst, making poverty history is much more complicated. The history of poverty demonstrates that highly unequal structures at all levels perpetuate and continue to create poverty” \citep[][p3]{CCIC2005}. Successful participatory methods require a micro-level of intense labour, and whilst these skills are not always available through aid workers \citep{Morrison2009}, external assistance is often necessary to empower a core group of locals who can continue to build resources from the local community.


\section{Prevailing thought on Development}


\subsection{World Development Report}
Despite the increasing recognition of the inefficacy of economic based models, this is often still at the core of many efforts \citep{Karlberg2016}. An `alternate' and globally recognised approach to development comes from the 2015 World Development Report. This report explores some of the challenges of development, identifying humans as not economically rational, and that thinking with mental models automatically and socially limit the potential for growth. The potential mental barriers that exist within every individual can, in this approach, be manipulated for the benefit of the individual: “Since every choice set is presented in one way or another, making the crucial aspects of the choice salient and making it cognitively less costly to arrive at the right decision (such as choosing the lowest-cost loan product, following a medical regimen, or investing for retirement) can help people make better decisions” \citep[][p38]{TheWorldBank2015}. As is evident, goals such as investing and low-cost loans are exhibited as the “right decision” for they lead to economic gains. Additionally, personal characteristics such as self-confidence is addressed as an important mental model which, if lacking, can limit opportunities: “A belief in a race-based or gender-based hierarchy can affect self-confidence in ways that create productivity differences that sustain the beliefs, although no underlying differences exist” \citep[][p69]{TheWorldBank2015}. A fundamental premise with this principle is that the systems and assumptions which perpetuate these injustices are not challenged \citep{Biccum2016}. The report appears to be a departure from traditional models of economic based development, however the strategies suggested and the desired results reinforce the importance of economic rationalism, providing financial and entrepreneurial behaviours as the desired \citep{Biccum2016}. 

Like many academics \citep{Biccum2016,Karlberg2016}, I have strong concerns around the assumption of an economic focused approaches to development for their appearance as naturally superior and developed countries as being the `saviours'. These approaches seem to assume that humans desire to be self-serving and not concerned with others (unless it directly or indirectly benefits them) or altruism, and they all seem to focus on the material aspects of life. 



%Revolution is often considered the antidote to capitalist structures \citep{Vakil2001} %SEE karlbergs discussion on the culture of protest 



%FOR PRINTING
 \begin{sidewaystable}
%FOR READING
 %\begin{landscape}

%FIX THIS TABLE - see https://www.sharelatex.com/learn/Tables
\begin{tabu} to 1.0 \textwidth {|| X | X | X | X | X ||} 
	\hline 
	\rule[-1ex]{0pt}{2.5ex} & Capacity Building & Capabilities & Capabilities Approach & Capabity Development \\ 
	\hline 
	\rule[-1ex]{0pt}{2.5ex} \begin{sideways} Usage \end{sideways} & Community Development, non-profits, WHO & Australian Curriculum, Public Service, & Amartya Sen, Human Development Index & UNDP \\ 
	\hline 
	\rule[-1ex]{0pt}{2.5ex} \begin{sideways} Definition \end{sideways} & “Capacity building has typically been defined as the development and strengthening of human and institutional resources.” 
	%“The United Nations Development Programme defines capacity as “the ability to perform functions, solve problems, and achieve objectives” at three levels: individual, institutional and societal.” 
	%“A process that supports only the initial stages of building or creating capacities and assumes that there are no existing capacities to start from.” 
	%“Capacity building is any support that strengthens an institution's ability to effectively and efficiently design, implement and evaluate development activities according to its mission (UNICEFNamibia,1996).”
	& “In the Australian Curriculum, capability encompasses knowledge, skills, behaviours and dispositions.” 
	& Based on freedoms which the individual chooses 
	& “Refers to the process through which individuals, organizations and societies obtain, strengthen and maintain the capabilities to set and achieve their own development objectives over time” \\ 
	\hline 
	\rule[-1ex]{0pt}{2.5ex} \begin{sideways} Distinctions \end{sideways} & Focuses on the process of growth and development & Focuses on outcomes chosen by institutions & Focuses on outcomes chosen by individual & . Approaches differ whether working with organizational, institutional,
	systems, and participatory. \\ 
	\hline 
	\rule[-1ex]{0pt}{2.5ex} \begin{sideways} Advantages \end{sideways} & Based on the principle that individuals should be provided “with the skills and tools they need to define problems and issues and formulate solutions” & & Places individual preferences at the centre of decision making process & \\ 
	\hline 
	\rule[-1ex]{0pt}{2.5ex} \begin{sideways} Criticisms \end{sideways} & Assumes no existing capacities, focuses on initial stages of development only (UNDP) & & Individusalistic - capability failures are individual based & Many varying definitions leads to lack of clarity around issues of long term efforts, power relations, purpose of development, and the role of technology \cite{Lusthaus1999} \\ 
	\hline 
	\rule[-1ex]{0pt}{2.5ex} 
	\begin{sideways} Notes \end{sideways} 
	& Often used interchangeably with capacity development 
	& 
	& 
	& Often used interchangeably with Capacity building \\ 
	\hline 
\end{tabu}
%SEE \citep[][p257]{Laderchi2003} for criticisms of Capabilities approach

%\end{landscape}
\end{sidewaystable}



\subsection{western approaches} to development are hegemonic, perpetuate power imbalances, and traditional knowledge systems
\subsection{post-development theories critiqued} for romanticising non-western cultures, not offering adequate alternatives to western development approaches
There is increasing criticism about the current educational system. This Western form of education, often based on the ideal of training people for employment, is increasingly being introduced into developing countries as a means to development. The consequences of this introduction are often far from beneficial, with examples world-wide of abstract, Western specific education resulting in skills and knowledge which is not relevant to the communities in which it is taught, and done so in a way which reflects traditional ways as backwards and ignorant. As a result, students either become failures by not attaining sufficient grades to continue onto secondary school, as was the case in Zambia, or they consider their culture inferior and ashamed of their way of life, as was the case with Ladakh \citep{Norberg-hodge2016}. It inadvertently teaches Western values of materialism and exploitation of the environment.
%WESTERN/DEVELOPED
Several operating principles were employed in this thesis... “Development is assumed to be a global challenge focused on more than the developing countries. Knowledge and empirical examples are therefore drawn from both the First and Third Worlds” [][p26]. **Developed and Developing, Western and non-Western countries!!! \citep{Vakil2001}        QUOTE FROM NON-PEER REVIEWED ARTICLE OF VAKIL - “In the “developed” countries, the problem is perhaps of a different nature. Communities are far less willing to acknowledge the underdevelopment in their midst. There is a perception that North American/European society is a kind of end-state, or point of arrival, and that it is the rest of the world that needs developing.”  ANOTHER SUCH QUOTE - “North America in particular faces serious development problems, which, similar to those of the developing countries, are a result of the colonial and post-colonial history of the region, such as the ... the horrendous treatment and persistent exploitation of the continent’s indigenous peoples... the steadily increasing poverty and violence in most of the urban areas, the lack of “world view” which has serious implications for the ability of Americans to perceive that they are a part of a global community, to name just a few. Surely these cannot be considered anomalies of what is otherwise a state of equilibrium, but are formidable development challenges which may, in the end, be much more difficult to surmount than those currently experienced in the “Third World.”    AND ANOTHER - “The key to creating a development agenda, one that is relevant to the nation, region, or locality in question, seems to be dialogue — a dialogue that is completely uninhibited. True dialogue leads to an ever-changing agenda, one which recognizes the real development context and is responsive to the needs of the time and place in question. Thus the agenda is not a fixed entity but a dynamic, changing one that needs to be re-created again and again.”
%DEVELOPED vs NON-DEVELOPED
Much of the development theory of the past century has almost exclusively focused on developing countries, and whilst some have extended this to apply to impoverished communities in developed countries, few have seen the implications for development in industrialised nations, assuming these to be already developed \citep{Vakil2001}.
%DEVELOPED vs NON-DEVELOPED
It is in many respects a methodological individualism, even though it proposes to account for culture, mental models and social networks, “requires a change in individual behaviour rather than changing the capacity for the individual to contribute to a change in the social order” \citep[p33-34]{Biccum2016}.
%INDIVIIDUALISM
Aggression and an individualistic self-serving human nature are often presumed to be the bedrock of civilisation, in particular Western civilisation. This is seen to form the competitive and self-serving society in which we find ourselves today, and to some, the cause of social progress (see Rose, Lewontin and Kamin 1987, p5, and Howell and Willis 1989, pp1-2). This may have been the case for many centuries, however Karlberg argues that this approach might have now reached a point of diminishing returns, and that a cooperative mutualistic interdependence is the best mode of creating social change. In this model, anger - the feeling behind aggression - can be considered a positive quality; it is a means by which injustice is identified and provides motivation to act determinedly and perseveringly to overcome these injustices. \citep{Karlberg2005}. Development, if conceptualised in the context of developed and developing countries, might be understood as increased material wealth, technological infrastructure or even the level of infrastructure... Increasingly the Human Development Index is recognised as an integral aspect of assessing the level of development in a country.....
%AGRESSION, ANGER, INDIVIDUALISM
Freidman's approach suggests overcoming poverty is the primary means to development, and takes a more holistic understanding incorporating the market and the state in his considerations of power at the political, social and psychological levels. His discussions however do not focus on a Western-style individualistic approach, but instead acknowledges that people are “moral beings” who have motivations beyond profit \citep{Vakil2001}. Friedman's theory has two basic shortcomings, acknowledged by Friedman himself. First his model is based on the household, and so does not account for the role of the individual [search for “black box” in Friedmans work]. Secondly, because of this there are significant limitations in communities with high numbers of single-person households, such as Western countries. These factors likely impact both the practical and theoretical implications of development \citep{Vakil2001} %SEE Friedmann, John. Empowerment: The Politics of Alternative Development. Cambridge, Mass.: Blackwell, 1992.
%DEVELOPMENT APPROACHE BEYOND ECONOMIC GAIN, INCLUDE WESTERN-INDIVIDUALISM, AND THAT PEOPLE ARE MORAL.




\section{United Nations Human Development Index, Capability Approach and Feminism}
%Comment on Page 5, on highlighted text “ There are other approaches to development that cut across this brief narrative. For instance, attention to the condition of girls and women has been growing for several decades now. Investments on this front, in the form of education and training, social assistance, microfinance and other empowerment strategies have proven to be an essential dimension of any successful development process (Martinez, 2012). Another approach is the capability approach articulated by Amartya Sen (1989, 1993, 1999), Martha Nussbaum (2000, 2011) and others, which has, among other things, been incorporated into the United Nations Human Development Index. This approach focuses on the development of underlying human capabilities that are requisites of economic prosperity, including substantive freedoms such as the ability to exercise personal agency in politi-cal and economic domains in pursuit of outcomes one values. The capability approach has, in turn, been criticised for its individualistic focus (Laderchi et al., 2003) as well as for the difficulty of operationalising the approach and of reaching agreement on the relative value of diverse capabilities (Clark, 2006). Other theories and approaches to development, too numerous to discuss here, have been documented elsewhere (e.g. Roberts et al., 2015; Haslam et al., 2012; Peet and Hartwick, 2009) “:
%Feminism, Capabilities (Amartya Sen and Nussbaum), Human development Index...  Gender and empwoerment has proven to be an essential aspect of successful development processes...

\subsection{Human Development}

The late 1970's saw a shift from economic-centric development approaches. In a 1976 report by the International Labour Organisation, the authors highlighted the ineffective nature of these approaches is that while economic growth was important, it was not sufficient to adequately address the issues of poverty, unemployment and disadvantage \citep{Vakil2001}. Additional strategies needed to include education, health care and improved employment. This new-found understanding sparked a focus on the “basic needs” approach to “human development” \citep{Vakil2001}. This diversion from economic development began to be addressed by theorists such as \citet{Haq1996} and \citet{Sen2006a,Sen1983} and became integral to the United Nations Human Development Index, but has yet to be integrated with development theory. It is however, indicative of some of the earlier integration of moral standards in development \citep{Vakil2001,Karlberg2016}. %JODY SAYS - What is `this'? Mentioned at least 3 times in this paragraph and unclear what I am referring to... (now 2?)





\subsection{Gender in development}
Cutting across most development efforts are issues such as gender and language. In a chapter underscoring the increased disparity arising from trickle down economic development efforts, \citet{Martinez2009} explores some of the dimensions of gender in development, ranging from social assistance to assist family well-being and combat child malnutrition to population control programs which offered surgical sterilisation as a means of family planning. The focus in gender has also gone through a number of phases: the \textit{Women In Development} approach demanded equality in economic, political and legal realms; while the \textit{Women And Development} - often confused by both agencies and laymen for the former approach - aimed to acknowledge the many invisible ways in which women already contribute to development efforts, such as child care, food preparation, etc; \textit{Gender And Development} which sought to understand the social, economic and political ways in which women continued to be disadvantaged and attempting to transform structures of power; and most recently \textit{Mainstreaming Gender Equality} which promotes full participation from both genders throughout all levels of development, community, programs, society and politics \citep{Martinez2009}. Despite her focus, she concludes offering three lessons which could be applied beyond the issues of gender specifically: she observes the lack of neutrality in development policy, theory and practice, identifying that all are subject to the effects of power relations and that failing to account for this only serves to reinforce inequalities; the importance of efforts to go beyond policies of gender equality to emphasis empowerment at the level of the individual, relationships, economics and sociopolitical; and finally she comments on the importance of integrating theory and practice \citep{Martinez2009}. 

\subsubsection{Feminism} %LINK THE FOLLOWING TWO SECTIONS

Feminism has also come to the fore in many development efforts, addressing a significant disadvantage globally. With an increasing awareness of the role of gender empowerment in development theories, critical insight into such theories demonstrated a focus on rational thinking, receptivity to new ideas, power and dominance. These are the hallmarks of progress and development and character traits traditionally attributed to men. In this context, women become invisible or treated as the determining factor of a country's backwardness \citep[][p280]{Hartwick2015}. The argument that men are endowed with rational thinking and women with emotional thinking reinforces gender inequality. This is by virtue of the assumption that problems are overcome through rational, and consequently male, thinking \citep[][p273]{Hartwick2015}. As more academic interest was shown in feminism, a greater understanding around the different challenges women both across and within different countries experience as a result of inequality become more evident. These differences were, initially, overlooked, yet as this movement gained momentum the differences became a source of empowerment \citep[][p276]{Hartwick2015}. Issues of gender and empowerment have proven to be essential aspects of successful development processes.

\section{Agriculture}
Another area that cuts across many development efforts is agriculture, largely due to the majority of the world's poor living in rural areas \citep{Ramisch2009}. Programs took the form of modernisation techniques, integrated rural development programs, participatory approaches and the `Green Revolution' in which ideas of rurality and crisis were synonymous, and that the poor were at fault for their poverty \citep{Ramisch2009}. Strategies also overlooked the diversity of lifestyles in which the world's rural poor lived, making rural and agricultural synonymous, and the homogenisation of rural populations \citep{Ramisch2009}.



\section{Language and Power}

In a chapter on post-development, \citet{Sahle2009} explores how language, discourse and power have influenced development efforts. Following WWII, many critics took it upon themselves to analyse the development efforts being undertaken globally. Increasingly patterns of depoliticisation and colonisation were being observed in which Western structures and disciplines were being introduced as the norm for progress and development. Depoliticisation - forms of language which legitimise colonisation while simultaneously silencing local response to colonisation - is one example of a social practice which resulted in the repression of alternate knowledge and perpetuates oppressive relations of power. A powerful form of representation, depoliticisation is an example of the creation of a narrative which represents the West as civilised and developing countries as savages. In other words, the discourse, knowledge production and circulation were used to legitimise the efforts of aid programs, their structures and disciplines, and present the ruling elite as apolitical or neutral, and hence desirable. This continued despite their efforts not achieving their desired goal \citep{Sahle2009}. 


The power dynamics underpinning this discourse is a key focus for post-development thinkers. For many, it is representative of Foucault's “Regime of Truth” (in other words, what counts as truth) in which hegemonic representations offer visions of Western lifestyle as the ideal, and hence the development processes which were experienced by the West as necessary steps for progress. Arguing along these lines \citet[][p39-40]{Escobar1995} states: “Discourse is the process through which social reality [political, cultural, economic] comes into being... the articulation of knowledge and power, of the visible and the expressible... the belief in the role of modernization as the only force capable of destroying archaic superstitions and relations, at whatever social, cultural, and political cost. Industrialization and urbanization were seen as the inevitable and necessary progressive routes to modernization. Only through material advancement could social, cultural and political progress be achieved. This view determined the belief that capital investment was the most important ingredient in economic growth and development.... Moreover, it was absolutely necessary that governments and international organizations take an active role in promoting and orchestrating the necessary efforts to overcome general backwardness ad economic underdevelopment”. This results in both the privileged and disadvantaged believing the processes are necessary and the continued repression of alternate forms of knowledge \citep{Sahle2009}. 








\section{Critical Pedagogy} 

Development as an economic quest couched as modernisation is “simply one possibility, among many, of development in a broader, more critical, sense” \citep[][p482, quoting his 1971 article]{Goulet1980} and, indeed, many scholars understood this and applied it to their work. Of the many community development approaches, one which has both come from, and contributed to disadvantaged communities is \citet{Freire2000}'s critical pedagogy. Experiencing the great depression of the 1930s and immersed in poor communities, Freire personal understanding of the plight of disadvantaged communities led him to develop the field of education, and specifically the critical education which allowed these communities to understand the underlying social forces causing their oppression. Using terms such as “educand” - a term who denotes the existing knowledge of students and their ability to contribute to the educational process - and “praxis” - the ongoing process of reflection and action aimed at social change and continually advancing understanding - he flipped the script on traditional forms of knowledge production and dissemination which reinforced social class, and built on existing knowledge that disadvantaged populations already possessed. The result was an increase in awareness of social forces shaping their lives, and their resulting individual and collective empowerment \citep{Freire1969,Freire2000}.








\section{Education and Service Learning} 
With an extensive number of benefits arising from community service, various attempts to maximise student participation have been tried in different educational contexts. 
One common way for community service programs to become more universal is to integrate them into school education; these curricula are often called service learning. Education has long served as a means to link citizens with social order. Traditional forms of education did so as a means to subdue deviancy % [see himmelfarb 1984, stoler 1995, hall 2008] \citep{Biccum2016}
while higher education, in its earlier years, was to inculcate the ability to become a good citizen, and to imbue young people with the skills for public service \citep{London2000}. 
For many American students, higher education served as a means to a lucrative career and ultimately to increase their earning potential \citep{Karlberg2005}. %BUT REALLY (Loeb 1994)
While in many contexts historically, education has served as a means for class reproduction, increasingly it is also serves as a platform for political activism and mobilisation, social change and empowerment \citep{Freire2000,Shultz2007}.


More recently however, higher education shifted in focus to “professionalize, vocationalize, and specialise in a manner that occludes its civic and democratic mission” \citep[][pxi]{Barber1997}. In this context, service to others is:
\blockquote{not just a form of do-goodism or feel-goodism. It is a road to social responsibility and citizenship. When linked closely to classroom learning... it offers an ideal setting for bridging the gap between the classroom and the street, between theory of democracy and its much more obstreperous practice... Service is an instrument of civic pedagogy \citep[][pxiii]{Barber1997}}
As this thought emerged, service learning programs began to emerge in schools and colleges. 
Beginning in college campuses in the US in the 1960's, service learning programs primarily took the form of college work-study programs \citep{Karlberg2005}.
Subsequently, there has been a significant academic interest in these programs, resulting in somewhat of a divergence around what service learning actually means. 

For some \citep{Furco1996,Eyler1999}, service-learning was about the application of academic knowledge towards practical challenges in the wider community, what some have referred to as experiential learning. Others have understood it as the acquisition of practical job skills in a corporate setting, similar to that of an internship \citep{Colby2003}. Alternatively, some associate the term service learning with citizenship training and volunteerism - advocating the importance of time and energy spent outside the corporate sector \citep{GilesJnr1994, Smith1994a} and some see it as a means to create social change \citep{Pollock1999}. Corporation based service-learning often had definitions which focused on serving the interests of the individual, not others 
\citep{Karlberg2005}. Based on the latter two definitions, service-learning will be explored as an experiential pedagogy which cultivates an attitude of service to others and an awareness of mutualistic interdependence \citep{Karlberg2005}. 
The link between undertaking service to others and personal development is encapsulated well in the words of \citet[][]{Biccum2016}: “Continual transformation and betterment of the self through education contributes to continual transformation and betterment of the social order.”



\section{Post development critics and the Development of `Developed' Countries}
Many post-development theorists have strongly criticised development approaches as the imposition of Western theoretical musings which suppress alternate or local knowledge and push a neo-liberal ideology. Presenting modern Western lifestyle as the desirable end-point of development, post-development theorists point to the perpetuation of existing power regimes as a shortfall of most approaches \citep{Sahle2009}. Inherent in this approach is also the assumption that Western countries no longer need to continue their development - at least not in the social sense. As noted by \citet{Vakil2001}, development is a global enterprise undertaken by developed countries for the benefit of developing, with the transfer of knowledge and resources from the empowered to those in need of empowerment. This is echoed in the World Development Report. Despite its claim to account for culture, mental models and social networks, it is, in essence, a form of methodological individualism, and in the words of \citet[][p33-34]{Biccum2016} seeks “a change in individual behaviour rather than changing the capacity for the individual to contribute to a change in the social order” 

This is arguably a key downfall in the assumption mentioned above - that of Western countries not needing further development. Individualism and aggression, for instance, are claimed to be the bedrock of modern civilisation, particularly Western civilisations. This forms the basis of a competitive, self-serving society which, to SOME % (see Rose, Lewontin and Kamin 1987, p5, and Howell and Willis 1989, pp1-2) according to \cite{Karlberg2005}
, is the cause of social progress. Yet this can also be understood as one of the potential arenas in which Western countries are in need of further development. It is this competition and self-serving attitude which perpetuates and exaggerates inequalities, steadily increases urban violence and poverty, and continues the atrocious treatment and exploitation of the world's indigenous \citep{Vakil1987}. \citet{Vakil1987} continues: “Surely these cannot be considered anomalies of what is otherwise a state of equilibrium, but are formidable development challenges which may, in the end, be much more difficult to surmount than those currently experienced in the “Third World”.” This suggests the need to constantly examine and re-examine the discourse surrounding development, the limitations and implications, and recognise that “Authentic development aims at the full realization of human capabilities” \citep[][p482, quoting his 1971 article]{Goulet1980}, which recognises the importance of continued effort for all individuals, in each and every community.



\section{Why this framework and why JY?}


Much of the present development thinking warns against outsiders coming into a community and imposing development strategies or undertaking the bulk of development work themselves. Instead, there is an emphasis on having the local population not only decide on the nature of development work being undertaken, but also for them to participate in the efforts of social change. In a slight variation on this approach, the theoretical framework I am going to employ will not draw directly from well-known developmental theorists - although concepts from some will be employed in relation to the theory being presented. Instead, I will use the principles from relatively successful development efforts that have began in Norte del Cauca, Colombia, but are now being used in countries in Asia and Africa. %CHECK THIS!

Sistema de Aprendizaje Tutorial (the Tutorial Learning System or SAT) is a development program that began in 1983 with a goal to empower the local population, but has taken a somewhat different approach to many other development programs. SAT focuses on the spiritual nature of every individual in the program and uses this as a means to motivate individuals to act for the greater good. One of the fundamental spiritual teaches that SAT focuses on is unity. Considered as the mutual interdependence of society, SAT teaches its students how society is like a human body, each cell has a unique role and function, and survives through reciprocal giving and taking of various elements in such a way for the entire body to function well. By using this approach, SAT students understand the importance of helping and empowering others, resulting in a strong sense of responsibility to those around them. 

This approach stands in stark contrast to similar development programs in the region. In a study comparing SAT to another community development program typical of such programs in Columbia, \citet{Honeyman2010} discussed CB's focus on individual rights. In this program, as in many others, individual rights were presented as central to freedom and development, and subsequently individuals as the primary concern. The relationship between the individual and society in this approach is one in which the individual needs to make certain concessions for society, individuals need to sacrifice in order to not infringe on the rights of others. In other words, to respect the rights of others is a necessary compromise that each individual needs to consent to in order to gain their own freedom. The SAT program however, talks of the relationship between the individual and society as one of mutualistic interdependence - assisting others and the community is a source of joy as the individual is one aspect of the whole and as one grows it benefits the whole. 

The organisation responsible for the SAT program, Fundación para la Aplicación y Enseñanza de las Ciencias (FUNDAEC), saw systematic learning as a key aspect not only for the students, but for the community, institutions and themselves. As such, the program continues to evolve and respond to the needs of the community. Learning, however, is a key aspect of decision making, for in this approach decision making is understood as an investigation of reality. %LEARNING AS CONCEPT BASED SO THAT INDIVIDUAL CAN SEE THE CONNECTIONS BETWEEN, INSTEAD OF SUBJECT BASED WHERE LEARNING IS FRAGMENTED

Education is obviously a key aspect of development, particularly a development framework that places learning in all arenas as a central tenet. The SAT program understands the importance of education for children and young people being specific to their emerging capacities. \citet[][pp224-5]{Arbab2000}, one of the founding members of FUNDAEC, talks about the need for education to prepare students for the challenges in their next stage of life. For children, education needs to prepare them for the challenges of adolescence:
\begin{quote}
	the realization that it is chiefly service to humanity and dedication to the unification of humankind that release creative powers latent in one's nature; the understanding that not only knowledge of principles but the exercise and application of will is essential to both personal growth and social change; a conviction that honour and happiness lie not in the pursuit of wealth and power for their own sake, but in self-respect and noble purposes, in integrity and moral quality; and a disposition to analyze and a desire to understand the features of different forms of government, law, and public administration. To these must be added other attributes that enhance social effectiveness: an adequate understanding, at least in the local context, of the concerns of programs of social progress in such areas as health and sanitation, agriculture, crafts, and industry; some development of the power of intellectual investigation as an instrument of successful individual and collective action; certain ability to analyze social conditions and discover the forces that have caused them; the corresponding ability to express ideas and to contribute to consultation on community problems; the capacity to take part in community action as a determined yet humble participant who helps overcome conflict and division and contributes to the establishment of a spirit of unity and collaboration; and a reasonable degree of excellence in at least one productive skill through which to experience the truth that work is worship when performed in a spirit of service.
\end{quote}
He here highlights the importance of children learning about the oneness of humanity, developing both practical and social capacities, and the ability to analyse basic aspects of the society. What is interesting, however, is that he suggests a clear distinction between childhood and adolescent outcomes. Building on the fundamentals learnt in childhood, he states that their high school education needs to cater to the transition from childhood to adulthood they are experiencing:
\begin{quote}
	the transition calls for a qualitative change, particularly in terms of scientific rigour, use of language, and social content, for it is in this stage of education that vague hopes and ideals regarding one's future and service to humanity must crystallize into the twofold moral purpose mentioned above. The student must now become a purposeful agent in charge of his or her own education. Every effort needs to be made to raise the student's consciousness to a higher level — a consciousness of the ramifications of personal choices being made, of the social forces to which one's community is subjected, and of the nature of the historical processes in which one is immersed.
\end{quote}
Suggested in these words, at least to me, is that the period of adolescence is a time when the individual is able to effectively think beyond themselves, and make a meaningful contribution to the lives of others. 


\section{Capabilities Approach versus SAT}

The Capabilities Approach (CA) gives important insights into the development of capacity, and leaves the “capacity of what?” question to the individual. This is arguably a means to empowerment and independence, and while empowerment and independence are crucial aspects of any development approach, Sen somewhat overlooks the importance of community and institutions on individual development. He certainly does not omit these considerations from his discussions, however his approach is lacking in how these social aspects can be changed to support the development of the individual. 

The SAT approach however is more targeted and specific, specifically focusing on capacity building for service. This approach incorporates developing an attitude of service to the community, resulting in efforts to improve the conditions of the whole. In this way, developing capacity for service not only empowers the individual, it leads to an understanding of mutualistic interdependence, which is, ultimately, how society functions - each individual performing a function which complements that of others. 

While the CA framework is important for how it acknowledges the unique interests and abilities of the individual, and their right to choose how those are actualised, the SAT program considers how those interests and abilities can serve the interests of the whole, resulting in an improvement to the entire life of the community, not merely the individual.  


\newpage
\subsection{Critical Pedagogy and SAT}

Many scholars have likened some aspects of the SAT model such as the relationship between tutors and students, the dialogic, problem posing methodology, and the integration of spiritual qualities to Freire's critical pedagogy \citep[see for example][]{Murphy-Graham2010,Murphy-Graham2014,Roosta2002,Murphy-Graham2012,VanderDussen2009}. For example, tutors in the SAT program are unique to many other development programs in a two key ways. The first is that, since the program is concerned with developing the capacity of the local population, this philosophy extends to the tutors of its programs. As such, tutors are hired from the local population, resulting in an intimate familiarity with the community, the social forces impacting the participants, and - since tutors are provided initial and ongoing training - another aspect of capacity building within the community. During their training sessions, tutors are also encouraged to examine their own beliefs around topics such as the equality of men and women, allowing them to not only explore the themes but be an example \citep{Graham-Murphy2009}. This is also a demonstration of the co-learning that tutors undertake alongside the students \citep{Murphy-Graham2014}, and one example of the broader theory concerning the attitude of the tutors, this being the second unique aspect of the role of tutors in the SAT program.

FUNDAEC is very aware of existing power dynamics in society, and aims to empower the entire population without reinforcing existing power imbalances. Whilst this is a delicate balance to create, they have had some success through training which explores the role of the facilitator in relation to the whole \citep{Murphy-Graham2012}. In the first books of the SAT program, as discussed above, the human body is used as an analogy to society - each cell has its own role and each has its own responsibility to the whole for the individual to function properly, to have a sense of well-being and to reach its potential. Tutors of this program see themselves as contributing to the well-being of the entire community as a form of service; they are not generally treated with any more (or less) respect than anyone else in the community \citep{Honeyman2010}. 



This relationship between tutor and student can also been likened to Freire's relationship between the educator and the educand \citep{Freire1995}. The educand is a term which Freire uses to denote the existing knowledge and capacity with which individuals come to the educational process - they are not empty vessels which need to be filled, but mines filled with gems that can bring value to the whole. In Freire's critical pedagogy the educator assists the educand to see their own capacity and bring about meaningful change in their own life as a result \citep{Freire1995}. It also, like SAT, acknowledges that while educators can assist others, this process is also beneficial for their own development, as educators are continuing along their own developmental path as they educate \citep{Murphy-Graham2014,Freire1995}. 


Closely linked to this is the method by which tutors elicit those gems. Both the tutors and the curriculum approach learning as an exploration of problems to be solved. For Freire, the problem posing approach develops peoples critical understanding of the world and their relationship to it; they understand the world as the continual transformational process of reality. This understanding gives rise to an organic process of thought and action - praxis \cite{Freire2000}. Problem-posing education acknowledges that humans are unfinished in their growth, and aware of this state. As unfinished, humans affirm the necessity of education as an ongoing process \citep{Freire2000}. In the SAT program tutors encourage discussion by posing questions and helping them find the answers. In this way, tutors encourage students to link their own reality to the lessons in the text \citep{Murphy-Graham2009}.

The praxis methodology used in critical pedagogy is comparable to SAT's model. In Freire's model of praxis, reflection and action become cyclical through which society is transformed:
\begin{quote}
"An unauthentic word, one which is unable to transform reality, results when dichotomy is imposed upon its constitutive elements. When a word is deprived of its dimension of action, reflection automatically suffers as well; and the word is changed into idle chatter, into verbalism, into an alienated and alienating “blah.” It becomes an empty word, one which cannot denounce the world, for denunciation is impossible without a commitment to transform, and there is no transformation without action." \citep[][p87]{Freire2000}
\end{quote}
The SAT model also integrates practical actions as part of the learning process, and those actions are designed to directly benefit the community while simultaneously challenging existing social structures \citep{VanderDussen2009}. 


Finally, both Freire's model and SAT also explore the social forces that exist in society, and by doing so examine existing power imbalances. This process ensures that power dynamics that have, in the past, led to inequality continue to be recognised and examined such that future interactions are less affected \citep{Arbab1988,Freire1969}. In short, both view education as a means to social transformation \citep{Kwauk2016}, incorporate action and problem solving approaches into basic educational practices, understand the role of the tutor not as someone with an abundance of knowledge but someone who can encourage them to develop their inner qualities and by doing so challenge pervading power imbalances that exist throughout society. 
\newpage













\section{Development as Capacity Building and Systematic Learning versus other frameworks}



\citeauthor{Karlberg2016}'s framework, development as capacity building and systematic learning, is distinct from many both in the community development discourse as well as the youth development/engagement discourses. The next few paragraphs will explore five reasons why this framework is useful in comparison to other, more prominent frameworks. First, many frameworks that explore developing youth capacity in relation to service to their community come from youth development/engagement discourses, which can be explored from either a psychological or sociological perspective. This framework was chosen in part because the youth development/engagement discourses are largely from a developmental psychology perspective which overlook the role of community and society in shaping the development of young people (although there is a recognition that this happens). There is significant research which evidences the benefits of engaging young people in the community, whether as a participant in community groups or undertaking service for the benefit of others. Much of this literature, however, centres on individual experience. Alternate sociology perspectives account for the environment, yet perhaps places too little emphasis on individual volition and action, suggesting that the individual is largely futile in changing the social circumstance. The framework under investigation looks at how young people can work with their peers, community members and institutions to create lasting, meaningful change which further supports the development of the individual. It also focuses on how the individual can build a sense of social responsibility, which contributes to a sense of trust, belonging and social cohesion - all fundamental aspects of wider social development.

The second reason this framework was chosen reflects its distinction from community development, or simply development frameworks. Many of the development discourses focus on human rights, and while this is a very crucial outcome of development, some studies have suggested that focusing on this aspect does little to develop the community as a whole. A focus on human rights places the individual at the centre of the relationship between the individual and society. Alternatively, \citeauthor{Karlberg2016}'s framework places the individual as merely one element of a society, and for the society to maintain a level of well-being and to have the possibility of progressing, each individual element of that society has to both contribute to, and receive from the whole. Efforts that emphasise rights often result in empowering a small group of individuals who will often come to be seen as `elite' in the community, and can reinforce existing structures and institutions and amplify inequalities \citep[see for example][]{Honeyman2010}. 

Third, this framework looks at capacity building specific to service to the community. Inherent in this approach is an acknowledgement of the whole, an understanding of the reality of mutualistic interdependence, and a willingness to sacrifice for others. Mutualistic interdependence is linked to the above concept of being one element of a society - interdependence implies both giving and receiving from others as a means to progress. It sheds traditional concepts of education as competitive and exam oriented, and encourages each student to be concerned with the advancement of his peers. The result is a clear increase in the knowledge of all. 

Fourth, this framework is embedded in a process of learning. The implications of this are vast, however it echoes many approaches emphasis on the importance of learning as key to development. What is particular to this framework however is that it integrates spiritual concepts into the curriculum. Virtues such as humility, compassion and love are discussed by prominent theorists such as Freire, however this framework integrates them in two particular ways. On the one hand, it aims to teach these virtues directly, instead of assuming that individuals will learn this through everyday interactions or from parents and caregivers. On the other hand it does not treat these concepts as a distinct area of learning, it integrates them into every aspect of the curriculum, which allows issues of inequality, injustice and the like to be intricately linked with scientific understanding. 

Finally, an integral aspect of this framework is systematic learning at all levels. Many theorists, perhaps coming from experience or observation and critical insight, offer a framework that they believe provides insights into existing realities. While many refine these ideas over time, often in response to academic criticism, learning is rarely an ongoing aspect that arises from, and informs practical action. Learning is, obviously, an integral aspect of education, and for this reason the SAT program has become a recognised educational experience for thousands who would not otherwise have access to secondary education. Additionally however, every level of the program has an intrinsic learning aspect. The tutors are distinct from teachers in the traditional sense for they do not teach students, they guide them through the materials and continue their own learning alongside them. This results in students taking responsibility for their own learning and that of peers. Tutors continue their learning experience through regular training and reflection with regional advisers. Intense courses of up to three weeks are held for groups of tutors throughout the region, and regional advisers keep in regular contact with the tutors, particular is issues arise. On an institutional level, FUNDAEC works with both the regional coordinators and delivering institutions and organisations to ensure the program and its materials are relevant and up to date. The community shares in the learning experience also, with service projects being an integral aspect in which students consult with members of the community about various aspects of their service project. This results in the continued production and sharing of community knowledge. This level of learning is unique to the development programs I have encountered, and provides a level of flexibility and responsiveness that has enabled the program to be implemented in an increasing number of countries worldwide. 

This capacity for systematic learning, I feel, is missing in many youth programs in both developing and developed countries. In many contexts youth workers are trained through tertiary education and sent into the community where there is little ongoing training or support, and from which learning is not generated. This means that when youth workers or community development practitioners do learn about effective strategies, this learned is not shared with others and the practice as a whole does not advance. FUNDAEC on the other hand, has provided an environment in which learning is cultivated at every level, and that learning fosters community wide social responsibility and development.




\section{Outline of Conceptual Framework}
\subsection{Context for developing SAT}
The SAT program was developed by FUNDAEC in the 1970's in response to a growing concern for the lack of rural education in Colombia. At the time, the Colombian government had made great strides in implementing universal primary education to the majority of its citizen. There was a recognition, however, that for the rural poor to bring themselves out of poverty, they would need further, secondary education. This was challenging for the Government in two primary ways. First, the cost of establishing secondary education in rural areas was prohibitive, with numbers from rural areas being lower than urban, making the cost of establishing schools in each rural area high, and the cost and reduced human resource implications of establishing boarding schools for larger catchment areas unappealing to the rural population. Secondly, the schools that were being established had the same curriculum throughout the country, resulting in education not suitable for rural populations. 

The result of a national initiative, FUNDAEC began not with an established framework for an educational enterprise or as a community development initiative, while these are a few of the outcomes. FUNDAEC began as a small number of professionals working alongside community members, attempting to translate a few key concepts into social reality. They knew that learning the sciences was critical for helping the community to emerge from poverty, yet they were also conscious of the importance of embracing the spiritual nature of man. They knew that whilst the Western world was becoming increasingly cautious of religion, the worlds majority continued to practice various forms of religion. Additionally, while they were conscious not to make the program a form of religious indoctrination, many of the spiritual concepts throughout the religious texts were universal, and could be used as principles throughout the materials. They believed that learning spiritual values alongside scientific understanding could lead to a more just application of science, relevant to the needs of the community. 








\subsection{Conceptual Aspects}
Since they came from various walks of life, they understood this program to be a process of learning. With this attitude, FUNDAEC was able to refine (and continue to do so) the program alongside community members and tutors, to ensure that it suited the needs of the community and led to the empowerment of its members. This process of learning allowed them to understand the importance of developing in their students a critical understanding of their surroundings - an ability to analyse the social forces impacting their lives and recognise their capacity to respond and contribute accordingly. What emerged from the development of this capacity was a critical assessment of the role of program facilitators, which FUNDAEC calls tutors. FUNDAEC recognised that, for communities to overcome unequal power relations, the dynamics of the teacher and student, and between students, must change. Thus, the model that emerged was that of a group of participants and a tutor who were all concerned with the development of each individual in the group, including the tutor. The tutor took on a very unique role, hence the term tutor was used instead of teacher. This term signified that the individual did not have all the answers, but that they were there to guide participants through the texts, while continuing their own educational experience themselves. The tutors role was to facilitate an environment in which every individual felt they could share, ask questions and be active protagonists in their own learning experience and that of their peers. This required the students to take responsibility for their own learning, and rid the group of any hint of competition. 

\subsection{Pedagogic Approach}

The FUNDAEC model considers education as more than just the teaching of materials, it is a means for social change. As such, the program goes beyond educating the individual in the sciences. The model consists of five central aspects of the program: texts books and practice; tutors; study groups; and accompanying institutions. At the writing of this paper, the SAT curriculum comprised of 80 texts books that were to be studied over the 6 years of secondary school. While these text books have grown since its inception, within the first few years of praxis, FUNDAEC realised that the books fell into five categories: science, mathematics language, technology and service to the community \citep{Arbab1988}. The classes run for between 20 and 25 hours per week to accommodate the various responsibilities students have, and texts from multiple categories are often studied simultaneously. The text provide regular opportunity for practical outcomes, and they are designed such that they are relevant to the population. While there is significant flexibility with the program in general, FUNDAEC do not allow for accompanying institutions to change the content of the books even though they are quite responsive to expressed needs for changes and new materials in a particular area. resulting in the existing 80 text books available today. FUNDAEC maintains copyright and distribution rights which support the ongoing costs of the program \citep{Kwauk2016}.

Tutors, as mentioned above, are distinct from teachers for they are not responsible for teaching the students, but instead guide them through the materials in a dialogic fashion, continuing their own learning at the same time. Tutors attend regular training which prepares them for the accompaniment of students through the creation of an atmosphere of learning in which students take responsibility for their own learning. Central to the effectiveness of this program for social transformation is that tutors are from the local area. This can create challenges for FUNDAEC, for while they are not always college graduates tutors must at least be attending college, which can limit the pool of potential tutors significantly in areas where the average person has only completed primary school. However despite this challenge, tutors are offered regular training in the FUNDAEC materials and reflect with regional coordinators. And because the tutors are from the local area, they are known and respected by the community and are invested in its progress, and the progress of its members. Instead of getting new tachers/tutors each year, tutors are expected to continue with the same group throughout the entirety of the program.

Small groups of students (the size can often depend on the country or community, but often around 20 students to a tutor) gather in local community buildings or in school rooms after normal school hours. These groups continue with each other through the entire secondary education, and undertake their study in a cooperative, dialogic manner. In many school environments in which this program takes place, education is imbued with competition and individual accomplishment. The FUNDAEC program, however, encourages students to think of themselves, their study group and their community as have mutualistic interdependent relationships which result in an atmosphere of cooperation and mutual assistance. The students are concerned with the progress of all, and as such if one person does not understand the materials, then all must assist him to do so. Only then does the class move on to the next lesson. This is made possible through two factors: the tutor has a level of flexibility with the group that allows them to change the pace at which the group goes through the material when needed, and students feel comfortable to ask questions and express their doubts. In contrast to traditional Latin American schools in which students who ask questions are either humiliated or exposed to physical punishment, the environment which FUNDAEC results in marks averaging 25\% higher than traditional forms of education. 

Communities are also a central aspect of the program. Since there is a strong practical component, the students are required to reach out to community members and local organisations and businesses when undertaking projects. The impact on the community comes about in two ways. Firstly, students are required to engage community members in many projects and draw on their experience, expertise and resources. For example some units require students to undertake surveys with community members, or talk to local farmers about appropriate crops for the local area. Secondly, the outcomes of many projects are designed to directly benefit the community. For example the surveys mentioned above feed into public health campaigns, while the combination of local knowledge and global knowledge leads to more productive crops resulting from technology suited to the local conditions. The community supports the group while they learn more effective and efficient means to better the conditions of the community. 

Finally, FUNDAEC works with a number of accompanying institutions. One of the challenges in expanding the program was the resources and support required to do so. As such, FUNDAEC works with a range of government agencies, business and NGO's to implement the program in an increasingly diverse number of communities. FUNDAEC supplies the materials and offers initial training to the institutions on the methods and methodology of the program, and provides ongoing support to maintain program quality. The accompanying institutions are responsible for the establishment and ongoing maintenance of the program in its local areas. This arrangement allows for significant opportunity for flexible implementation of the program. In Honduras for instance, one of the accompanying institutions is the Government. The official recognition of the program as an alternate secondary education, and subsequently their willingness to financially support the tutors, provides a level of credibility for the program and has allowed it to expand to into a number of new regions in the country. While accompanying institutions are important for the expansion of the program, FUNDAEC maintains a level of connection with the program that allows it to continue generating knowledge and insights from its various tutors and continue the learning process. 



\subsection{Educational Hypothesis}
FUNDAEC had fundamental beliefs about human nature which informed their approach. The group did not intend on establishing an educational program, indeed they were concerned with the spiritual and economic advancement of rural communities. They found quite early on however, that a radically different approach to the education, primarily of young people, was an effective means to achieve this. Education, however, never comes without inherent values - despite many institutions not stating them - and FUNDAEC was very explicit about those from the beginning. FUNDAEC began with the belief about the importance of teaching scientific knowledge alongside spiritual, with the assumption that spiritual knowledge and insight will advance the human condition. While many were understandably moving away from religion and spiritual principles at the time, FUNDAEC saw value in understanding the spiritual nature of man and deliberately developing spiritual qualities as part of the educational process. This required explicit attention to issues of justice, compassion, unity and the like. 

Unity, for example, is one of the fundamental aspects of the SAT program, and its use significantly re-frames the curriculum from one of human rights, which is the focus of many other similar programs. With this concept, the relationship between the individual and society is explored with the individual not at the centre and society as a potential impediment to our own rights, but that each of us is both the recipient of and contributor to the well-being of the whole, and only through this relationship can our own well-being be maintained. This leads directly into the next educational hypothesis - that education should lead to practical skills and knowledge. 

The importance of theory and action complementing each other is the focus of much academic understanding, yet ironically this is rarely seen in many educational institutions. The SAT program not only considers practical applications of the educational content an integral aspect of the program, it uses this aspect as an opportunity for the group to serve their community. In their lessons on biology, for instance, the students are required to undertake a public health campaign that is suited to the community in which they reside. By doing so, these students not only learn about living organisms, they also understand its practical applications in a way which benefits the entire community, learning social responsibility in the process. 

This approach necessitates an integration of materials throughout the program. In many educational enterprises, subjects are taught as distinct areas of knowledge. The result is a fragmented understanding of reality and a superficial understanding of reality: “In most cases the result of such education is the fragmentation of the student’s mind and its final outcome is compliance with the social and spiritual vacuum that characterizes present-day society” \citep[][p9]{Arbab2007}. This is particularly problematic for rural youth, as their reality necessitates an integration of knowledge. As such, SAT explores concepts and develops capabilities instead of studying subjects, themes are presented through which several concepts are woven together in real world contexts. This allows a significantly greater means to integrate not only relevant scientific knowledge, but it also allows the integration of spiritual principles. For example, while studying the text `Systems and Processes' from the `Reading and Understanding Descriptions' section of the curriculum, they explore the gastrointestinal system and the process by which food is digested. This section also includes a discussion about how our mind processes ideas it comes across, such as the equality of men and women, and how some people struggle with its acceptance. In this way, English language texts integrate both scientific and spiritual lessons \citep{Arbab1988}. 



 \section{Outcomes}
 
 While the program is under constant revision and responsive to the needs of each context, there have been a handful of studies looking at the benefits of the program. From a purely practical point of view, SAT was established to tackle the challenge of lower educational access to rural populations. This has certainly been accomplished, with %LOOK UP STATS IN HOW MANY STUDENTS HAVE COMPLETED (as a percentage and number), AND HOW MANY ARE CURRENTLY ENROLLED.
The classes are conducted locally and the times in which the group meets is determined by the group itself - taking into consideration their family and financial responsibilities. The curriculum has also has been designed to suit the needs of rural populations, making both the curriculum and the classroom discussions centred around their daily lives. This has likely contributed to a greater understanding of the materials and reduced fragmentation in their learning. And, while the program does not focus on test scores, the results of these have certainly demonstrated its worth - with SAT students getting an average of 25\% higher marks than their traditional education counterparts. 

Additionally, they have managed to do so with lower drop out rates and lower overall costs compared to both traditional education and other alternate forms of education. A study by \citet{Marshall2014} analyses four different alternate secondary education models in Honduras, and found that the SAT program was better able to retain students and had lower ongoing costs per student compared to the other models. The authors noted that this was likely a result of a high quality of both the curriculum and tutors, since tutors were from the local area and were known to the community, their sense of responsibility led them to take extra lengths to ensure students continued to stay with the program. For example, some tutors were known to go to the house of the students and share a meal or coffee with them if the student indicated they were not able to continue, sometimes encouraging their peers to do the same. 

Engaging tutors from the local area also contributes to the social and economic progress of the community in another way - employment creation. In an effort to raise the capacity of a community, FUNDAEC recognised the importance of locals being empowered at all levels. As such, tutors are trained from the local area which builds on an already established level of trust between the tutors, the group and the community, and a sense of responsibility that the tutor, and consequently the group, feels for the communities progress. While often the wages of the tutors will be paid by government agencies or accompanying institutions, these sources of funding can be unreliable, and since the tutors are from the local area and the community recognises the worth of the program, communities have been known to compensate the tutor in the interim. 

This, in addition to the curriculum and the methodology of the program leads to greater community involvement, trust and a sense of social responsibility for the participants and the group as a whole.  This was explored in study by \citet{Honeyman2010} who compared the SAT program with another alternate education program, Centros B\'{a}sicos (CB), a government initiated rural education program, using conventional educational models. Her comparison looked particularly at how the programs developed a sense of social responsibility, and how this led to student engagement in local development programs. She noted that how they programs explored an individuals relationship to ones society seemed pivotal to how they understood their sense of responsibility, and this often led to other outcomes such as community involvement and greater social trust \citep{Honeyman2010}. For example one of the SAT students reported that their sense of social responsibility had completely changed as a result of understanding himself as one aspect of a greater whole - for the whole to function and progress, each part had to both give and receive. The student reported that if he was asked to help his neighbour, he would have previously expected a financial reward, whereas now he is happy to do it as a form of service to his fellow man. He states: 
\begin{quote}
“there is this kind of service, this character of serving the community which has grown in me. And yet before I didn’t [think] of this. I was planning [to] go and study in these high schools, after that get a job and then help my family only... Before I used to think that, if like, if Olive come[s] and tell[s] me that `come and help me somewhere,’ she has to pay me. She has to pay for that work. But now there’s this thing that has grown in me of service... I know that, if I do this, I know I’m helping people. I’m helping my community to develop... In life, we don’t always have to work for money. But we have to, I can say, sacrifice" \citep[][p59]{Murphy-Graham2014}.    
\end{quote}



Finally, \citet{Murphy-Graham2008,Murphy-Graham2009,Murphy-Graham2010,Murphy-Graham2012} has written extensively about how the program contributes to the equality of men and women. As one of the spiritual qualities integrated throughout the SAT curriculum, equality is discussed in the broader context of justice. This seems to have a significantly impact on both the direct outcomes related to womens empowerment, but also since it is studied in the context of broader social justice, it also accounts for other prejudices that might arise once women overcome this inequality - for example class, race or nationalism. \citeauthor{Murphy-Graham2009}'s \citeyear{Murphy-Graham2009} article for instance, suggests that the SAT program raises consciousness around unequal relations, and its exploration of inequality particularly through social structures - using the family as the base unit of society - allows them to understand the 'rights' of men to be part of the domestic duties and the importance of dialogue through both subtle and direct ways to renegotiate family roles. She suggests there are 5 key features of the program that contribute to this transformation: (1) gender is prominent in the curriculum; (2) gender is conceptualised as an aspect of justice; (3) students reflect and dialogue about gender and its implications; (4) tutors are also encouraged to re-evaluate their belief systems through their regular development sessions; (5) it explores how gender roles are changed through simultaneous change in social structures such as the family. \citet{Murphy-Graham2009}'s comparison of the SAT program to other models demonstrates the effectiveness of such approaches. 


 
 
 
\section{unsorted}
\index{Ethics} \index{Capacity Building} \index{Undertaking tasks beyond current capacity}
In \citet{Sercombe2010}'s book about the need for ethics in youth work, he details the importance of inviting young people to be protagonists in their own lives and in the community, even though they are not yet fully ready to do so. This way, young people learn, under the watchful eye of those who have the capacity, the skills, knowledge and experience to undertake those roles independently, creating a safe environment by which to leap into the unknown, sometimes faltering and getting back up, all with the close accompaniment of a caring practitioner.


One of the challenges discussed by \citet{Sercombe2010} is that of possibly dependency when an individual assists young people's empowerment, as in the situation of a young person assisted to escape domestic violence by a youth worker. MY THOUGHTS - this is one of the factors that distinguishes \citet{Karlberg2016}'s capacity building for service , for even though young people may need assistance to escape such a scenario, they are simultaneously thinking about how their own actions can benefit others, and consequently are less likely to create a dependence on those in authority as a result.\cite{Sercombe2010}

It is no longer tenable to simplify the discussion to nature versus nuture. Instead we need to understand the dynamism of epigenetics and social systems in development. The plasticity of the brain allows individuals to respond to varying stimuli and experiences, and the genetic, hormonal and environmental factors create a complex cascade of interactions and outcomes. [see Karmiloff-Smith 2012 and Crick and Zahn-Waxler 2003; Tremblay 2010 and Zahn-Waxler et al 2008]\cite{Shute2016}

GOOD QUOTE ABOUT GENDER and SOCIALISATION ISSUES - United Nations’ observation that, ‘[i]n some parts of the world, improved educational opportunities for young women and an increased awareness of their position have built up expectations of greater equality and partnership in both professional and family life – which may be frustrated by encounters with the realities of the labour market and male behaviour’ (UN, 2003, p. ). \citep[][p265]{UnitedNations2003}. %THIS SUGGESTS that approaches to combat gender inequality need to include the men in a community as well as the women.

SOCIALISING DURING ADOLESCENCE - Culturally determined developmental tasks such as fulfilling gender roles, changing relationships with peers, accepting physical changes, making plans for their careers and futures, developing values systems and developing their sense of self [Havinghurst 1953 and Kroger 2004]\cite{Shute2016}

\citet{Shute2016} argues that post-feminism implies that girls issues have been resolved, and the newfound focus on boys issues is indicative of this shift, however she also argues that the integration of sociological feminist discourses would bolster psychological discourses on female issues. This seems to overlook the significant role boys and men play in the social environment, and how a discourse around the education of boys might contribute to the social challenges contributing to girls well-being


Collective self-development falls under the community development tradition and concerns the mobilisation of individuals for the development of their own community and situation through active citizenship \citep{Eriksson2011}

Community development work can come in the form of community organising, community education, informal education, popular education and socio-cultural work. Each has it's own origins - conceptually and geographically - yet often exist in relation to each other \citep{Eriksson2011}

Social pedagogy has significant similarities to community work, however its practised by organisations and not educational institutions is one of the distinguishing features. Social pedagogy is fundamentally a way of thinking about the world, and the related practices that arise \citep{Eriksson2011}

Social pedagogy is, according to its founder Natorp, a theory which fosters community. Individuals and communities have a complex relationship: according to Natorp without the community an individual cannot be considered human. This led him to conclude the individual and community are each other's prerequisites. Hence, fostering community, he argued, was a means to moral development throughout one's life \citep[Natorp 1904 as cited in][]{Eriksson2011}

To liberate oneself from oppression and marginalisation is the goal of social pedagogy and a way of thinking developed by Freire, among others \citep{Eriksson2011}.


Minimum of 94 different ways to define community development \citep{Mayo2008}

Whilst there are numerous definitions of community development, many researchers consider these initiatives as an educational enterprise since a largely associated with learning \citep{Eriksson2011}.

The educational side of community building is described by \citep{VanderVeen2003}. It consists of 3 forms of citizen education: training, consciousness raising, and service delivery \citep{Eriksson2011}

As society changes, so do the theories and practices of community development \citep{Eriksson2011}






It is worthwhile to pause, momentarily, and explore the usage of capabilities and capacity by development agencies and theorists. In the case of development scholars such as Sen, what is at stake is the understanding of well-being and freedom. He notes that simplistic approaches to development, based on increases in economic income, do not account for the varying choices people make and circumstances in which they live. For Sen, the concept of capabilities %SEE TABLE

%CHECK THIS OUT - http://bahai-library.com/vakil_toward_development_ethics  
%SEE section 2.3 community development and 4.8 community as social action in tensions document







%COMMUNITY DEVELOPMENT DEFINITION
- p334 - The 1950's saw British efforts to create an independent India; this `Community Development' approach was defined as a “process, method, program, institution and/or movement involving communities in the solution of problems, teaching democratic processes, and activating and/or facilitating transfer of technology to a community for more effective solution of its problems” \citep{Ramisch2009}



- MY THOUGHTS - 98\% of 5 year olds were assessed as highly innovative, tested at 10 and 15 years of age, this same group scored 30\% and 12\% respectively, and the same tests performed on groups of adults reported only 2\%. Genius is defined as someone who thinks outside the box, and as individuals grow they slowly are imposed limitations which limit their perception of possibilities... If JY are encouraged however.... %SEE page 2 of What Box? Out-of-the-Box Thinking for Career and Life By Sean Griffin - https://books.google.com.au/books?id=tiU1O39undUC&dq=%225+year+olds%22+best+at+thinking+outside+the+box&source=gbs_navlinks_s



Lack of meaning in western countries brings about increased rates of suicide \citep[][p481]{Goulet1980} % RE-QUOTED FROM 3. D. Goulet, ‘Strategies for meeting human needs’, 9. in M. E. Jegen and C. K. Wilber (eds.), Growth with L. J. Lebret and R. Moreux, Manifeste d’Economie Equity (New York: Paulist Press, 1979), p. 49.

authentic development is that which creates networks of solidarity based on reciprocity, not domination \citep{Goulet1980}

reflexivity - symbiosis between transformative action and active reflection \citep{Goulet1980}








P7 - Whilst many alternative approaches to development creare a dichotomous view - state is corrupt while community is moral, etc - Friedman posits the importance of the state in its ability to positively influence social conditions, in support of local empowerment efforts. \cite{Friedmann1992}

P7 - Communities also need to examine their own inequalities (eg gender) and understand the potential for conflict within and between, Inevitably, too, communities influence both regional and national agendas, since no community is completely self-contained, and there is a need for awareness around common resources (which insitutions are best suited to oversee) \cite{Friedmann1992}


P8-9 - Friedman focuses on the moral justification for development, observing the systems of Maoism, Ghandi and Islamic radicalism for their rejection of comptetition and profit-seeking approaches, and demonstrating the impact of social imbalance in individual capacity \cite{Friedmann1992}


P11 - KARLBERG's fully develop capacity as ideal ==> See “Human flourishing” Margaret Jane Radin (1987) -> considers that is means to be fully developed human being and conditions which support this... \cite{Friedmann1992}

P38 - Economic growth does not equate to development \cite{Friedmann1992}

P46 - “Black box” wherein differing concerns of men and women are rendered opaque (model of household) \cite{Friedmann1992}

P46-47 - Friedman acknowledges the limitations of the household framework - ie single person households and extensions of the household - extended family, neighbours, market economy, and civic and political associations \cite{Friedmann1992}

P47-48 - Households are a useful analogy since they generally require mutual obligation and recirpocity, wich, as principles, can extend beyond the household.\cite{Friedmann1992}




None of the above exposition means that community development is no longer valid or is futile. Nor is it suggested that the patterns of inequality and disadvantage are not in evidence - they are. In fact there is much that can be learnt from these attempts. 
%NICE SUMMARY OF WHAT IS LEARNT FROM EACH APPROACH: “The poor track record of mainstream economy-centric approaches to development; the participatory rhetoric intended to correct the top-down imposition of development schemes; the post-development focus on language, knowledge and power in development; the growing focus on empowering women in development processes; and the increasingly influential focus on capability development – these all provide helpful points of departure for understanding the emerging framework explored in this chapter.” \citep{Karlberg2016}





\section{SAT vs other programs}


Social capital in this context refers to social solidarity - the “development and reinforcement of an attitude of social responsibility at the individual level” - the strengthening of which will contribute to the processes of economic and social development \citep{Honeyman2010}.

Many conventional schools, particularly in Honduras, use a human rights based approach, which contrasts to SAT's focus on developing social responsibility \citep{Honeyman2010}

QUOTES ON SOCIAL CAPITAL “...a broad term encompassing the norms and networks facilitating collective action for mutual benefit’’ (Woolcock, 1998, 155). Along with ‘‘human capital,’’ social capital has emerged as a point of central concern the literature on economic development. While not all economists agree as to its importance, ‘‘[social capital] provides a useful umbrella term for those aspects of societies which, though difficult to measure and incorporate into formal models, are widely thought to be an important determinant of long-run economic success’’ (Temple, 2001). “: \citep{Honeyman2010}


Whilst social capital generally refers to social relationships, the benefits are often described in economic/market terms \citep{Honeyman2010}

Social solidarity and social cohesion is often used to distinguish it from its economic meaning \citep{Honeyman2010}

Despite the lack of clarity around the term `solidarity' (SEE Bayertz 1999), OECD defines it as social solidarity involving co-operation, a sense of social duty and reciprocity not founded on any immediate payback for those contributing to the welfare of others \citep[][p59]{OECD2001}.

Social responsibility is the bedrock on which development can occur - “Yet ultimately their foundations lie in the attitudes and actions of individuals” [p600] Referring to individuals as “rooted within a larger social network,” Sheldon Berman underscores the importance of “social responsibility” as an individual quality or attitude that allows cohesion, through opening pathways for individuals to be “...active and responsible members of the larger social and political community” \citep[][11–12]{Berman1997}.


Social responsibility, for \citep[][]{Berman1997} centred on the “...nature of a person’s relationship with others and with the larger social and political world”, and involves “...social and political consciousness, a sense of connectedness, acting on ethical considerations, prosocial behavior, integrity of action, and active participation” \citep[][p12-14]{Berman1997}. Building on Berman’s concept of social responsbility \citet[][p600]{Honeyman2010} defines it as “the personal investment in the well-being of others and society as a whole.”

McLaughlin and Davidson (1994) state that “...individual change [can] become a bridge to community solidarity and social change” as an outcome when individual empowerment “...motivates people to improve not only their lives but the lives of others” \citep{Honeyman2010} %CAN'T FIND ORIGINAL QUOTE!

Having social responsibility as a key aspect of educational processes will, according to \citet{Honeyman2010}, facilitate the process of positive social change 

The resulting sense of social responsibility that is a crucial outcome of the SAT program has led to world-wide recognition \citep{Honeyman2010}

The study deliberately identified tutors ranging in quality - three high, two middle and three low, according to the regional coordinator \citep{Honeyman2010}.

“Complemented by a review of the curricula of both the SAT and CB programs, my analysis of SAT students’ free-answer responses and interview transcripts led to the identification of five factors as playing important interconnected roles in SAT’s apparent effectiveness in cultivating social responsibility: (1) First, the use of a principle of human interconnection as an organizing concept threaded throughout the curriculum created a coherent context for thinking about their relationships with and responsibilities towards others. Next, the program reinforced understanding of that principle and its implications through (2) frequent opportunities for student interaction and discussion and (3) encouraging close study of others’ profound reflections on this issue in both poetry and prose. And finally, SAT seemed to encourage a general enthusiasm for acting on this understanding through (4) a consistent linking of theoretical explorations to practical applications in all curricular domains and (5) a focus on continual improvement towards excellence and increasing coherence between principles and actions in one’s personal life. “ \citep{Honeyman2010} % 5 aspects of SAT that distinguish the program, no need to use as quote

A key outcome of the SAT program is purpose \citep{Honeyman2010}. When students find their own purpose, they are able to positively influence others also, so the outcomes of the SAT program were not limited to the participants \citep{Honeyman2010}

The educational approach of SAT required students to put into practice the lessons learnt. The program encourages participants to think about the application of the texts in their lives in practical ways. This extends to a general pattern of thought and behaviour that demonstrates a willingness to apply any new ideas being learnt \citep{Honeyman2010}.

This focus on practical applications gave students a greater confidence to contribute to their community, whereas previously participants reported not feeling capable to help others \citep{Honeyman2010}

Students in the SAT program were internally motivated. While exams and grades did exist, they did so in a minimalistic way. The program instead focuses on excellence throughout their lives and the result seems to be greater internal motivation \citep{Honeyman2010}.

The centrality of unity in the curriculum seemed to tie together many other concepts in a much more meaningful way; instead of approaches education as a set of distinct subjects, SAT explores the concepts that link different areas of their lives, allowing them to see different patterns and connections and think more deeply about topics \citep{Honeyman2010}. %SEE Erickson 2002, p75, possible quote about concept learning

The way in which students view society will dictate their actions around it. If one sees society as a collection of distinct individuals without any common purpose, this would direct their actions in a very different way to one who sees society as connected individuals who constitute part of an “organic whole that must become unified if it is to grow and prosper” \citep{Honeyman2010}[p610]

Purpose is presented through metaphors around unity and the human body - society needs to emulate the collaborative diversity, reciprocity, and unified effort demonstrated in the human body for the body of humanity to become healthy and prosper \citep{Honeyman2010}.

This metaphor explores the importance of ensuring every member of society is considered, since harming even one part of the body affects the whole \citep{Honeyman2010}.

Causes of harm and division are discussed as an outcome of human action, and as such can be rectified \citep{Honeyman2010}

Unity is explored as distinct from uniformity in that each individual has their own unique faculties and capabilities they can offer \citep{Honeyman2010}

Similar programs delivered in South America focus on the concept of human rights \citep{Honeyman2010}.

Whilst human rights education may touch on the importance of showing respect for others, it does not necessarily establish a connection between ones own life and the lives of others \citep{Honeyman2010}. 
“[socially responsible people] experience a sense of connectedness and interdependence with others. The boundaries of their identity are not drawn tightly around themselves...Others and the world as a whole are part of the self” \citep[][p13]{Berman1997} %LINK THIS to the human body analogy used in SAT

How the relationship between the individual and society is explored seems to influence the participants attitude towards society. In human rights programs the individual is often discussed as the central concern, and while social relationships are essential, they are an aspect of life for which one has to compromise their personal freedoms. The SAT program on the other hand, focused on community as a source of joy and that connections to others is a means to grow as an individual, that the individual is one aspect of a whole \citep{Honeyman2010}.

Human rights, in some programs, focus on noninterference instead of being an active protagonist in the achievement of a particular goal \citep{Honeyman2010}



Most development efforts are set up in such a way that it is the outsiders who are working for the benefit of those inside a community - “us” and “them”. Yet while science analyses society in such a way to demonstrate separateness, it also clearly demonstrates “underlying patterns of oneness”. [p154] \citep{Arbab2000}

Discarding notions of “us” and “them” provides development the opportunity for rich and poor, educated and illiterate, to work alongside each other in the building of a materially and spiritually prosperous world for all \citep{Arbab2000} %DEVELOPED COUNTRIES...

When we avoid dichotomies of “us” and “them”, we also change the way in which those “in need” of such development programs are perceived \citep{Arbab2000}.

Discourse around poverty imply a level of backwardness, particularly when discussing rural populations; perceived as lazy, superstitious, ignorant and unmotivated, half the population of the rural poor - the “hidden capital of the developing nations” - were assumed to be unproductive and easily removed to the cities to further industrialisation processes \citep{Arbab2000}.

The Green Revolution also assumed humans to be Homo economicus \citep{Arbab2000}

Whilst there was some advance in the production of food, economic inequality continued, and even expanded \citep{Arbab2000}

What cuts across development efforts throughout the significant changes is that of the perception of the poor - that of deficit...Compare this development approach to that of youth development - the change in thinking about young people from a deficit perspective to one of developing capacity did much to change the programs that young people would engage in. This shift however, has not made its way to most development approaches \citep{Arbab2000}.

QUOTE RE: the nature of human beings: “The problem runs very deep. Efforts to ree development thinking from such paternalistic views tend all too often to fall into ideological traps, at the heart of which is a misconception of human nature. In the cherished notions of these ideologies, the liberated agents of change are either competitive, tireless labourers and entrepreneurs busily accumulating wealth or politicized social actors focused single-mindedly on matter of individual and group power. Neither the excessive individualism of the former nor the consecration to conflict of the latter, of course, supposedly serves only the self. Through some alchemy never quite explained, these labours and struggles result in social forces that will modernize underdeveloped nations and usher humanity into an age of prosperity. At the altars of such tragic misconceptions of human nature the lives of the masses of humanity have been sacrificed for decades.” [p155-6]  \citep{Arbab2000}

Arbab's concept of human nature is that of duality surrounded by social forces that constantly reshape that nature \citep{Arbab2000}.

The dual nature is that of animal traits - those that are required for the survival of the individual and the collective, and are neither good nor bad. This nature is contrasted by the spiritual nature which puts appropriate limits on the animal nature while also allowing us to rise above it \citep{Arbab2000}.

The knowledge of science and religion provide a basis for us to balance our dual nature and participate in the collective evolution of society \citep{Arbab2000}

Arbab's approach sought insights instead of formulating complex models and grand theories \citep{Arbab2000}

QUOTE: “The path of development must be illumined by the light of moral and spiritual principles emanating from religion, but religion willing to submit its proposals to the scrutiny of science.” [p158] \citep{Arbab2000}

The view of rural populations as backward fuel ideas of the progressive urban populations, resulting in self-fulfilling policies which allocate increasing resources and funds to urban areas and accelerating the disintegration of rural areas \citep{Arbab2000}

The path of development is generally defined by the elite, those in power \citep{Arbab2000}

Whilst most in the development field would disapprove of approaches that focus on individual empowerment and neglect the social structures that causes inequality, many programs have done little to address such issues and instead focused on issues such as employment or micro credit \citep{Arbab2000}.

On the one hand the enlightenment brought about the removal of superstitious veils in society, while on the other hand it labels that which is base and ugly `real' and dismisses the ideal. The result is a tendency to overlook the capacities of the human spirit - love, truth, unity, service - that have brought about the many accomplishments of humanity \citep{Arbab2000}.



QUOTE: “This reductionist approach to knowledge leads most development specialists to become one-eyed giants: scientists lacking wisdom. They analyse, prescribe and act as if man could live by bread alone, as if human destiny could be stripped to its material dimensions alone.” \cite[][p481]{Goulet1980}

Alongside mainstream development efforts were organisations of civil society; both approaches highlighted issues such as gender inequality, people-centred development [participation], equity, appropriate technology and basic needs \citep{Arbab2000}.

The 1980's would come to be known by some as the “lost decade” \cite{Singer1989,Easterly2001}

POSSIBLE QUOTE? “[separates the world's] materially rich and materially poor” [p169] \citep{Arbab2000}

While there is growing understanding in development thinking over the history of these efforts, these have not always translated into policy and practice \citep{Arbab2000}.

Values, the driving force for many policies and apparent to all involved, have been dormant from the theoretical discourse surrounding development \citep{Arbab2000}

With a growing awareness by development thinkers around the importance of people, culture, traditions and values, spirituality continues to be missing from development theory resulting in a continued focus exclusively on the material \citep{Arbab2000}

“A frightening trait of many cultures — ancient and modern — is that of associating different levels of dignity with a hierarchy of professions and activities. At the bottom of the totem-pole are, of course, adults who have never worked or have lost their jobs and cannot provide for their families. The “job” — not what he is or does — determines the individual's identity. One must have great courage and inner resources in order to resist the social and cultural pressure which strips the individual of his dignity when he is no longer “productive.” At the international level, the dominant culture also tends to strip social groups and nations of their dignity when they do not contribute or no longer contribute to the growth and prosperity of the world economy. As with poverty eradication, the fight against unemployment and underemployment must begin with recognition of the dignity and value of all human labor, even if it is humble, insecure, “unprofitable” or unremunerated.” United Nations (1995, pp. 32–33) \citep{Arbab2000}

Stewardship is also discussed in the World Bank's 1992 World Development Report: Development and the Environment in which opportunities in which incomes of the poor would be raised and the condition of the environment improved continue to be “unexploited” \citep{Arbab2000} %STEWARDSHIP

QUOTE: “Experience suggests that policies are most effective when they aim at underlying causes rather than symptoms”  \citep{Arbab2000} %SEE World Bank page 1

QUOTE regarding the crisis currently experienced by both science and religion: “ crises in systems of knowledge are welcome occurrences, for they are invariably harbingers of progress” [p179]  \citep{Arbab2000}

While science claims to be free from value statements, the language used to describe scientific rigour include value statements such as objective, reliable and rigorous \citep{Arbab2000}.

Scientific endeavours have certain aspects of faith as fundamental to their approach: “faith in the existence of order in the universe and faith in the ability of the human mind to make sense of that order and express it in a precise language” [p180] Einstein also wrote: “those individuals to whom we owe the greatest achievements of science were all of them imbued with the truly religious conviction that this universe of ours is something perfect and susceptible to the rational striving for knowledge” \citep{Arbab2000}

Science also posits theories which cannot be proven directly, yet are believed to be true because theories based on these assumptions have explained our observations, legitimating their widespread acceptance \citep{Arbab2000}.


QUOTE: “Scientific knowledge is thus an expression of truth that sheds light on interconnected realities: the physical reality of the universe, social reality, and the inner reality of the human being.” [p182] \citep{Arbab2000}


Likening the complimentarity of science and religion to that of a electron acting as a wave or a particle - complementarity asserts that under some conditions an electron will act like a wave and under others it will act like an electron. It is impossible to assert that the electron is one or the other \citep{Arbab2000}.

Complementarity helps us understand the realities of two coexisting entities - objective reality and human consciousness - and their interactions \citep{Arbab2000}

QUOTE: “True prosperity has both a material and a spiritual dimension.” [p196] \citep{Arbab2000}


Scientific advances largely define material progress, yet without the principles of a spiritual civilisation guiding these efforts, material progress can lead equally to adversity or well-being \citep{Arbab2000}.

Learning is a central component of translating principles into action, and the process of learning must be scientific accompanied by a sense of humility. Principles that are central to development, and must be translated into action through this process of learning include: the oneness of humankind; justice; equality of men and women; stewardship of nature; work and wealth; freedom and empowerment \citep{Arbab2000}.

The oneness of humanity can be likened to the human body - each cell has a unique function yet all must function cooperatively for the individual as a whole to live and function to their fullest potential \citep{Arbab2000}

The implications of this for development are threefold: first that every individual must be part of the whole, united, any efforts which perpetuate isolation or domination of one group cannot achieve this; prosperity must be extended through justice and cooperation - conflict and domination cannot be the means by which this can occur; bringing people together - locally, nationally and globally - must be achieved through both the goals and methods of development \citep{Arbab2000}.

Whilst justice is universally recognised as innate to the process of development, \citet{Arbab2000} suggests that understanding it as a spiritual principle innate within every individual can be manifested through an understanding of concepts that does not rely on others. That is, individuals must be responsible to understand reality through a critical awareness of social forces, free from the nativity of propaganda and ideologies which are not conducive to our own well-being.

The equality of men and women in all fields of endeavour - scientific, economic, social, political and cultural - must be realised through education \citep{Arbab2000}

The realisation of the equality of men and women will, according to \citet{Arbab2000}, bring about a restructuring of human society, resulting in a move away from violence and towards the establishment of peace, it will affect every level of society from the family to the government and all aspects of culture and society.

Stewardship of nature resembles the principle of mutual reciprocity discussed by \citet{Karlberg2004} as an attitude of cooperation, respect and interconnectedness, in which humanity must be compassionate and conscious participants in the evolving life of the planet.

The extremes of wealth and poverty are inextricably linked - the former cannot continue if the latter is to be abolished. Wealth, however, must be achieved through the undertaking of honest work - work which does not impoverish another either indirectly or directly \citep{Arbab2000}.

Freedom from oppression can only be actualised through the “systematic propagation of spiritual and material knowledge for the clear purpose of empowering people” [p205] \citep{Arbab2000}

A central component of development must be that the masses of humanity must partake in the creation and application of knowledge \citep{Arbab2000}

The purpose of development according to \citet{Arbab2000} is to bring prosperity to all the worlds people \citep{Arbab2000}.

Capacity must be built in the three protagonists of civilisation - the individual, the community and the institutions \citep{Arbab2000}.

Understandings of both the individual and community will continue to be reexamined as humanity continues to examine the concepts of power and authority in depth. Insights into power and authority, too, will arise from the discourse between science and religion and the corresponding impact that has on development \citep{Arbab2000}

Individual initiative and participation in collective endeavours are important aspects of development, but not an individual initiative which encourages freedom to pursue whatever one desires or some romantic definition of creativity, these efforts must accept discipline and be directed towards oneness \citep{Arbab2000}. %SEE \cite{Honeyman2010}'s discussion of sense of responsibility

Discipline must come from an inner sense of responsibility, not imposed externally \citep{Arbab2000}

As institutions develop to channel individual talents and energies of the members of its society, individuals will develop a sense of reciprocity \citep{Arbab2000}

QUOTE - DEFINITION OF CAPACITY BUILDING - “Capacity-building, as proposed here, entails the enabling of the individual to manifest innate powers in a creative and disciplined way, the shaping of institutions to exercise authority so that these powers are channeled toward the upliftment of humanity, and the development of the community so that it acts as an environment conducive to the enrichment of culture.” \citep[][p213]{Arbab2000}

Imperative to the development of capacity in the individual, the community or the institutions, the generation of knowledge around the use of the planets material resources as well as the intellectual and spiritual resources of its inhabitants needs to continue \citep{Arbab2000}.

“Many forms of Government have been tried, and will be tried in this world of sin and woe. No one pretends that democracy is perfect or all-wise. Indeed it has been said that democracy is the worst form of Government except for all those other forms that have been tried from time to time” \cite{Churchill1947}  \citet[][p215]{Arbab2000} discusses the concept of democracy as follows: “...democracy, defined as the dividing of people according to interest, talent, and ideology, who then “negotiate” decisions”. Continuing the discussion he notes that this form of democracy “continues to embrace violence. The purpose of each component group is to win. The means to this end are economic advantage and the mobilization of support to overwhelm the opponent. So strong is this legacy of “he who wins is right” that it essentially determines the way justice is administered.”

In opposition to the manipulation that commonly accompanies decision-making today, \citet[][p215]{Arbab2000} suggests that development should view decision-making as the “collective investigation of reality and the rational analysis of option”

Strategies already exist that allow for collective decision making, yet without this penetrating the development work the worlds majority continue to be unheard \citep{Arbab2000}

In order for these consultative processes to be effective, qualities such as honesty, courtesy, patience, tolerance and fairness have to be inculcated in each individual in order to avoid self serving power imbalances and conflict \citep{Arbab2000}.

\citet[][p216]{Arbab2000} suggests that decision making needs to include: “the abilities to maintain a clear perception of social reality and of the forces operating in it; to detect some of the opportunities offered by each historical moment; to properly assess the resources of the community; to consult freely and harmoniously as a body and with one's constituency; to realize that every decision has both a material and a spiritual dimension; to arrive at decisions; to win the confidence, respect, and genuine support of those affected by these decisions; to effectively use the energies and diverse talents of the available human resources; to integrate the diversity of aspirations and of activities of individuals and groups into one forward movement; to build and maintain unity; to uphold standards of justice; and to implement decisions with an openness and flexibility that avoid all trace of dictatorial behaviour.”

To achieve such a decision making process, learning has to be at its heart \citep{Arbab2000}

Capacity refers to the attitudes, understandings, convictions, habits and skills that constitute the behaviour of an individual or organisation \citep{Arbab2000}

What is needed is a spiritual approach. And, despite a growing apathy towards organised religion in many (particularly) Western contexts, religion continues to be a source of motivation to the worlds majority. Additionally, if undertaken with a globalised mode of learning, those who are apathetic to this source of inspiration may still derive some spiritual insights and principles from these sources in a more indirect way, as will the more religious derive scientific advancements (and their application) from the Western world - although this is not to suggest that the two are entirely separate \citep{Arbab2000}.

Young people need to be endowed with a “twofold moral purpose”, namely to develop their own intellectual and spiritual selves and to contribute to the transformation of society [p223] \citep{Arbab2000}

Education that imparts a selection of carefully chosen capabilities instead of skills and subject matter can be far more effective in educating the young people of society \citep{Arbab2000}

Education that prepares children for the challenges of adolescence requires: “the realization that it is chiefly service to humanity and dedication to the unification of humankind that release creative powers latent in one's nature; the understanding that not only knowledge of principles but the exercise and application of will is essential to both personal growth and social change; a conviction that honour and happiness lie not in the pursuit of wealth and power for their own sake, but in self-respect and noble purposes, in integrity and moral quality; and a disposition to analyze and a desire to understand the features of different forms of government, law, and public administration. To these must be added other attributes that enhance social effectiveness: an adequate understanding, at least in the local context, of the concerns of programs of social progress in such areas as health and sanitation, agriculture, crafts, and industry; some development of the power of intellectual investigation as an instrument of successful individual and collective action; certain ability to analyze social conditions and discover the forces that have caused them; the corresponding ability to express ideas and to contribute to consultation on community problems; the capacity to take part in community action as a determined yet humble participant who helps overcome conflict and division and contributes to the establishment of a spirit of unity and collaboration; and a reasonable degree of excellence in at least one productive skill through which to experience the truth that work is worship when performed in a spirit of service.” [p224] \citep{Arbab2000}

High school education needs to build on the fundamentals of primary: “the transition calls for a qualitative change, particularly in terms of scientific rigour, use of language, and social content, for it is in this stage of education that vague hopes and ideals regarding one's future and service to humanity must crystallize into the twofold moral purpose mentioned above. The student must now become a purposeful agent in charge of his or her own education. Every effort needs to be made to raise the student's consciousness to a higher level — a consciousness of the ramifications of personal choices being made, of the social forces to which one's community is subjected, and of the nature of the historical processes in which one is immersed.” [p224-5] \citep{Arbab2000}

Development efforts need to work with the rich as well as the poor, for if one were to continue inevitably the other would also \citep{Arbab2000} \citep{Arbab2000}


An example of the basic education and high school requirements mentioned above exist in the SAT program \citep{Arbab2000}

REALLY GOOD DESCRIPTION OF CAPABILITY AND THE REQUIREMENTS THEREOF (may not need to quote): “By the term “capability” we mean developed capacity to think and to act in a well-defined sphere of activity and according to a well-defined purpose. We use the word to refer not to individual skills but rather to complex spheres of thought and action each requiring a number of related skills and abilities. Moreover, we place great importance on the notion that the gradual acquisition of a given capability, in addition to the mastering of skills, is dependent on the assimilation of relevant information, the understanding of a set of concepts, the development of certain attitudes, and advancement in a number of spiritual qualities.” [p233]\cite{Arbab2000}

GOOD LANGUAGE CAPABILITY EXAMPLE: “In language, to cite another example, the mechanics of reading and writing are skills, but to read at a certain level of comprehension is a rather complex capability. Another language capability is that of describing what we observe in the world around us in ever greater contexts.” DO NOT QUOTE \cite{Arbab2000}

EXAMPLES OF INTEGRATING THEORY AND PRACTICE: Comment on Page 197, on highlighted text “ For example, the skills of animal husbandry were taught in conjunction with the study of animal physiology, and the steps to establish a village store with the analysis of abstract social and economic theories. “:Highlighted on Page 196: To describe the world around us quantitatively is a mathematical capability. Examples of highly desirable scientific capabilities are those of making organized observations of phenomena and designing experiments to test a hypothesis. Participating effectively in consultation is a capability needed in the social realm, as is the capability of participating in collective enterprises. To manage one's affairs and responsibilities with rectitude of conduct is a moral capability. Another essential moral capability is that of building environments of unity based on an appreciation of diversity \citep{Arbab2000}. 



The integration of theory and practice increasing a sense of purposefulness and a willingness to learn and change, while reducing prejudices and notions of class \citep{Arbab2000}

Integration of spiritual concepts throughout the entirety of the curriculum allowed participants to consider themselves as contributors to the creation of a world civilisation \citep{Arbab2000}

Spiritual qualities in balance: “justice moderated by compassion, not half-justice; lavish generosity together with humility, not cautious giving; absolute truthfulness acting in the medium of love, not the mixing of truth with lies whenever it is convenient.” FUNDAEC 1998, pp72 as quoted in \cite{Arbab2000}




Solving social injustices, inequalities and material deprivation can only occur after the world sees itself as one human family... The opposite is often presented as true - that we must solve the inequalities of the world for human oneness to be realised \citep{Karlberg2008}. %ROSIE

The frequently held assumption that identity is formed in opposition to the other is argued  to be a myth \citep{Karlberg2008}.

Comment on Page 9, on highlighted text “ We need to recognize that the nation-state is a cultural construct rather than an essential expression of our species-nature. In the broad scope of human history, the nation-state represents a transitory stage of social evolution. The era of unfettered national sovereignty, which began only a few centuries ago, is now, for all practical purposes, over. But our reluctance to accept this has very high social and ecological costs. This reluctance has enabled an emerging global market economy” : Evolution of humanity from nation state to global!!!  \citep{Karlberg2008}








FUNDAEC aims to integrate development with science, technology, organisation and education \citep{Arbab1988}.

Instead of delivering programs to the poor, FUNDAEC went beyond traditional forms of participation which merely encouraged participants to engage in already formed programs \citep{Arbab1988}.

Participation for FUNDAEC meant learning which perpetuated the institutions and organisations of the community which addresses the needs of the population \citep{Arbab1988}

Institutions are a key means which exacerbate poverty levels \citep{Arbab1988}

For many of the rural poor, “their destiny is inextricably tied to that of their neighbors; their village has to progress, to be educated, to have access to information, credit and technical assistance, and to develop its own viable organization.” \citep{Arbab1988}

The institutions established embarked on a process of consultation which provided a unity of thought around the principles by which they would function. These principles formed the basis for a conceptual framework which would evolve through ongoing learning and consultation. These consultative processes brought about 8 key principles which would guide its activities: 
1. The population with which FUNDAEC interacted would be viewed not as poor or communities of deficit, but as “irreplaceable resources in a self-sustaining process of change”. The mindset of the group then became how to translate that potential into viable paths of action that would create sustainable development, turning this potential into reality.
2. Assuming that individuals are resources for social change led the FUNDAEC team to assert the significant potential for every individual to channel those energies to the development of their community; this process came about through an educational process and led to increased resources for sustained research and action.
3. The relationship between the individual and society, in this framework, assumed that individuals were not merely the products of market and state manipulation, but that these social forces could be analysed by individuals which would facilitate empowerment.
4. FUNDAEC assumes a spiritual and material aspects of human nature: that there exists characteristics of animals which support the survival of humanity, and complementing this is the spiritual nature - love, justice, generosity - which leads humanity to prosper.
5. When focused on the spiritual aspect of development rather than the material, the outcomes of such programs tend to have a more balanced understanding of the relationship between urban and rural life
6. The use of technology would also be responsive to the needs of the community, and not a result of large agricultural applications. The research and education components of the program would respond to the technological needs of the community in a way that respects both traditional and emerging modern systems of the area.
7. Rejecting the idea of development emulating that of developed countries, FUNDAEC then realised it was not ideal to have predetermined goals and objectives. Plans of action emerged from the accumulated learning that was, and continues to be, an integral aspect of this approach
8. As modernisation sees the destruction of many old structures of this nature, new structures would also need to be created which would support all aspects of change - education, production, health, infrastructure and organisation - occurring in these communities  \citep{Arbab1988}

One set of activities which FUNDAEC focuses on is the flow of information - the horizontal and vertical communication throughout the community and the region \citep{Arbab1988}

Learning processes come about either as a research project of the Rural University, or through simple lines of action at the community level. Yet despite the approach, all learning is aimed at the development of human resources to contribute to their own path of development \citep{Arbab1988}. % TALK HERE ABOUT THE ANALOGY OF THE PATH - it is worth pausing here to consider the analogy of the path

While these programs have often taken the form of direct development projects, this is a direct result of the principle that each individual should participate in activities which have meaningful and immediate benefit to the community \citep{Arbab1988}

While many development approaches provide alternatives to formal education, FUNDAEC acknowledges the importance of formal education yet suggests that it needs to be more relevant to the needs of the people for which it serves \citep{Arbab1988}

While there has been a strong focus on delivering primary education to rural populations, the desire for furthering this education has been increasingly expressed by local populations. This needs to be delivered, however, in a way that changes rural areas for the better \citep{Arbab1988}.

The need for secondary education arises from the acknowledgement of young peoples inability to make advancements in their own life or contribute meaningfully to the community \citep{Arbab1988}

The program requires students to meet for 10 hours a week and undertake a study of 28 text books over a period of 18-24 months \citep{Arbab1988}.

The course offers skills in the ares of agriculture, animal production, health, education and organisation, and is run by members of the local community who are trained at the Rural University \citep{Arbab1988}.

Through SAT, students gain a social vision as well as practical skills that they use in service to their community. Students exceed the expectations of many around the rural poor \citep{Arbab1988}

FUNDAEC also developed a set of textbooks to follow the SAT program which together is considered, by the Colombian Ministry of Education, to constitute the entirety of the secondary school curriculum \citep{Arbab1988}

After participating in the secondary education offered by FUNDAEC, community members are encouraged to continue their engagement with the Rural University through continued participation in the generation and dissemination of knowledge and as tutors of the program for future students \citep{Arbab1988}.

Many development programs result in a small number of elite community members who reinforce the existing realities and themselves becomes barriers for others to develop. Empowering the entirety of the local population instead of a small group, as acknowledged by FUNDAEC, continues to be a challenge in this endeavour, yet a promising means to overcome this is to work with particular groups within the community, for example young people, and imbue within them an attitude of service to the wider community. Only after this attitude is developed do students work on projects that will benefit themselves. This is done through the incorporation of facilitator training, for smaller educational programs, into the curriculum - family education and animators of community organisations \citep{Arbab1988}

FUNDAEC approaches development in modest terms: “to help one or more rural populations to take the necessary first steps and begin their search for new options, by carefully examining the diverse processes of life in the region, looking for alternative technological and organizational practices, learning from these activities, and in doing so, educating new generations who rather than simple objects of oppression can become effective actors in an unavoidable process of profound social change.” This seemingly simple task has had profound implications for the communities in which it exists \citep{Arbab1988}.

FUNDAEC learnt that technology needed to be adopted appropriately. This meant that it had to be assessed according to its relationship to the development of the individual. Appropriateness in this context means that technology has to be a reflection of the systematic learning taking place within the context of a developing community, considering the existing educational systems and the desired patterns of social organisation \citep{Arbab1988}

Traditional forms of education guide students into careers and professions based on academic tradition, and not in response to the needs of the society \citep{Arbab1988}

“The educational programs... would address fundamental intellectual and spiritual issues, not only of a single individual but also of a community and of an entire people.”  \citep[][Section C]{Arbab1988}

Social forces is often perceived as something one responds to, but has little or no control over. In addition to this understanding, FUNDAEC suggests that social forces can also be generated within a population \citep{Arbab1988}

Instead of rigid curriculum and narrow objectives, FUNDAEC's approach focused on the science of education and the art of teaching. This included lessons which were beyond the students capacity and took students outside their comfort zone. This was done in order to challenge the students and allow them to respond to real world situations and well-defined situations \citep{Arbab1988}. Reply: GOOD QUOTE - In efforts to develop capacity, one must look beyond what exists, for as \citet{Sercombe2010} states, “if you wait until someone has demonstrated responsibility before you give them any, they never develop those capacities.” %THIS QUOTE IS COPIE ELSEWHERE % FUNDAEC uses this approach for a very specific purpose - to equip students to deal with real world issues and to have the flexibility to react to unknown situations. 

The content of the SAT program is directly relevant to the community, so for example when a student is learning mathematics, they explore the applications of percentages and fractions as it applies to the health indices in a village \citep{Arbab1988}

The educand is a term which denotes the existing knowledge and capacity which which individuals come to the educational process - they are not empty vessels which needs to be filled, but mines filled with gems that can bring value to the whole \citep{Arbab1988}.

RElationship with tutors - consultation between tutors and students, raising questions and trying to find answers together \citep{Arbab1988}

Role of the teacher - it is acknowledged that the teacher has a level of authority, but this authority comes as a result of the knowledge they possess about a particular subject, and not an innate quality of the teacher \citep{Arbab1988}.

Active participation comes from students understanding their own responsibility towards the educational process - students need to be active agents in their own learning. In many cultures, including Colombia, this requires a level of unlearning cultural traditions and customs. Once students take responsibility for their own learning, they actively engage in planning, detailing schedules, supervising activities, and revision of course content \citep{Arbab1988}.

Lessons on social and economic theories are integrated into the discussion of a village store as an example of the integration of abstract theory and practical skills and knowledge \citep{Arbab1988}

Not only are abstract concepts integrated with the practical, but in FUNDAEC's courses the material is integrated with the spiritual. Attitudes of spiritual relevance - compassion, justice, humility, trustworthiness and confidence - are integrated into the curriculum \citep{Arbab1988}. %SEE \cite{Murphy-Graham2008} about the integration of gender equality into a lesson on the digestive system


In a comparison of alternate middle school programs in Honduras, the SAT program was significantly better at retaining students and had lower cost-effectiveness than three other analysed programs. The study suggested that this came about as a result of the quality of design and support - ie the curriculum and the training and ongoing support of tutors \citep{Marshall2014}. 




The governmental recognition of the SAT program allows students to gain entry to university or employment that requires a secondary education \citep{Kwauk2016}

The cost of the SAT program is approximately \$450 per student annually \citep{Kwauk2016}

To date, SAT has engaged 300,000 students through its direct programs, and also has 155 indirect centres \citep{Kwauk2016}

Comparatively, students graduating the SAT program have test scores 45\% higher than government schools in neighbouring rural villages \citep{Kwauk2016}.

Since the SAT program also addresses social issues, students have a considerable sense of social responsibility, and students come out of the program with a sense of empowerment around gender inequality (for women, this is defined in terms of their ability to make strategic life choices and capacity for self-determination) \citep{Kwauk2016}

SAT program has been adopted throughout Latin America, and has been adapted in various countries throughout Asia, the Pacific and Africa as the Preparation for Social Action (PSA) \citep{Kwauk2016}

Employment is created through hiring local tutors, and the application of learning outcomes improves the well-being of the local community \citep{Kwauk2016}

Urban youth are twice as liekly to attend secondary school as their rural counterparts (63\% vs 28\% in Latin America; 73\% vs 45\% in Colombia) \citep{Kwauk2016}


In addition to community service and moral and character development, the SAT program emphasises analysis and inquiry \citep{Kwauk2016}

In the words of a SAT graduate, students are seen as “a mine rich in gems of inestimable value that education reveals” \citep{Kwauk2016}

An example of a practical application is encouraging students to start study groups which improve the literacy of children and adults in the village \citep{Kwauk2016}

Capacity is developed through the application of principles - such as gender equality, human interconnectedness and social responsibility - through acts of service \citep{Kwauk2016}

An outcome of the systematic learning undertaken in the SAT program is the text books themselves. Whilst there are quote a number, the text books are arranged according to specific capabilities that students are to gasp, and the texts themselves were developed through action research and continue to be revised so that they can be utlised more universally \citep{Kwauk2016}.

Many SAT programs charge \$33 per year for the purchase of texts books on the belief that students investment increases their commitment to the learning process. Parents and community members acknowledge the importance of such sacrifices, with one mother of a SAT graduate commenting that “all sacrifices have their rewards” \citep{Kwauk2016}

The facilitators of the program are called tutors. This term is used deliberately to distinguish between traditional forms of teaching in which teachers are the source of knowledge and students are empty vessels. The term tutors denotes a dialogic relationship in which the tutor must guide and facilitate the group, that there be a culture of respect and trust between students and teachers, and that the texts are the source of knowledge, not the tutors \citep{Kwauk2016}.

SAT tutors are expected to continue their own learning through the process of guiding others \citep{Kwauk2016}

More examples of practical applications - conducting surveys as an example of maths, public health campaign as part of biology \citep{Kwauk2016}

Groups of 15-25 students \citep{Kwauk2016}

One possible explanation for the low drop out rates (see \cite{Marshall2014}) is that tutors will often go to the home of the student and share a meal or coffee while discussing the challenges that students are facing. Since the tutors are from the local area, this is a very natural experience for both the tutor and the student \citep{Kwauk2016}

The quality of tutors is maintained through the rigorous and regular training received. Tutors are not required to be trained teachers, however they must be attending university, must participate in 10 day training seminars every three months, and are often graduates of the SAT program. Each tutor is linked with a field adviser who is responsible for 10 tutors \citep{Kwauk2016}

Each group meets for 20-25 hours per week, but the schedule is flexible and changes according to the other responsibilities of the tutor and the participants. The tutor stays with the group throughout the entirety of the 6 year curriculum \citep{Kwauk2016}

The largest cost for the program is tutor salaries, which in Honduras and Colombia is covered by the ministry of education \citep{Kwauk2016}


Another key component of the SAT program is the network of institutions which support the program. FUNDAEC formed partnerships with a number of organisations which would support the implementation of the program throughout the region. This approach led to more flexibility, allowing organisations to run it in a way that suited the community and exigencies of the organisation - for example it could be run within existing education systems, or alongside health care efforts \citep{Kwauk2016}

There is also a strong emphasis on community involvement, from the provision of classrooms, to land in which students can practice their newfound agricultural skills, to the contribution of time and expertise of community members in assisting with community service projects such as public health and childhood education, providing internships or assisting to establish income generating projects \citep{Kwauk2016}

This community involvement builds community trust \citep{Kwauk2016} %SEE \cite{Murphy-Graham2014}

PSA since 2006 \citep{Kwauk2016}

Arguably the better outcomes are as a result of the significant learning materials provided to students alongside the rigorous and ongoing training provided to tutors \citep{Kwauk2016}

Since the tutors are employed from the local population and the practical outcomes of the program are directed to the local community, often in direct response to identified needs, the program has significant benefit to the community as a whole \citep{Kwauk2016} %see \cite{CRECE2001}

The SAT program, and particularly extensions thereof, see themselves as a social movement based in education, as opposed to an education program itself \citep{Kwauk2016}

The systematic learning approach were also employed in its expansion to other countries, with constant monitoring and evaluation - and course corrections as required - the program was able to change to suit the needs of the community. The program continues this attitude of learning through the three monthly reflection gatherings that are held with SAT communities \citep{Kwauk2016}.

Some programs are certified by the government \citep{Kwauk2016}

The SAT program continues to refine and complement its program in response to the changing needs of communities \citep{Kwauk2016}

In some manifestations of the program rural well-being has been renames to sustainable development so that it was more widely recognised \citep{Kwauk2016}

While strategic partnerships with governments has allowed the expansion of the program and increased recognition, it also has its limitations. Wages of tutors, for instance, come from a discretionary budget which is distinct from that of permanent teachers, and as such wages can be sporadic which can leave the community to pay for the tutors wages  \citep{Kwauk2016}%SEE CRECE2001

Formal recognition also leads to limitations in the foundational principles. Service for instance, which is an integral aspect of the program, becomes secondary when implemented in a more mainstream educational environment, as in the Nicaraguian manifestation \citep{Kwauk2016}

Wealth, in the SAT program, is measured according to ones talents and capabilites, not material possessions \citep{Kwauk2016}

While some aspects of the program are non-negotiable, such as the core curriculum and community service, aspects such as the frequency and number of hours to meet each week, the practical activities and communities service can be tailored to the needs of the community. The curriculum itself has been designed in a universally suitable way, for example in 2000 the currency used in the books was changed from pesos to universales \citep{Kwauk2016}

The attitude of SAT tutors and their relationship with the students - another SAT core - requires tutors to use the experience to continue their own learning and development. Tutors encourage students to engage meaningfully with the text instead of rote learning, and by doing so they develop critical faculties that avoid negative attitudes to rural or traditional lifestyles and contribute to moral disintegration  \citep{Kwauk2016} %SEE RICHARDS 2005

Having tutors from the local area ensures that the group is more committed to community service and development \citep{Kwauk2016}

Tutors also feel a sense of responsibility beyond the classroom \citep{Kwauk2016}

“Together, SAT tutors and students redefine a more equitable learning process oriented toward critical problem solving and the sustainable development of their communities”  \citep{Kwauk2016}[p18]



Social responsibility was discussed in terms of why it is important to have concern for others, the challenges to and support for such principles and practical applications  \citep{Kwauk2016}










In the context of development, Sen's framework can be conceptualised as “expanding people's opportunities to lead lives that they have reason to value” [p51] \citep{Murphy-Graham2014}

Definition of education??? “education expands freedoms, agency, and empowerment” [p51] Reply: Sounds very similar to Capabilities approach  \citep{Murphy-Graham2014}

QUOTE!! Education ‘‘might be operationalized to form human beings who can contribute to shaping the kind of society which values human capabilities’’ \citep[][p392]{Walker2012}    This quote demonstrates the link between education and the capabilities approach... it also shows that Education contributes to the shaping of society...   She goes on to say “[these students will] want to contribute to capability building and a society and public culture which can sustain capabilities for all.” \citep[][p392]{Walker2012}


Affiliation, one of the central capabilities discussed by Nussbaum, is closely associated with trust \citep{Murphy-Graham2014}

While many believe that trust can enhance the educational experience, few consider trust a crucial element of the educational curriculum or what value may arise from its study \citep{Murphy-Graham2014}

Trust is described as the ‘‘civic lubricant of thriving societies,’’ (Delhey et al., 2011, p. 787) and ‘‘the keystone of successful personal relations, leadership, teamwork, and effective organizations,’’ (Forsyth et al., 2011, p. 3). \citep{Murphy-Graham2014}

Uslaner emphasizes that ‘‘\textit{\textbf{trust must be learned, not earned}},’’ and that ‘‘trust stems from an optimistic view of the world that we initially learn from our parents’’, “early in life” \citep[][p76-7, emphasis in original]{Uslaner2002a}.

USE FOR MY OWN PURPOSES: While both X and Y make claims about the importance of trust in schools, they do not hypothesise how schools might actively build trust among students, or more explicitly, how schools can “teach” trust... Nevertheless, we learn from this research that trust is a core dimension of effective schools, and that levels of trust among school personnel and parents vary \citep{Murphy-Graham2014}

Honduras less than 25\% of the population had completed grade 9 education... Less than 10\% completing high school \citep{Murphy-Graham2014}

FUNDAEC's use of the term capabilities is not connected to Sen's capability approach. Instead, it is used to describe the organisation and goals of the curriculum, specifically, FUNDAEC emphasises development of scientific, technological, linguistic and service capabilities \citep{Murphy-Graham2014}

Capabilities are developed such that they can be channeled to the building of better communities and to transform society as a direct outcome of the process of taking charge of their spiritual and intellectual growth \citep{Murphy-Graham2014}

SAT 60 textbooks \citep{Murphy-Graham2014}

Training of tutors is undertaken in the same way tutors will conduct their class \citep{Murphy-Graham2014}

Whilst the group normally consists of 15-25 students who start the program soon after primary school, it is open to older students as well \citep{Murphy-Graham2014}.

The groups has a large level of flexibility with its 20-25 hours of study per week. The length and regularity of the classes can be  based on the availability and preferences of the students in consultation with the tutor, which can change according to the economic and family responsibilities of the group. For example in harvesting season the group may postpone for a period and make up for the missed classes later \citep{Murphy-Graham2014}.

PSA = 26 texts \citep{Murphy-Graham2014}

PSA is still overseen by FUNDAEC in Colombia and received a grant from Hewlett Packard to implement the program in Zambia, Kenya and Uganda \citep{Murphy-Graham2014}

The PSA and SAT environments not only create a trusting family environment, but they also signal institutional and policy changes which contributed to the construction of trust \citep{Murphy-Graham2014}

One indicator of structural change is the arrangement of the classroom: each group is arrange in a circle or square so that everyone is visible, as opposed to the rows of traditional classes in which all students face the teacher. In fact, this approach is completely contrary to traditional forms, with one student sharing the possibility of corporal punishment for simply asking the meaning of an unknown word to teachers \citep{Murphy-Graham2014}

This classroom arrangement seemed to promote classroom interaction, with students taking turns reading from the texts, actively engaging in classroom discussion, and working in small groups \citep{Murphy-Graham2014}

The tutors made efforts to minimise the hierarchical differences between student and tutors \citep{Murphy-Graham2014}.

Tutors admit to not knowing the answer, and being on a learning path themselves, reinforcing the importance for students to take charge of their own learning experience \citep{Murphy-Graham2014}

The physical arrangements in the classroom and the positioning of the tutor as an equal learner, the program is effective in developing trust between the group \citep{Murphy-Graham2014}

The attitude of the teacher as learning alongside the participants creates an atmosphere of trust in which participants feel they can ask questions and take risks \citep{Murphy-Graham2014}.

The collaborative classroom environment students learn to depend on each other, which is fundamental for social functioning \citep{Murphy-Graham2014}

In Uganda one of the challenges has been to find tutors who have this attitude of humility and continue their process of learning alongside your group. Many cling to traditional notions of teaching in which the teacher has authority over students \citep{Murphy-Graham2014}

The creation of these trusting environments in PSA - breaking free from the humiliation and possible physical punishment of traditional schools - is indicative of structural change and starting something anew \citep{Murphy-Graham2014}

This attitude not only allows participants to admit vulnerability to the tutor, but encourages participants to support each others learning. One tutor shares that when one participant struggles to grasp the material all pause to assist them in the learning process, cultivating an atmosphere of shared responsibility and shedding competitive norms \citep{Murphy-Graham2014}

This format encourages even `misbehaving' students to have positive relationships with the tutors \citep{Murphy-Graham2014}

The relationship between peers was strengthened through the family like environment and the collaborative learning experience \citep{Murphy-Graham2014}

Students regularly worked alongside others in groups to complete assignments, relying on others and contributing to the building of trust in the group \citep{Murphy-Graham2014}

The group becomes a family in which the tutor is seen as a mother figure and the other participants like siblings \citep{Murphy-Graham2014}

Although the group size of PSA and SAT are often small in comparison to traditional classrooms in these countries (in Uganda for instance the average size is 68 students), students comments that the attitude of learning together would accommodate for larger numbers, where the students who picked up the material quicker would help those who struggled \citep{Murphy-Graham2014}

The sense of collective responsibility that comes from this program is evident. For one group in Honduras, for example, if one student drops out then the entire group goes to visit them in their home to encourage their continued participation in the program. This is to demonstrate concern for each individuals challenges \citep{Murphy-Graham2014}

The trusting environment allows students to express themselves even if they are unsure if they have the correct answer, or feel confident in reading the text aloud. Every opinion is sought and collectively examined \citep{Murphy-Graham2014}

Since there is no consequence for the wrong answers, trust is built and students are able to learn not just from the texts but from the experiences of their peers \citep{Murphy-Graham2014}

The service projects extend the level of trust beyond the group itself, into the community. Often the projects require involvement from community members working with participants \citep{Murphy-Graham2014}

Service projects allow the group to consult with various members of the community who are often willing to share the collective community knowledge \citep{Murphy-Graham2014}



Reword and quote: “The notions of a ‘‘kind of service’’ and personal sacrifice to benefit others echo the idea that, ‘‘by taking responsibility for others we can mitigate the vulnerability which results from dependency’’ (Misztal, 2011, p. 368). “ \citep{Murphy-Graham2014}

The sense of responsibility developed in the SAT and PSA programs assist individual participants develop self-esteem, care and respect for themselves. This brings awareness of mutual interdependence and fosters community solidarity \citep{Murphy-Graham2014}

One of the projects is about sharing some of their learning with the younger children in the community. The research found that this service project required students to build a level of respect for the younger ones in their village before parents would allow their children to be part of the group. This required a complete rejection of traditional attitudes of younger people as `inferior' or `disgusting', and ultimately built a sense of self-worth and belonging for the PSA students \citep{Murphy-Graham2014}

Students in PSA/SAT are to become “engaged in the construction of knowledge that is both locally relevant and globally informed.” [p59] \citep{Murphy-Graham2014}

For FUNDAEC, the textbooks are a way to create dialogue with the students, the tutors, the community, the institutions and FUNDAEC themselves. As such, the texts are frequently updated to reflect the advancing discourse \citep{Murphy-Graham2014}

The SAT program uses parables to explore how to foster particular dispositions of trust, honesty and social responsibility \citep{Murphy-Graham2014}

In many of todays educational systems, competition, achievement and accountability come to the forefront of educational practices, while in SAT and PSA there is a deliberate attempt to incorporate concern towards others, the building of strong social relations and a sense of belonging \citep{Murphy-Graham2014}

Trust building can be an essential element in overcoming distrust and social fragmentation. Education can contribute to the building of trust, both particular and generalised trust \citep{Murphy-Graham2014}

An important aspect in overcoming inequality is for every individual to commit to the welfare of the distant (and not so distant) other. PSA/SAT, while not specifically addressing issues of a global community, develops within their students a sense of responsibility to all members of ones community, which if extended, would go a long way to building collective trusteeship and reducing inequality \citep{Murphy-Graham2014}






Service learning and critical pedagogy have been referred to as utopian  \citep{VanderDussen2009}

In her study of the PSA program in Uganda, \citet{VanderDussen2009} identified the importance of service as an expression of the social relationship between the individual and others that serves as a means for construction and reconstruction

Through service learning the individual is removed as the central element of concern, and the community's well-being becomes paramount. This is contradictory to many approaches which centre the individual to analyses of social change - contradicting the intended purpose of their actions \citep{VanderDussen2009}.

Visions for their future, while full of hope, remain unclear and are constantly negotiated through dialogue, however the principles by which to accomplish them are well-defined - love, cooperation, compassion and unity \citep{VanderDussen2009}.

Service is, according to FUNDAEC, considered "meaningful action to build a better world". As such, the service undertaken is dependant on the groups understanding of "meaningful action" and their growing understanding of what it means to "build a better world"  \citep{VanderDussen2009}

Service in pursuit of social change not only response to the needs of the community, but also the the emerging capabilities of the students undertaking that service \citep{VanderDussen2009}

When service is aimed at the promotion of social change, individuals are no longer recipients of services or aid, and one does not have to be qualified or have extensive experience to contribute to the progress of ones community. Such models assume that recipients are disempowered and the privileged elite are able to empower them \citep{VanderDussen2009}

The inclusion of dialogue in service for social change may be self-evident, however the implications for practice are significant \citep{VanderDussen2009}.

Dialogue included an evolving and ongoing conversation with the community, from informal conversations with various members to participation in formal decision-making processes \citep{VanderDussen2009}.

Instead of thinking about service learning as a way for one individual to offer service to another less fortunate than themselves, or even something that adults and organisations have to tolerate in pursuit of individual growth and learning, service should be considered as "dynamic interplay of the individual, community, and social structures" [p141] \citep{VanderDussen2009}

These three should each be considered protagonists in a "continual process of organic change in which service is an expression of action, learning and dialogue."  \citep{VanderDussen2009}[p141]

Dr Martin Luther King was quoted saying: “Education without social action is a one-sided value because it has no potential; social action without education is a weak expression of energy” \citep[as quoted in][p437]{Claus2010}

LOOK UP SEIDER 2007 study on how religion and spiritual based courses are most effective in changing individual frames and subsequently commitment to service  \citep{VanderDussen2009}

Motivations for serving can often be driven by less tangible realities than material or social gain, but instead by a concern for the well-being of others, spiritual aspirations and love \citep{VanderDussen2009}



\citet{VanderDussen2009}'s research suggests that initiatives such as awareness-raising, writing petitions or fundraising are not sufficient actions to enhance learning

Service needs to be seen not as a one-off act, but as part of an ongoing process. This suggests that while service such as serving in a soup kitchen might be beneficial, if considered in the context of ongoing change in response to the needs of the community, service is not an "end" in itself, but as a step towards larger social and structural change \citep{VanderDussen2009}.



Subsequently the content of the curriculum was shaped by these learning processes with an emphasis on integration. So often young people are faced with an education system which fragments their learning and subsequently their understanding of the world \citep{Arbab2007}.

FUNDAEC described its aim in these words: "to make it possible for the masses of humanity not only to have access to information, but to participate in the generation and application of knowledge" [p5] \citep{Arbab2007}

The process by which FUNDAEC developed these courses is also noteworthy. In the region which they were based, they identfied the various social, economic, political and cultural activities which the local population would engage, and understood the need for corresponding learning processes that would contribute to the development of these processes and the well-being of those within. Each learning process was established through a process of action, research and training, and occurred within the population itself \citep{Arbab2007}


One such learning process required an understanding of the vertical and horizontal flow of information throughout the region \citep{Arbab2007}

"...FUNDAEC’s texts can be regarded as chronicles of an evolving experience aimed at transforming social reality through dynamic and effective research and the participation of an ever-greater diversity of minds.  They are tools for the systematization of the lessons learned through collective action, on the one hand, and the result of that systematization, on the other." \citep{Arbab2007}[p9]







While there has been considerable advances in changing the gender disparity and ensuring girls get equal access to education, the education itself - it has been suggested - continues to perpetuate the disadvantage experienced by women \citep{Murphy-Graham2009}.

While elements of education such as peer group norms, teacher attitudes and interactions with students, and the curriculum can be a means to reproduce gendered differences, strategies such as critical reflection, specifically addressing gender in the curriculum, and professional development for teachers to address gender bias all contribute to undoing gender (SEE Gilbert and Gilbert 1998) \citep{Murphy-Graham2009}

in 2009 70 texts \citep{Murphy-Graham2009}

The text books are written using conversation style text and include direct questions which elicit discussion in the group. They draw from both traditional knowledge and global scientific sources, integrating theory into practice. The combination of abstract ideas with practical activities builds from a concept of unity which integrates the practical and spiritual aspects of a community in a way which emphasises the individual as part of the whole \citep{Murphy-Graham2009}



15-25 participants per group \citep{Murphy-Graham2009}

Community members not only share their knowledge and provide learning opportunities, they can also be drawn upon to support their economic needs, such as providing a venue for their study, subsidising texts, etc \citep{Murphy-Graham2009}

SAT features issues around gender directly into the curriculum, exploring the importance of overcoming gender bias as individuals and through institutions \citep{Murphy-Graham2009}.

Gender is discussed in the context of broader social justice goals \citep{Murphy-Graham2009}

FUNDAEC QUOTE about considering gender as part of broader injustices expands the implications of such principles such that when women are able to overcome gender dispairities, they are not faced with other discrimination such as race or economic power. This requires a change in the very structures of society: "Many feminist thinkers recognize, for example, that oppression will remain in society even if action is taken against discriminatory practices. They take ‘‘feminist praxis,’’ then, beyond questions of law, norms, and regulations for the elimination of discrimination to a profound examination of the fundamental elements of social theory and methodology in the context of the status of women...Some of this discourse clearly indicates that the challenge is not to simply open room for women in the present social order but to create a new one which embodies among other things the equality of women and men...One of the reasons we insist on relating the status of women with the principles of justice is that we are not satis-fied to see some women liberated from that form of oppression that is based on sex only to join the institutions of oppression operating in the context of class, race, nationality and political and economic power. The challenge for us is to bring about the kind of change, both in people and in social structures, that turns relationships of domination into relationships of collaboration, cooperation, and reciprocity. We hope that, in the process of achieving the equality of women and men, humanity will be able to eliminate oppression and create a society that embodies the principles of justice (FUNDAEC2007)."  \citep{Murphy-Graham2009}


The SAT program achieves this through a constant dialogue, not only with other students, but also with tutors, community members, institutions and FUNDAEC itself \citep{Murphy-Graham2009}

The tutors also examine their own beliefs about gender in training and reflection spaces with other tutors. They are asked to explore how society perpetuates `false definitions' of gender and what it means to be human, as well as how structures such as the family need to undergo profound changes to reflect equality \citep{Murphy-Graham2009}

Considering the family as the "basic structure of society" gender is explored both in the context of the family, with the exploration of daily interactions and the division of domestic labour, as well as the broader implications for larger societal structures \citep{Murphy-Graham2009}.

The SAT program, particularly in places such as Honduras where many are not able to undertake secondary education, is attended by both students who have recently completed primary school, as well as those who were unable to continue their education previously. As such, adults and youth often study alongside one another \citep{Murphy-Graham2009}

QUOTE: ‘‘shift our inquiry about ongoing social reactions to focus on change’’ ( Deutch 2007: 114). \citep{Murphy-Graham2009}

One SAT student focus on the limitations imposed on men by traditional gender roles, suggesting that men should have the "right" to participate in domestic labour. \citep{Murphy-Graham2009}

Men also share in this attitude change. One male participant suggested that traditional forms of gendered division limit men's development, while another reports that participation in domestic labour would `advance his character' \citep{Murphy-Graham2009} \citep{Murphy-Graham2009}

Dialogue is extended beyond the group. The lessons from class are often discussed with spouses, for example, who then examine their own beliefs \citep{Murphy-Graham2009}

Creating change can be a slow process, one that that can be inconsistent and is not always sustainable. In the case of the SAT program, for example, some students were discouraged from continued participation in the program through verbally and physically threatening spouses. However the program also has the potential to create profound change beyond the group itself \citep{Murphy-Graham2009}.
