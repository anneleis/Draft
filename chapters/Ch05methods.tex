\chapter{Methods} %PROOF


The significant research identifying the benefits of community engagement have omitted the perspectives of early adolescents. This research was designed to probe early adolescents attitudes and behaviours in this regard. To achieve this, questionnaires and focus groups were designed to probe what early adolescents think about the importance of community building activities, how active they are in the community, and how they see the creation of their ideal futures being realised. The questionnaires drew from two main sources - from a survey by \citet{Lakin2006} which probed attitudes around the importance of this efforts, and behavioural questions on community engagement from the Citizen Audit of Great Britain, 2000-2001 which covers a range of possible community engagement initiatives and probes the individuals attitudes towards these efforts. These, along with basic demographic questions, form the basis for the three surveys. Specifically, wave I will consist of the first survey - taken before the first focus group - asks about attitudes and behaviours, and the second survey - taken after the last focus group - will only ask about attitudes since behaviours will have unlikely changed since the first survey. Lastly, the wave II survey will be undertaken 6 months after wave I and will ask about both attitudes and behaviours again. 




The objectives for the interviews are straightforward: participants will be asked to describe their desired societies, what they think the capabilities of that society would be, and how they view their contributions to the creation of that society. They will be asked to comment on and identify valued capabilities of society, and explore which ones are both achievable and should be prioritised; explore how those will be achieved, who will be involved and whether they see their contributions as important to the actualisation of those goals. They will explore some of the different areas of social development, including economic, social, and intellectual. Identify their role in the creation of these capabilities, and how their existing individual capabilities will contribute to the realisation of these, or how they might need to develop those capabilities to achieve those results. If they are long term goals, it will be necessary to explore some of the steps required to achieve those, and what a realistic time frame might look like.



Basic demographics have been used to identify the likelihood of engagement (albeit this was a less distinguishing feature when other factors were taken into consideration), so the first set of questions are designed to assist in the identification of individual participants, and then subsequently to identify the likelihood of their engagement. This is done primarily through understanding the participants socioeconomic background through postcode and parents occupations. This is based on the assumption that not all early adolescent youth will know the parents income; this method has been used successfully in other studies to give a general indication of family income. Demographic questions consisted of 5 questions and were asked in every instance of the questionnairee. 

The next set of questions related to participants attitudes of engagement - the importance they placed on community engagement behaviours. This set of questions were used by \citet{Lakin2006}, and used a 5 point likert scale of importance. \citeauthor{Lakin2006}'s \citeyearpar{Lakin2006} study drew on a previous study which had a high internal reliability (=.75), and added a few additional questions around the community's future which he also identified as having a high internal reliability (=.83). Attitudinal questions were asked at every instance of the questionnaire. 

The questions used for the behaviours sections of the questionnaire are derived from the Citizenship audit which was undertaken in 2000. This survey looked at all aspects of life in Britain, and was discussed in the Citizenship in Britain: Values, participation and democracy book published in 2004. Question 6 of this audit identified a number of different categories of organisations (giving examples for clarification) in which someone might be engaged in community life. This was assessed according to different ways in which the individual might engage: member, participated, donated money or carried out voluntary or unpaid duties. Since early adolescents often engage in these sorts of actitivies alongside family members or peers, this was changed slightly to reflect this: the first two options were changed to read “Participated by myself" and “Participated with family or friends". Additionally, a few of the examples and categories were used to make it more socially, culturally, and age appropriate (while simultaneously trying not to assume that early adolescents are not interested in joining certain organisations). In addition, since the questionnaire will be administered 6 months apart, where the question asked about activities undertaken in “the last 12 months", this was changed to “the last 6 months" so as there would be no overlap of reporting. Since the behaviours were not likely to have changed in the week between the first and the last focus group, the behaviours section of the questionnaire was administered only at the beginning and at the very end of the research (6 months apart). 

The final set of questions are specifically related to the participants experiences of the focus group itself. This is a set of 4 questions asking participants about how comfortable they were to share their thoughts during the groups, and whether the conversations reflected their personal beliefs and attitudes. These questions are designed to elicit the sometimes felt tension around what an indiviaul believes and what they are comfortable to say that they believe around others. Despite efforts by the facilitator to ensure everyone feels comfortable to share their opinions, Since early adolescents is a time of increased social awareness, these efforts might be insufficient to do this, and as such, these questions will attempt to gauge this. These questions will be administered directly after both focus group sessions. 

%CHECK if these paragraphs go here, also proof
There is a possible tension that I might face with the discrepency between what an individual might explore on their own compared to what they might be willing to say or agree with in a group setting. In other research, [see Rosie's results] there was a noticable difference between what children identified as important, versus what they would identify as important when in a group setting. this is perhaps because they believe there is a certain level of acceptibility of discussing or prioritising certain things for an individual that does not exist to the same level when that person is in a group setting. In a group setting, this might present as individuals starting out with a certain belief, but as the discussion continues that understanding and perception changes, and the individual adopts or alters their beliefs accordingly. Alternatively, it may present as the individual not feeling comfortable enough to share their views, as it contradicts with the views of others, perhaps more dominant individuals in the group. This latter understanding may deprive the group of a valuable addition to the conversation, which may have given them a new perspective by which to understand the situation. 


In the context of this research, I hope to reassure the participants to extent possible that they should feel free to express their views openly. Irrespective, since the likelihood of this cannot be completely ameliorated, one of the first questions in the post-focus group questionnaires will be whether the individual felt comfortable to share their opinions, whether their opinions were in agreement with the final consensus of the group, and, if not, whether they were still happy with the outcome of the groups decision (ie whether their position may have changed somewhat to be more in line with the final group consensus). This will be done to ensure the focus groups functioned as well as we had hoped in respect to participants ability to share their ideas freely, and to garner additional ideas that may not have come up in the context of the focus group itself. 

Actually, the more I think about this, the more i realise that there may need to be an additional questionnaire at the end of the first session's focus group - asking simply about their experiences of the focus group itself, and whether they felt comfortable to share... 1 or 2 questions only. 


\section{Participants \& Recruitment}

The research will focus on early adolescents, aged 12-15 years of age; therefore the participants will be students from grades 7 to 9, from four schools in Victoria and New South Wales. Since individual development occurs during these early adolescent years \citep{Blakemore2012}, the focus groups will limit the age range of participants by having one focus group per grade and multiple grades represented from each school. 

To ensure a diversity of data, the hope is to get a range of students from schools with different demographics; schools will include single and mixed gender, and high and low socioeconomic communities. As explored in the literature review, gender and socioeconomic status are two of the factors influencing engagement practices. To achieve groups representative of this diversity, schools in known high and low socioeconomic communities will be approached, while single gendered groups will be established through single sex schools or by selecting group composition based on gender from mixed gender schools. 

Initially, I will make contact with schools by drawing on my existing professional networks. After outlining the research to school representatives, I will ask them to assist in identifying a willing group of students that are representative of a diversity of community engagement attitudes and experiences. % HERNAN SAYS - Careful, many times schools give you the 'best' and 'brightest' and skew idea of diversity

Establishing a conducive and comfortable atmosphere will begin from recruitment \citep{Gibson2007}. Since most participants will not have been involved in research previously, the plain language information sheets and initial introduction will expound the goals and procedures of the research. Early conversations will create some familiarity between the facilitator and participants and assure participants that their contributions are valued. 

\section{Wave I - Visit 1}
The first visit to each school will consist of a questionnaire and a focus group discussion. 

\subsection{Pre-Focus Group Questionnaire}
The initial questionnaires will cover demographic and background information of the respondents \citep{Johnson2010}. Since a portion of the youth engagement literature focuses on predictors of engagement, these initial questionnaires will provide insight into the likelihood and actuality of a particular participant being involved in community service activities. Questions will include date of birth (to determine age), parental occupation and postcode (to determine socioeconomic status), and family and individual community engagement practices. 


\subsection{Focus Group 1}

Focus groups will use a problem posing approach similar to that proposed by Freire. The problem posing approach develops people's critical understanding of the world and their relationship to it; they understand the world as the continual transformational process of reality \citep{Freire2000}. This understanding gives rise to an organic process of thought and action \citep{Freire2000}. Focus groups provide insight into: similarities and differences among respondents with respect to community engagement behaviours, experiences, interests, perceptions, opinions and attitudes; potential problems with new initiatives; how respondents talk about community engagement; and new ideas and creative concepts \citep[][p591]{Stewart2009}. These discussions, it is hoped, will produce ideas and conclusions that may not have been possible through other research methods such as individual interviews \citep{Stewart2009}. 

Activities will be used as part of the focus group to explore issues and encourage participation \citep{Colucci2007}. Activities will imbue the discussion with meaning and a deeper understanding of issues, whilst simultaneously avoiding triviality, as trivial approaches have proven ineffective in eliciting adolescent engagement \citep{Dawes2011}. Meaningful activities are important as a means to reduce potential anxiety and promote active participation. Activities may include: free listings - participants list elements that might contribute to a particular problem or solution; ratings - listed items can then be rated in terms of importance or relevance; and rankings - items can be ranked against each other \citep{Colucci2007}. Activities may also be used concurrently \citep{Colucci2007}. An example may include each participant being asked to identify desirable areas of social progress, which are then brought together and grouped into categories. The categories are named and items within and between categories ranked to identify capabilities and priorities. Discussions will be used to explore issues in more detail using questions such as “why is that more important?”, and “what criteria did you use to make that assessment?” \citep{Colucci2007}. 

Introductory remarks will establish the goals, format, character and confidential nature of the sessions, as well as duration and procedures of the discussion, some “get to know you” activities, an explanation of the facilitator's role, and participant questions and comments \citep{Gibson2007}. The facilitator will encourage participation and listening between group members; this requires patience, respect, active listening, flexibility, humour and a non-judgemental attitude on the part of the facilitator \citep{Gibbs1997,Gibson2007}. To avoid distractions, the researcher will request that all mobile devices be placed on silent or turned off for the duration of the session \citep{Dilorio1994}. Activities which elicit every individual's opinion will be utilised to ensure full participation and questions will generally be directed to the entire group, instead of individuals, to reduce participant anxiety \citep{Gibson2007}.

The first set of questions - to be discussed in the first session - will relate to the formation of a vision with the group - a vision of how they would like their society to be and what capabilities the community and each individual should have. In order to undertake this task, participants will be asked to each identify three things they think would change their community for the better. These will then be categorised and prioritised through reasoned discussion \citep{Sen2009} and be used to formulate a vision for the whole group that is at once realistic and bold, and that participants believe will create meaningful change in a manner which is valued by the community. The discussions will ensure a plurality of opinions, and explore any assumptions \citep{Sen2009,Sen2005}. Whilst it is envisaged that this age group will not have preconceived ideas about, for instance, economic priorities, the discussions will need to avoid false dichotomies that suggest economic development should be prioritised over cultural or traditional practices. 

\section{Wave I - Visit 2}
The second visit to each school will consist of the second focus group and a post-focus group questionnaire. 

\subsection{Focus Group 2}

The next set of questions - to be discussed in the second focus group session - will focus on how to achieve the established goals, what role youth have in the creation of this vision, and the relationship between their participation and the creation of that vision. This discussion necessitates an examination of existing social realities and capabilities, including its challenges and opportunities, its strengths and weaknesses. Both focus group discussions will follow a set of predetermined, open questions which explore key areas of community, social progress, and participation. The facilitator will explore ideas, clarify meaning and probe for explanation when necessary \citep{Gibson2007}. The subsequent session will also include time at the beginning to recap the previous session. Recapping will ensure accuracy and boost their recollection to provide a more fruitful discussion. 

\subsection{Post-Focus Groups Questionnaires}

The focus group will be followed by a questionnaire probing the participants experience during the focus group itself. It will explore their experiences of focus group proceedings, whether the outcome was a reflection of their views and if not, whether they were still happy with the outcomes. While it is hoped that the focus group will consolidate the views of all participants, this questionnaire will attempt to capture any views from participants who may not have felt confident to share their perspectives previously \citep{Tisdall2008}. This questionnaire will be conducted according to Likert-type scales and include short answer questions.

\section{Wave II - Follow Up Questionnaires}


Whilst the focus group discussions are not designed for the specific purpose of eliciting action in the participants, the use of critical theory as an approach makes this a foreseeable possibility. Additionally, development during this period is significant, and changes in thoughts, attitudes and behaviours are likely \citep{Pfeifer2012,Blakemore2012}. As such, an additional questionnaire will be undertaken 6 months after the focus group. This questionnaire will ask about ways they have since engaged in the community, and the regularity and form of that engagement, as well as general thoughts and attitudes towards community service \citep{Johnson2010}. 

\newpage
\section{Participant Commitment}	
%	\begin{wrapfigure}{r}{8.5cm}
%		\includegraphics[width=0.5\columnwidth]{ParticipantCommitment}
%		\caption{Expected Participant Commitment}\label{wrap-fig:1}
%	\end{wrapfigure} 
\begin{wraptable}{c}{7cm}
	\caption{Expected Participant Commitment}\label{wrap-tab:1}
	\begin{tabular}{lr} \\\toprule  
		Questionnaire I & 5-10 mins \\\midrule
		Focus Group Session I & 90 mins \\  \midrule
		Focus Group Session II & 90 mins \\  \midrule
		Questionnaire II & 5-10 mins \\  \midrule
		Questionnaire III - Wave II     & 5-10 mins \\  \midrule
		\textbf{Total Maximum Time} & \textbf{3.5 hours} \\  \bottomrule
	\end{tabular}
\end{wraptable} 

It is estimated that the time required for the focus groups will be two 90 minute sessions, while the questionnaires should take less than 10 minutes each. The total time required of each participant would not exceed 3.5 hours. See Table~\ref{wrap-tab:1}.

\section{Data Analysis}


The questionnaire and focus group responses will be coded and analysed to identify emerging themes and to draw conclusions from these data. Coding will be undertaken through an iterative process of reading interview transcripts, identifying themes, and relating themes to the research questions and hypotheses. To analyse interview questions, both open coding \citep{Strauss1998} and theory-based content analysis \citep{Stewart2009} will guide understanding on both the nature of their contribution as well as emerging themes. Additionally, individual participant responses will be analysed to compare changes over the 6 months between wave I and II and differences between individual and group contexts. 

These data will then be mapped and interpreted for: words; context; internal consistency; frequency; intensity of comments; specificity of responses; extensiveness; and big picture \citep{Rabiee2007}. This process will be undertaken using qualitative data analysis software Nvivo \citep{Stewart2009}. My previous research experience has familiarised me with Nvivo software, and I will also refresh my skills with an updated training course. 

\section{Ethical Considerations}
The proposed research has considered a range of ethical issues. Of primary concern is the vulnerability of research participants - early adolescents aged 12-15. Both participants and parents will be provided with information sheets and consent forms; for the protection of participants, individuals will be made aware of the nature of the project and their rights prior to consent. Should any participant raise concerns relating to cultural beliefs, customs or personal issues, they will be reminded that at any time participants have the right to withdraw from the research, as well as physically remove themselves from group discussions, and the procedure for them to do so. The procedure for students withdrawing from groups discussions will be established in consultation with the school. 

Whilst there is no expectation for questions to bring about negative reactions, the researcher will include on information sheets, and have on hand, information about freely available counselling services. Additionally, since this research is likely to be conducted on school grounds, during school hours, a school counsellor may also be available for students to consult. Participants, parents and schools will be made aware of Working With Children checks undertaken by the facilitator, as well as past experiences and training in relation to young people. All data will be de-identified. 



%FIND SOMEWHERE TO PUT THIS
 All study procedures were approved by the University of Melbourne Ethics Committee and informed consent was obtained from all participants and their parents. The researcher has extensive experience running youth programs and facilitating group discussion. The participants were told at the beginning of the workshop that they can leave at any time if they are feeling uncomfortable or for any reason. Throughout the discussion they were encouraged to... This methodology offered the least intrusive approach for the full participation of all students.