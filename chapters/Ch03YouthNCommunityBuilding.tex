\chapter{The Act of Engaging}

\begin{bclogo}[couleur=blue!30, arrondi=0.2, logo=\bclampe]{“Like a sculptor, if necessary, carve a friend out of stone. Realize that your inner sight is blind and try to see a treasure in everyone.” }
	- Rumi.	
\end{bclogo}


%INSTEAD OF highlighted use the word underscored (sometimes)
\label{YP&CB}

The last chapter discussed how young people develop in the context of community and in particular community engagement. I used the term community engagement as a catch-all phrase to include any engagement, from participating in sports and clubs, to volunteering, etc. In this section I will attempt to differentiate community engagement from terms such as community service and community development. This is done partly because the literature on youth engagement was often used by different researchers in different contexts and hence was difficult to distinguish, and also because few spoke of community service (or in the context of education it was often referred to as service-learning) specifically. Hence, this chapter will differentiate between community service - an act or project undertaken by one or more individuals for the greater good - and community development - a more systematic approach to ongoing efforts aimed at transforming a community. Community engagement will, then, be discussed as mere engagement in community organisations which often serve as a means for connection, but are not primarily for the benefit of others.

\begin{table}
\centering
\caption{Forms of Engagement}
\label{FormsOfEngagement}
\begin{tabu} to 1.0 \textwidth  {|| X[ 1 ] | X[ 5 ] ||} 
\hline
Community Engagement   & A catch-all phrase to include any engagement, from participating in sports and clubs, to volunteering, etc \\ 
\hline Volunteering  & Volunteering Australia's definition of volunteering changed in 2015, from acts of unpaid work for formal institutions, to a broader definitions including any act that benefits the common good and does not elicit monetary gain. This earlier definition is often reflected in the usage of the word as well, and because of this ambiguity I do not use the term volunteering, but draw on the updated definition when referring to community service \citep{ClaireEllis2004,VolunteeringAustralia2015}. \\ 
\hline Community Service         & Actions undertaken which benefit the common good, generally not for monetary gain, and often done as part of a group. \\ 
\hline Participating In Community Development & Participation in community development(or building) is more systematic, ongoing actions of community service which are aimed at contributing to social or community development.    \\ \hline
\end{tabu}
\end{table}
 %HELEN SAYS - Along the way define key terms such as engagement and development. Point out how they are used differently by people in different traditions of scholarship
 
 
\section{Participation and Empowerment} 

For young people to realise their desire to partake in the creation of social change, they must engage with society, in whatever form. This often comes in the form of community or civic engagement, volunteering, or community service. Community engagement, sometimes referred to as youth participation, can be understood as a spectrum of involvement, from tokenistic attendance and having a voice to more influential democratic community consultations and engaging as responsible citizens who can meet a particular need \citep{Camino2002}. Volunteering generally refers to ongoing efforts to assist strangers, and is usually carried out in an organisational setting. Community service however, requires an individual to be engaged in sustained prosocial actions \citep{Penner2002}. It is this latter definition, participating in sustained, civically-oriented, prosocial voluntary activities in one’s community which is most likely to bring about the change that young people seek. Yet not all theory or research has taken this strict definition, and as such this thesis will often use a catch-all term for the participation of young people in community activities - herein referred to as `community engagement’ or simply `engagement'. 



Development must go beyond notions of participation, since participation cannot be equated with empowerment. Individuals must be able to choose not only in how they participate, but also whether they participate in existing structures, or create new ones. Participation in a flawed system merely perpetuates current patterns of injustice. As such, an individual needs to become empowered so they are able to effectively assess the strengths and weaknesses of existing social structures, and then be able to freely participate in them, work to reform them, or even participate in the creation of new structures. By simply participating in existing social structures, structures designed and built by those with power and resources, individuals may be contributing to existing inequalities they are trying to eradicate.

There is a belief that the younger generation (under 20) are less interested in participating and do not have the civic values which support participation \citep[][p17]{Pattie2004}. This belief is based on the idea that younger people are less invested in the system. This idea may be challenged when considering different forms of participation, and whether those forms of participation contribute to existing social structures, or whether the participation itself is an indication that they are undertaking the task of building new systems of governance (or at least engaging in systems which are not familiar to the general society). This must also be viewed as a legitimate avenue to participation.

Despite noble intentions, many efforts to empower communities, to remedy injustices, place responsibility on the very people who struggling from the effects of these injustices, struggling from the systems which deny them access to full participation and decision making opportunities. Efforts need to recognise this phenomenon, and take responsibility - as communities and as social institutions - to open up possibilities for every individual to contribute to the creation of equitable and just communities.


%WRITE UP THIS SECTION
Empowerment is often conceptualised as a linear process, often constructed as a gradual forward-movement. However empowerment is rarely straightforward, but dynamic, it is better conceptualised as a “dance that takes two steps forward and three steps back before moving slowly in a spiral around the floor”  \citep[][p54]{VeneKlasen2002}.

Many of the definitions of empowerment are rooted in the advocacy work with women and poor communities, focusing on the elimination of barriers and building confidence to overcome exclusion and powerlessness \citep{VeneKlasen2002}.

Critical consciousness is a critical aspect of change within the individual, and it is this change, personal change, which can often be more difficult than institutional chance. Individuals will often resist the urge to change even if it is for their own benefit \citep{VeneKlasen2002}.

Empowerment requires all individuals, men, women, white, black, rich and poor to question their roles and their society. It can cause discomfort and anger. It changes depending on the social space and the individuals which occupy them. It can often be so confronting or involve potential violence that some may opt out of the process, despite the undoubted long-term benefits \citep{VeneKlasen2002}.

One depiction of the empowerment process, the Chaz Framework, demonstrates the integral nature of empowerment through individual and collective transformation - as the two processes connect, both become empowered. It also identifies the importance of this being an unending process of construction and deconstruction in which the process of conscientization which continually questions ideas of equality, equity, rights and discrimination \citep{VeneKlasen2002}.



\section{Who Engages?} 
When thinking about the predictors, one might also need to consider how these predictors impact the likelihood of engagement. \citet{Metz2003} suggested these predictors were actually an indication of the resources available to them, in both the material sense - as in time, money and skills - as well as cognitive and emotional disposition - a sense of efficacy and interest in community affairs, as well a psychological concern for the common good.

A number of studies examined individual characteristics to identify the extent to which they influence the likelihood of engagement. A longitudinal study looking at Australian high school volunteering, for example, found gender, family, society, values, religion and socio-economic status all influence volunteering practices: higher levels of education, private and rural schools and full time education all positively influenced volunteering rates; pessimism led males to volunteer, whilst optimism led females; and tertiary education and mothers working outside the home increased volunteering in females \citep{Brown2003}. In addition, this study also suggests volunteering is a habit - those who engage at a young age are more likely to continue their engagement through different life stages \citep{Brown2003}. However, if individuals are able to overcome these potential barriers, there are advantages. 


\subsection{Demographics}
A pertinent body of empirical work has been undertaken on understanding distinct paths to engagement. One body of work is looking at demographics, yet this approach is perhaps the least accurate form of prediction, and becomes almost indistinguishable as a predictor once other factors such as identity and purpose have been accounted for \citep{Taylor2007}. Irrespective, factors such as age, gender, education, socio-economic status, religion and marital status have all been found to impact engagement rates. Age for instance, was identified as correlating to engagement in a study looking at volunteering as a form of community service. \citet{Penner2002} conducted an online survey and reviewed existing literature, and when comparing volunteers, age was positively associated with both number of organisations and the length of time an individual spent working for that organisation. In a similar finding from a study of religious volunteers who participated in surveys, \citet{Garland2008} also found that volunteers tended to be older church goers. Longitudinal research has also identified that those who engage during their formative years, are also more likely to engage as adults \citep{Barber2013}. Even amongst youth themselves, a short-term longitudinal study looking at youth from 15 years of age found that younger adolescents were more likely to engage by \citep{Cemalcilar2009}. Linking these studies, it would suggest that the younger an individual initially engages, the more likely they are to engage throughout their lifetime. 

Gender was also analysed in two of the above mentioned studies. For adult volunteers, whilst the likelihood of engagement was significantly higher if one was female, 77\%, the number of organisations, as well as the length and time of volunteering commitments did not differ between genders. This trend was seen, albeit less noticeably, even during adolescence. The study by \citet{Cemalcilar2009} found that 56\% of volunteers and 45\% of non volunteers were female. 

Education and perceived intelligence both significantly influenced engagement. In the above mentioned \citet{Penner2002} study, levels of education were positively correlated with number of organisations, and the length and time to which a volunteer committed. A similar study compared religious volunteers to non-volunteers and found that those with tertiary education were more likely to volunteer, yet less likely to report a change in values as a result of volunteering \citep{Garland2008}. Additionally, longitudinal research on young people identified that as an individual gets older, higher levels of education corresponded to higher levels of community engagement \citep{Barber2013}.

For socio-economic status however, income was not likely to affect the likelihood of engagement, but the number of organisations with which an individual was likely to engage \citep{Penner2002}. Socio-economic status was not found to be a predictor of engagement in a study in which education of mothers was also used as a proxy for socio-economic status \citep{Barber2013}. The authors suggested that this might arise from an increased likelihood of higher education from students whose mothers are also highly educated, from which point the students education is more influential on engagement since higher forms of education often support engagement practices. Similarly, in another study which followed African-American youth from early adolescence (M=13.1 years), family income became less important as other variables such as spirituality (the experience of the sacred in their lives) and receptivity to mothers were considered \citep{Smetana2005}. 

The concept of spirituality is closely linked to religiosity, which can be seen to influence engagement significantly. In the \citet{Penner2002} study, religiosity was found to have the strongest association with volunteering, in the number of associations, the length and time of volunteering. Even for studies that looked solely at religious participants, there was a strong positive association between those who were long time members of the church and attending services more regularly and volunteering \citep{Garland2008}. In a study looking exclusively at recipients of volunteering awards, surveys and interviews were utilised to identify that 85\% of respondents found religious or spiritual reasons as significant motivators, despite half the respondents being selected on the basis of non-religiosity. Most did, however, cite additional motivations alongside religion, including the importance of community \citep{Perry2008}. The previously mentioned longitudinal study with African American youth identified the importance of spirituality during middle adolescence as a key predictor of community involvement 3 years later in late adolescence \citep{Smetana2005}.

% WHILST SOME CURSORY INSIGHTS CAN BE GAINED FROM UNDERSTANDING DEMOGRAPHICS, USING MORE NUANCED ANALYSES HOWEVER, PROVIDES AN UNDERSTANDING OF OTHER ASPECTS OF AN INDIVIDUALS LIFE CREATES GREATER UNDERSTANDING AROUND PARTICIPATION, AND IN PARTICULAR THE IRRELEVANCE OF DEMOGRAPHICS. 

\subsection{Qualities and Attitudes}

Few studies have also examined the qualities an adult is likely to possess that increases the likelihood of engagement, fewer still have looked at young people. \citeauthor{Penner2002}'s \citeyear{Penner2002} study comparing volunteers with non-volunteers identified that volunteers scored significantly higher on scores for other-oriented empathy and helpfulness. \citeauthor{Omoto1995}'s earlier \citeyear{Omoto1995} study looking at motivations for volunteering with AIDS patients also identified helpfulness, alongside other values. In addition, however, a concern for community, personal development, understanding and enhancing self-esteem were also identified. \citeauthor{Ballard2014}'s \citeyear{Ballard2014} study used semi-structured interviews of school children in divers settings. This study will be discussed in more detail in section \ref{Motivation} on motivations, however one finding emphasised that young people engage because of a belief in the importance of civic action.



\subsection{Environment}

A discussion of who engages cannot be complete without a discussion of the impact the environment has on individual engagement; family, school, and community contexts are known to impact engagement likelihood, frequency and form. From a study on religious volunteers, marriage was identified as a correlate for increased engagement \citep[]{Garland2008}, while for African American adolescents religion and spirituality was important, as well as the influence of mothers \citep{Smetana2005}. Interestingly, in the above mentioned study on AIDS volunteers, perceived social support actually has a negative correlation to volunteering longevity, although a marginal level of social support is required \citep{Omoto1995}. Arguably the study by \citep{Duke2009} provides some insight into this seeming contradiction. In their longitudinal study of young adults, they found that the quality of connected significantly affected the likelihood of engagement. Put simply, the stronger a young person feels a connection to family and community, the more likely they are to participate in that community, contributing to its development in a myriad of ways - community service, social action groups and voting. 

This community involvement, as well as other extracurricular activities, can also influence engagement, particularly for young people. In three longitudinal studies in the United States, engagement was most likely increased if individuals engaged as adolescents \citep{Obradovic2007}, more specifically in a combination of required and voluntary service during their adolescent years \citep{Barber2013}, as well as participated in community (not political) and church \citep{Smetana2005}. Yet a study with American and Italian college students found that external incentives may be detrimental to intrinsic motivations for individuals to volunteer. Not surprisingly, those with amotivation were significantly less likely to volunteer. However student with high extrinsic motives, despite also having high intrinsic motives, volunteered less frequently than those with high intrinsic but low extrinsic motives \citep{Geiser2014}. The final noteworthy study is the importance of life-changing events. The study identified that 25\% of volunteers cited the importance of life-changing events being a catalyst for engagement \citep{Perry2008}. Some may argue that adolescence itself is a life changing event, for as Anna Freud \citeyearpar{Freud1958} states “adolescence is by its nature an interruption of peaceful growth... [it] is in itself abnormal” or in the words of \citet[][p72]{Erikson1956} “adolescence is not an affliction but a normative crisis.”





\section{Benefits of Community Engagement} 
\subsection{...to the community} 
\label{Benefits}
A focus on youth in community engagement is useful in revealing broader social benefits; it provides insights into individual development as well as future community development possibilities. Theorists discuss the importance of volunteering, particularly youth volunteering, for effective social change. Fukuyama, for example, suggests that true and lasting change can be accomplished by winning the hearts and minds of individuals and communities \citep{Fukuyama2006b}. % % HERNAN SAYS - winning hearts and minds can be a top down exercise. Careful. This sounds like US army winning hearts and minds of Afghans. Check where that got Afghans. Its now called Nation Building. Careful with Fukuyama, Jeffrey Sachs, and developmental theorists of the West and their top-down approach. Different in Sen's position. 
% % SEE Fishstein2012 for conclusions indicating the negative impact of this approach (an approach which largely relies on monetary gain and often perpetuates corruption, but also Beath2013 for conclusions which indicate positive impact except where initial levels of violence are high (considering the goal is to overcome insecurities and the problems of counter insurgencies, you would expect there to be high levels of violence, hence not sure how they came to the conclusion it is positive). 
Winning the hearts and minds is also dependent on the approaches used, in particular whether such efforts use a top-down or grass-roots approach. Top down approaches of winning the hearts and minds often use economic incentives and military force to create change, however there needs to be an understanding that development takes significant investments of time and resources to empower individuals and to allow them to develop trust in those assisting with this transformation. This was identified by \citet{Fishstein2012} in their analysis of effective development. They identified that quality programs that make greater investments of time - not money - are better at building and sustaining relationships and resulting in more positive development outcomes. Alternatively, when “winning the hearts and minds” is the implementation of development initiatives which institute programs and facilitate grass-roots growth, the effects can be positive, ranging from greater well-being, more positive attitudes to local and foreign military forces and NGO's and reduced security incidences \citep{Beath2013}. This process likely takes longer because of the immense injustices which have been heaped on these societies, which, by many accounts, have been at the hands of Westerners. As such, for Western aid organisations to come in and expect their “saviour” facade to be seen over any other previous challenges would be short-sighted. 


Winning the hearts and minds of youth has also been reflected in Erikson’s work, suggesting it is the key to social evolution. % %HERNAN SAYS - whose social evolution? 
According to Erikson, adolescence is “...a vital regenerator in the process of social evolution; for youth selectively offers its loyalties and energies to the conservation of that which feels true to them and to the correction or destruction of that which has lost its regenerative significance” \citep[][p126]{Erikson1994}. Erikson's discussion of evolution is also interesting to consider for he speaks of evolution both in terms of human development as well evolution of virtues. He cautions that the tendency to link virtues with evolution as a “naturalisitic fallacy” - that `good' is thought of in terms of `desirable' or `pleasant' properties \citep[][p141-2]{Erikson1994}. Although idealistic, Erikson believed in the capacity for social evolution based on the importance he placed on identity. Identity to act in an intelligent pursuit of one’s goals, reinforced through social relatedness which develops trust, understanding and cooperative feedback, and - perhaps most relevant to this discussion - guided by their ethical framework \citep{Youniss1997b}. 


In his article \textit{The Golden Rule and the cycle of life} explores how adolescence is a time to develop ones ethical framework. He proposes a difference between morals and ethics which centres on their area of application: morality considers the individual, their outer security and sense of self as well as their internal feelings of shame or guilt, while ethics considers a wider scope of ideals, based on a higher perfection to which one should strive \citep{Erikson1963}. Their awakened sense of the future leads them to consider their own ideological position \citep{Erikson1963}. For Erikson ethics “can not be fabricated; it can only emerge from an informed and inspired search for a more inclusive human identity, which a new technology and a new world image make possible as well as mandatory” \citep[][p418]{Erikson1963}. He goes on to reinforce the importance of this ethical ideology emerging through engagement in the community: “one can study the nature of things by doing something to them, but one can really learn something about the essential nature of beings only by doing something with them or for them” \citep[][p420]{Erikson1963}. The development of ideology and ethics are, according to Erikson, firmly in the realm of the adolescent \citep{Erikson1963}. Whilst this statement was originally published in 1964, viewing adolescents as vital regenerators was largely overlooked for Erikson’s 1968 discussion of adolescence as a stage of crisis \citep{Erikson1968}. Rarely, if ever, do community programmes understand young people as a resource in the process towards community growth and transformation \citep[][p105]{White2008}. There is also a revealing pattern about the common perceptions of the capacities of young people, with much of the sociological literature showing a paradox of ambiguous expectations: they are encouraged towards freedom and ‘to make a positive contribution to society’, and simultaneously identified, both in media and in policy, as instigators of anti-social or harmful behaviours \citep{Ellis2009,Wyn2011}. 


\subsection{...to individual contributors}

For those who do channel their energies into community service there are benefits for the individual including the development of pro-social identity and qualities. US research by Youniss and colleagues identified a number of benefits ranging from identity, pro-social qualities, mental health and increased academic performance \citep{Kerestes2004, Youniss1997b}. In their 1997 paper, for instance, they examined high school civic behaviours and subsequent adult participation, identifying that adolescent engagement forms the basis of a strong civic identity with a sense of agency and social responsibility \citep{Youniss1997}. Later, they suggested that morality formed a developmental link between actions and identity \citep{Youniss1999a}. This identity-morality integration was particularly evident in the response of one research subject, reflecting on their actions to overcome injustice by saying: “I acted as I did because I am who I am” \citep[][p373]{Youniss1999a}. This identity-morality development, then, informs an individual's relationship to others and society, extending one's values and sense of justice beyond their own group or community and is observed when youth apply their sense of agency toward the development of society \citep{Youniss1999a}. Presently, identity development is understood to begin in early adolescence, however it solidifies later in adolescence, or even young adulthood \citep{Steinberg2003, Meeus2010}. 

Community service then becomes a powerful deterrent for anti-social behaviours. Indeed, a study looking at 13,000 high school students identified that lack of involvement in community service was a more accurate predictor of deviant behaviour, in this instance the illegal use of marijuana, than individual background characteristics \citep{Youniss1999}. This may, in part, be due to the establishment of supportive networks while serving. A US study found that engaging in supportive cultures - whether they be youth cultures or wider social cultures such as volunteering organisations - helps youth avoid risk taking behaviours \citep{Kerestes2004}. In another study looking at social change and identity with at-risk early adolescents in New Mexico, the researchers found that participation during early adolescence significantly reduced alcohol consumption. This applied to all participants, but was particularly noticeable for females (mean = 5.58 down to 1.16), and it remained low after an 8 month follow up (mean = 2.05). They identified that at-risk youth experience greater benefits from engaging in community service programs \citep{Verlande2002}. Whilst it is difficult to argue that engaging youth as a means to deter anti-social behaviour is not beneficial, the question must be asked - is it the only reason we are wanting them to engage? Can we view them as a potential resource for community building instead? While there are significant numbers of youth programs aimed at “troubled” youth or at reducing antisocial behaviours, there needs to be a conscious effort to broaden our vision of community youth programs, ensuring that the ideas and efforts of youth are appreciated for the benefits they bring to society not merely the reduction of potential problems. 

In addition, research has identified a range of positive qualities and behaviours resulting from youth engagement. Such research has identified benefits such as critical thinking \citep{Gellin2003}, academic, behaviour and civic gains \citep{Schmidt2007}, and positive identity development in relation to normative, unconventional and deviant behaviours \citep{Youniss1999}. The Australian longitudinal study mentioned above also observed that volunteering increases levels of overall happiness \citep{Brown2003}. In fact, the importance of youth engagement was underscored by more recent research which suggests that many prosocial qualities are more likely to be acquired if developed during adolescence. This study looked at qualities and attitudes related to civic involvement and found that civic knowledge saw a sharp increase prior to 17 years of age, after which it plateaued \citep{Hart2011}, which may be indicative of a developmental window particular to adolescence. One final study is noteworthy in this respect: a US study of almost 500 high school students examined attitudes and behaviours towards community service. This study identified that youth who had fewer predictors of engagement, when required to do community service, showed greater gains from their experiences compared to those with numerous predictors. In other words, those who were less likely to engage derived greater benefits, particularly in schools mandating such involvement \citep{Metz2005}. 




\subsection{Early Adolescent Community Engagement} 
There is an awareness about the importance of development happening at certain stages of an individuals life, and there is strong evidence to suggest that if this stage is missed, it can be difficult for an individual to develop fully in that area. The few detailed studies on early adolescent community engagement which do exist are all the more valuable because of their scarcity. However, to date there is insufficient research to provide a coherent understanding of the whole. A significant body of research exists to suggest the same applies for the development opportunities arising as a result of engaging in community initiatives \citep[see for example][]{Hart2011}. This section will explore some of the development opportunities of individual and social development as it applies to early adolescent youth. It is in no way prescriptive or to suggest that individuals are unable to gain these skills in other ways, but whether engagement is an effective means in which to develop these skills. 

As discussed previously, youth who engage are more likely to become adults who engage. However there is also evidence to suggest that throughout the lifespan of an individual, a individuals alter the amount of time they are able to dedicate to community engagement initiatives; both Australian and US statistics demonstrate the declining involvement as the individual ages. In the United States for example early adolescents were 30\% more likely to engage in community activities than their older youth counterparts \citep{Spring2006}. Similarly, whilst there was no data about Australian early adolescents in particular, the Australian Bureau of Statistics saw a similar trend when surveying all Australians over 15 year olds, with 15-17 year olds having the highest volunteering rates of any age group \citep{Yates2015}. These statistics suggest there is strong motivation and perhaps greater availability of time for young people to engage which decreases with age. In addition to motivation and availability however, there is some evidence to suggest early adolescent engagement is more beneficial than mid or late adolescent engagement. 

A frequently heard argument for adolescent engagement is to reduce problem behaviours. This has certainly been found to be the case for early adolescents, more so than middle and older adolescents. An example of this is the two-fold increase in alcohol consumption between grades 7 and 9 \citep{Verlande2002}. This information was used as a basis for a social change study in New Mexico examining changes in anti-social behaviours for early adolescents. The study found that participating in social change programs significantly reduced alcohol consumption, and this effect lasted at least until the 8 month follow up questionnaire \citep{Verlande2002}. For males, there was a 40\% reduction of frequency in alcohol consumption compared to the control group, whilst females who participated in social action saw a reduction down to 20\% of previous drinking rates. 

Moreover, in a US study looking at community service programs for middle school and high school students (early and mid-adolescents), the researchers identified better outcomes for early adolescent students than those who engaged during their high school years in the areas of academic performance, school suspension and pregnancies. In addition, programs which encouraged students to select their own service - service that was enjoyable, taught them new skills and challenged them to think about the future - was more effective in the reduction of problem behaviour in middle school students (early adolescents), but not high school students \citep{Allen1994}. The researchers suggested that high school students are likely to have already found opportunities to develop their autonomy and identity, and as such did not gain as significantly from these programs. Early adolescents, however, derived significantly greater benefits, including reduced risk taking and anti-social behaviours \citep{Allen1994}. Recognising that older youth have already had opportunities to become empowered and connected to their community captures the importance of targeting efforts towards these earlier adolescent years, whilst also highlighting a lack of understanding around this age group. \citet{Allen1994} suggests that programs targeting early adolescents may do well to focus on providing opportunities and discussions for students to become independent and capable individuals. This caters to their specific developmental requirements instead of trying to `manage' them, but doing so in a productive, service oriented environment \citep{Allen1994}. %ADD THIS - Eccles notes an important shift in motivation, achievement and positive development around the start of early adolescence which she seems to attribute to the transition to middle school... \citep{Eccles2009a}

Going beyond the reduction of problem behaviours to view young people as potential assets, a few studies give us a glimpse into the development opportunities of engagement at this age. In a similar pattern of benefits observed in the previous study, another US study analysing service-learning programs surveyed young people from grades 6 to 12, and found that grade 6 students displayed greater acceptance of diversity after engagement, which did not hold true for older youth \citep{Blyth2014}. Meanwhile, a more in depth study of empowerment programs for early adolescent youth in India found social action was an effective means to generate and apply new knowledge which improved the quality of their lives \citep{Merchant2013}. The researcher found this to be particularly true for changing age and gender power dynamics - an ongoing issues in many Indian villages. On the issues of age relations for example, the researcher found that older children had learnt to abuse the power they had over the younger members of the community, instilling fear and mistreating them. However through this program the young people not only learnt to show care and respect for the children of the village, they created spaces for them and the children to learn together about positive behaviours, resulting in bidirectional love and affection \citep{Merchant2013}. 

The nuances of gender inequality were also explored. The study found that although there was discussions and increased awareness about the importance of gender equality, the boys continued to dominate the discussions assuming that girls would interrupt them if they had something to share. Whilst this furthered the group discussion, it underscores the subtle distinctions of gender discrimination that are often entrenched in culture and the slow nature of social change as discussed in \ref{RosaParks}. \label{RosaParks}%FIX THIS REFERENCE
While the overall process of social change may take decades, noticeable changes in dynamics can be seen in a relatively short period of time. In the case of this empowerment program, it took only months for young female participants to see themselves as capable and actively assisting the male participants in their reading \citep{Merchant2013}. In the context of Indian village life, girls often undervalue their role in society, believing they should not aspire to careers and hence their studies are futile. Empowerment programs in this region were found to alter this mindset, resulting in girls increased interest in study and increased aspirations for careers \citep{Merchant2013}.


\citet{Merchant2013} suggests that empowerment cannot be measured, hence identifying predetermined indicators can create a bias around the outcomes of empowerment programs, narrowing the scope of understood outcomes. In her findings, each youth expressed different outcomes as significant moments of learning, demonstrating the unique outcomes of empowerment for every individual. As eloquently identified by \citet[][p23]{Merchant2013}, “it is problematic to define empowerment as a linear process whose outcomes can be anticipated”; it is arguably more constructive to conceptualise empowerment as “a dance that takes two steps forward and three steps back before moving slowly in a spiral around the floor” \citep[VeneKlasen and Miller 2006, p54 as cited in][p23]{Merchant2013}. In this context, empowerment must then be understood through a more holistic lens, in a way that views this process as an ongoing struggle to generate and apply knowledge which can be used to improve their own circumstance \citep{Merchant2013}. \citet{Merchant2013} concluded by observing the importance of undertaking social action as a collective experience instead of as individuals. This assisted individuals to develop group-decision making skills, understanding the importance of, and building, a strong group dynamic, as well as learning from other points of view. For \citet{Merchant2013}, the collective space for action and reflection was a critical component of the empowerment process. These insights give us a glimmering into the importance of this age, however more needs to be understood about whether happenings during the early adolescent years may serve as developmental precursors to later civic participation. 


\subsection{Factors Affecting the Engagement of Young People}
The likelihood and form of youth engagement can be impacted by these vacillating social perceptions. Adolescence is frequently described by society and many theorists as a time of crisis or stress. Aristotle, for example, criticised young people for their lack of self-control, instability and impatience \citep[][p12]{Santrock2010}. The early 1900’s saw adolescence being characterised as “the storm and stress period” \citep[][p73]{Hall1904} in which an individual experiences a temporary mental illness \citep{Mueller2009}. Today, there is still ongoing academic debate challenging the notion that all adolescents go through this tumultuous stage \citep{Muuss1975}, yet many societies assume it is axiomatic for this age group \citep{Stevens2007}.

These concepts seem to permeate social attitudes towards youth, and in doing so give youth the impression that their input is not valued, however along with notions of crisis, theorists also acknowledge the burgeoning capacities of adolescence. Plato for example recognised the development of the power of reason, while Aristotle saw the emergence of self-determination and choice \citep[][]{Santrock2010}. Rousseau, on the other hand, saw the development of self-consciousness and reason occurring from around 12 years old, and recognised that emotional maturity came years later \citep[][]{Santrock2010}. Piaget identified new capacities including abstract thinking and the understanding of long-term consequences \citep{Dulit1972}. Adolescence is seen as “...the most critical period of life” in which the individual “now realizes in a deeper sense the meaning of maturity and is protensive towards its higher plateau” \citep[][p72]{Hall1904}. It is a time of idealism in which they look ahead to new possibilities. The altruism experienced during adolescence is discussed in terms of a previously unseen self-sacrifice in an effort to improve the lot \citep{Hall1904}. The juxtaposition of these views contributes to young people vacillating between extremes - adolescence being a time of both virtue and vice: cruelty and sensitivity, conservative and radical, selfish and altruistic \citep[][]{Hall1904}. 

Despite this vacillation, youth engaging in community often coincides with a growing awareness of the capacity of young people and the importance of their contribution \citep{Holdsworth2000, Holdsworth2007, Wyn2011}. One study on “the impacts of youth on adults and organizations”, for example, demonstrated that the integration of youth resulted in a cultural change in an organisation. Youth integration necessitated a greater focus on learning and vision, and resulted in an increased appreciation of the contributions of youth by their adult counterparts. This study did acknowledge however that little could be understood about the impact on the general community without ongoing participation from multiple organisations encouraging youth engagement \citep{Zeldin2000}. This study, alongside Erikson’s quote identifying youth as a vital social regenerator, both hint at a capacity and energy unique to this period of life, and highlights the importance of youth inclusive engagement initiatives. Whilst Erikson saw the value of youth for society, both individual and social factors seem to influence youth engagement rates. 

Influential theorists such as Sigmund Freud and his daughter Anna, as well as Erik Erikson all wrote about adolescence as a period of crisis; this notion still permeates many social discourses today. However the validity of this notion is being increasingly questioned, with most theorists now rejecting the idea of a universal period of stress, instead suggesting that the experience of most adolescents is stress-free \citep[Bandura 1964 and Offer 1969 according to][]{Muuss1975}. 

Many developmental theorists have offered insights into normative adolescent development. Both Mead and Piaget speak about the period of adolescence as one of re-examination and one in which the individual forms their identity \citep{Muuss1975}. This process of identity formation, however, is influenced by the increased critique of what is being presented to them; it is often a time in which they critically examine social and familial ideologies, rejecting or revising their beliefs and solidifying their own standpoint. Piaget extends this notion of re-examination as a “duty” by which the adolescent must “revolt against all imposed truth”, and in doing so build ones intellectual and moral character. In this context it is easy to see why incorporating adolescence into community building approaches, in particular critical theory approaches of community building, would be advantageous. With their thinking already inclined towards critiquing previous traditions and ideologies, and the available time and energy accessible to most youth, channelling this energy would naturally lead to discarding of outdated ideologies. Additionally, whilst engaged, these youth will also build their own capacities in this manner, and refine the practices of building communities, but also they contribute to the body of knowledge being generated \citep{Muuss1975}. 



\subsection{Foundations Developed while Engaged with Community} 

Sen's approach makes certain assumptions about the capacities of individuals - they have access to information and skills that allows them to make an informed decision and that they have the capacity to make decisions about what is in their best interest. In order for individuals to make decisions about what they value and what they desire for their own development, arguably one will need to have developed - at least to some extent - critical thinking skills and pro-social identity; these abilities and characteristics are often developed during adolescence. This is not to suggest that each individual should not have the opportunity to choose, but it does identify the importance of certain foundational capabilities for people to be able to make the best decisions about desirable capabilities. A child, in an extreme example, might value a pony, even though they do not realise the challenges inherent in the care and upkeep of a said pony. Sen does identify that children and the mentally ill might need assistance in identifying capabilities that they value and desire, however the average person might also benefit from skills such as critical thinking to assist them in this process of identification. In so doing, individuals are opened to new understandings about the possibilities of each capability, and whether they would be truly valuable to them. 



\section{Motivations and Mandating Engagement}


\subsection{Mandating Community Service} % OLD STUFF MOSTLY
\label{Mandatory}

% WRITE THIS UP
Leaving aside the oxymoronic conceptual contradictions intrinsic to the term compulsory volunteering, many educational bodies have enacted compulsory community service as an element of the school curriculum. Using this experience, academia has learnt some very interesting insights. .....



 \citet[][p635]{Allen1994} come to the conclusion that mandating community service in high school may detract from the experience, and suggests that the data supports the voluntary aspect of the program is a critical feature, particularly for younger adolescents. However this conclusion might be limited, in that it suggests that there is no choice within the program itself. In fact, the discussions within his own paper referred to autonomy in the sense of individuals choosing what service to be part of, not whether one chooses to engage in service at all.

Given the above research on the benefits to the young people themselves of engaging young people as volunteers in community development activities, questions arise around the validity of mandating such engagement. Although many suggest mandatory or compulsory volunteering implies a lack of free will \citep[cf.][]{McLellan2003}, the effects of mandatory volunteering were examined after a number of Canadian provinces introduced it into their High School curriculum. \citet{Brown2007} did qualitative interviews and found that mandating volunteering in this context did not have a negative impact on the volunteering experience or on continued volunteering rates, however two factors positively influenced ongoing engagement - whether the experience was positive and if they volunteered with one organisation consistently \citep{Brown2007}. Even youth who held a negative view towards mandatory volunteering appreciated its positive outcomes for the individual and the community \citep{Brown2007}. These two factors - namely a positive experience and engaging with one organisation consistently - were examined more closely by \citet{Taylor2007} in a Canadian university setting, identified these factors as more important than background characteristics in determining the likelihood of ongoing volunteering. 

The requirement of service in the curriculum did not deter volunteering behaviour or attitudes in youth. In fact there were significant increases in the intent of youth to continue their volunteering beyond the requirements of the school \citep{Metz2003}

These findings were also replicated in a longitudinal study looking at three schools - one which did not require service learning and two which did \citep{Metz2005}. This study looked at changes in 4 civic measures for almost 500 students in the United States, and found that youth who were already likely to engage did not benefit significantly, those unlikely to engage gain significantly \citep{Metz2005}. 

Some studies have suggested that programs that require service, rather than inspire service, devalue the experience and can lead otherwise motivated youth to disengage \citep[see][]{Stukas1999}. \citeauthor{Stukas1999}'s \citeyearpar{Stukas1999} study looked solely at students who were required to undertake service, and compared those who indicated they were likely to engage without the requirement, to those who were unlikely to engage without the requirement. The duration of this study is also noteworthy - 12 weeks - as other research suggests a motivational transformation after continued engagement with a program. %TALK ABOUT Pearce2006... 
In \citeauthor{Pearce2006}'s \citeyearpar{Pearce2006} study it was only in the final stage of engagement in which motivations towards such programs changed, and intrinsic motivations were developed. This change often comes about after the frequently experienced positive involvement of engagement %SEE Pancer1999. 
It was also found by \citet{Henderson2007} that requiring service does not create a negative effect on students. Mandating service was insufficient in motivating student to continue their engagement, but instead sustained commitment seems to determine ongoing motivation and engagement \citep{Henderson2007}

 \citep{McLellan2003} - Looked at 2 types of compulsory school engagement - one which required service but left it to the individual to identify and engage according to their own interests, while the other integrated service into the curriculum. The latter tended to engage students emotionally and cognitively, and reflected on the political and moral aspects of the experience. This article suggests the importance of integrating service into the academic curriculum, and in doing so create a more systematic experience of planning, action and reflection. 

Young people themselves have also identified the importance of mandating such developmental opportunities. Whilst not specific to volunteering, programmes such as that of Pathways to Education - also based in Canada - designed to improve academic performance for low income communities, sought feedback from youth at different stages of the programmes implementation, and it was the youth themselves, in particular the ‘naughtiest’ young people, that identified the importance of making such opportunities mandatory:
\begin{quote}
	“The kids actually told us to make tutoring mandatory. After our first year... we asked the young people “how do we increase attendance at tutoring?” and the naughtiest boys of all looked at us and said “you have to make it mandatory”. So the young people want structure, and they want to know that somebody is caring about them.”
	 \citep[Caroline Acker as quoted in][]{Dubner2014}
\end{quote}
Whilst this quote centres on the use of mandating participation to improve academic performance, it highlights how youth themselves acknowledge the benefits of a particular venture while still needing a ‘nudge’ to ensure participation. It also highlights that mandatory service may be a way to demonstrate to the younger generation that ‘somebody is caring about them', which touches on the importance of supportive networks and cultures, an important element in deterring anti-social behaviours \citep{Kerestes2004} as well as in the establishment of a fair society \citep{Marmot2010}. %% HERNAN SAYS - double tick

%					This research does not necessarily advocate the mandating of community service, no matter the benefits it may bring. However if it is not mandated, and we are striving for an increasing number of youth to engage, then we must attempt to further understand motivations of youth towards community building practices. This is an area in which little is understood, however a few studies can gives us a glimpse of what motivates youth. 


\subsection{Motivation to Engage} %WRITE MORE
\label{Motivation}
The benefits of community engagement are numerous, however surveys of youth internationally show volunteering rates range from 20-60\% \citep{Flanagan1999}, and of those who do, most begin these practices during adolescence \citep{Hart2007, Youniss1999a, Metz2003, Zaff2008}. Individual and group motivations are rarely universally definable, but instead are context-specific and locality-bound; these motivations are also influenced by government policy, culture, history, and social, economic and political structures \citep{Botes2000}. A more robust understanding of the factors that motivate young people to engage, or not engage, could potentially assist those working with youth to increase retention rates, as well as to engage an increasing population of youth in the community, consequently increasing positive outcomes for young people and their communities. 

In 1999, Chapman and Morley investigated the motivations that led US tertiary students to engage in community service projects \citep{Chapman1999}. This quantitative study with 60 students - about half of whom had a required service-learning component - revealed two motivators universal to service: expressing their beliefs on the importance of helping others, as well as being able to understand others and themselves in relation to those they served. However those engaged in service-learning rated these two motivators as higher motivating forces. They also found that females identified values, understanding and self-esteem motives to be of greater importance than males \citep{Chapman1999}, which may also contribute to higher female participation rates \citep{Metz2003}. 

\subsubsection{Motivation - internal vs external}
Finally, US research explored motivational differences within low and high resourced schools, finding that student motivations to engage ranged from: beliefs and attitudes about the importance of civic engagement; personal goals derived from ideas of self-improvement; causes and issues where youth felt passionate; and individual or group invitational responses \citep{Ballard2014}. The first two findings again touch on the concept of values identified in earlier research \citep{Chapman1999}, however the expression of these values differed for higher and lower socio-economic groups: higher resourced students reflected more internal motivations reporting belief about the importance of engagement as a key motivation, while lower resourced students tended to be externally motivated, being motivated in relation to particular causes or issues about which they felt passionately \citep{Ballard2014}. Arguably, this suggests a lower likelihood of ongoing engagement for lower resourced students once that issue is no longer of concern.



\subsubsection{Motivation and Identity} 


Many have looked at the precedents of engagement with the intention to better understand motivation. In doing so, models of engagement have been proposed which include values, beliefs, demographics and identity. For example, social psychologist \citet{Penner2002} proposed one of the earlier conceptual models for initial motivation which indicated the importance of, inter alia,  demographics, values and beliefs, and characteristics of, and relationship to, the organisation. Additionally,while there was no specific evidence to support the idea that sustained engagement was more beneficial than sporadic  \citep[c.f.][]{Penner2002,Musick2003}, the model suggested that for sustained engagement, a volunteer identity was essential. If it is assumed that identity is formed during adolescence, as developmental psychologists do, this model seems to support the notion that engagement during adolescence would be important for sustained engagement and to derive the benefits.  

The link between identity and motivation has since gone beyond mere theoretical models. A  \citeyear{Hardy2006} study demonstrated the importance of identity as a motivator for prosocial behaviour. The study looked at 91 university students and found that when participants identified closely to prosocial values, their motivation to act on those values was present across different social contexts \citep{Hardy2006}. Furthermore, a study looking at extracurricular activities suggest particular identities motivate action. This study examined the different forms of identity - personal, role and collective - and found that each has different implications for how a person thinks and behaves: role identity such as gender, race and social class have implications within a society and gives rise to particular values and behaviours; collective identity serves to strengthen social ties between individuals and within groups; and personal identity has a twofold motivation - that of skills and abilities, as well as values and goals \citep{Eccles2009}. As such, the researcher concluded the importance of linking individual values to the values of the group as a means to motivate \citep{Eccles2009}.

Researchers \citet{Dawes2011} furthered this understanding looking more specifically at young peoples experience of social change. They conducted longitudinal interviews with a hundred 14-21 year olds examining motivation and motivational shifts. The researchers linked the importance of purpose and the formation of moral identity as key in furthering motivation for future engagement. Internal questions such as “Who am I?” and “What do I want to achieve or become?”, which then evolve into questions which link directly to the service they are undertaking, “What personal or transcendent goals are served by participation in program activities?” serve to further motivation and continued engagement \citep[][p266]{Dawes2011}. In this way, they found that the most effective means to engage with young people is to connect with their “earnest and serious side”, and not, as many programs attempt to do, attract youth through “fun” activities or material gains \citep[][p266]{Dawes2011}.



\subsubsection{Motivation - initial and sustained}

The relationship between initial and sustained volunteering was examined in more detail in later research when US researchers used longitudinal interviews to look at motivational changes towards community programmes when youth engage \citep{Pearce2006, Dawes2011}. A 2006 study identified that personal connection to a programme and its activities creates enjoyment, triggers interest and develops intrinsic motivation. Additionally, this research also found both peers and programme leaders were integral in facilitating motivational change throughout the program. Collective engagement developed a sense of efficacy and group meaning, resulting in the creation of both personal meaning and practical skills \citep{Pearce2006}. Their subsequent 2011 study identified that the practice of reflection proved effective in linking activities with individual values and goals, and whilst many programmes aim to attract youth through “fun” activities or material gains, the most effective means to engagement is connecting with their “earnest and serious side” \citep[][p266]{Dawes2011}. These findings challenge popular conceptions of youth as apathetic and narcissistic \citep{Nelson2014}.

Many motivation studies identify intrinsic motivation to be the primary reason for engagement, particularly ongoing engagement, and external motivations as secondary \citep{Dawes2011, Ballard2014, Eccles2002, Eccles2009}. The exception was a US study which interviewed 10 youth over 4 months of their community service \citep{Pearce2006}. They found that mandatory service brought with it extrinsic motivations, but that most youth experienced a motivational transformation in which continued engagement with a programme saw an elicitation of intrinsic motivations. Arguably then, external motivators such as peers \citep{Youniss2001,Wilkenfeld2009} or invitations \citep{Marzana2012} perhaps tap into an already existing, but unidentified, internal motivation, and this suggests that benefits of mandating engagement may outweigh any potential drawbacks. 

%	motivation, 
%	Despite knowing little about motivating youth, there has been some research that has looked at the likelihood of particular youth engaging, including demographics and values. 
%	predictors 

%	No matter how they come to engage in the process, youth experience numerous physical, emotional and academic benefits from community engagement. 

%	outcomes, 

%	In addition to benefits to the individual, we must also consider whether the purpose behind their engagement is fulfilled. Specifically, whether the desire of youth to help others in their community is actually realised. 

%	All these benefits evidence the importance of community engagement, yet almost none of the research is longitudinal, and as such there is still much to learn about the importance of, and understanding around continued volunteering. There are significant insights, however, in research that has examined volunteering experiences from lifelong volunteers, retrospectively. Whilst limited in its validity, it can gives us insights into whether ongoing volunteering gives added advantages (to the individual or to society), and what might motivate someone to continue, or discontinue their volunteering efforts. 
%	ongoing volunteering, 




***

The relationship between individual and social transformation is not just about how individuals benefit more when serving in society, nor is it about how when people work together they are more beneficial than if they were to work alone. These facts are true and important, however the relationship between individual and social transformation is also about the importance of having two foci simultaneously, regularly exploring with individuals ways in which they can develop their own capacities, their critical thinking, their compassion, their love for humanity, and simultaneously regularly exploring how they can make their society a better place, what injustices exist, how small acts of service can significantly improve the life of others or the community as a whole, believing that society has the capacity for continual advancement. There is potency in the link between these two concepts, however there is also importance when thinking about them individually as well. To omit thinking in terms of a two-fold purpose, could mean missing out on potential advances in all areas. 




\section{Approaches to Youth Community Engagement} % RESEARCH \citep{Hwang2005,Nelson1999}

Definitions and concepts of civic engagement, citizenship and participation all relate to one’s connection with community and a willingness to enact that connection in ways which promote its progress \citep{Lerner2007}. Despite definitions often having similar elements across discourses, desired outcomes of engagement vary. Young people express a range of reasons for engagement, yet most fall into one of two categories: that of personal development or to make a positive contribution to their community. Similarly, most academics argue the importance of engagement based on the positive effects it can have on the development of the individual, such as Positive Youth Development (PYD), or a focus on changing the community (often described as political empowerment) importance of engaged political citizens, such as Democratic Citizenship. These discourses are also the most prominent of those informing civic engagement practices \citep{Shaw2014,Jennings2006}. 

Words such as citizenship and civic are used ubiquitously in engagement literature, making them synonymous with implied and stated implications of responsibility. Civic engagement has a myriad of definitions: more general definitions include the rights and responsibilities of being a denizen of a community \citep{Zaff2010} to definitions focused on the ability for an individual to contribute to the well-being of their immediate or greater community \citepalias{UnitedNationsDevelopmentProgramme2010}; while youth focused definitions include young people's contributions to society as a virtue of citizenship \citep{Lister2007} and the ability of young people to influence society as part of their responsibility towards a democratic process \citep{Camino2002}. The  \citetalias[][]{UNDP2002} offer definitions for civic engagement and participation which are almost identical and underscores the systematic, ongoing nature of development, reflecting the interconnectedness of individual and social development: they define civic engagement as a “process, not an event, that closely involves people in the economic, social, cultural and political processes that affect their lives” (\citetalias[][p1]{UNDP2002} \citeyear[][p1]{UNDP2002}; \citetalias[see also their definition of participation in][p21]{UnitedNationsDevelopmentProgramme1993}). Additionally, they note that civic participation and civic engagement are used interchangeably, and that community service and volunteering are a subset therein \citepalias{UnitedNationsDevelopmentProgramme2010}. 

A range of citizenship concepts, from minimal to maximal, are used to describe the role of individuals within society, and these can limit or include youth in various ways. Minimal citizenship focused on legal status, voting and rights and responsibilities which often exclude youth. Whereas maximal citizens should be conscious of their role in society and accounts for an array of roles that an individual can play in ‘forming, maintaining and changing their communities’. This maximal concept of citizenship lends itself more to community building processes. Conceptualising citizenship in this way allows youth to be valued and valuable with their existing capabilities and not only as future contributors \citep[][p108]{White2008}. Citizenship, according to maximal interpretations, develops reflective practices, critical thinking, self-determination and autonomy. Additionally, they allow individuals to understand their relationship to a shared, democratic culture, the social disadvantage resulting from lack of participation, as well as their own involvement in the community \citep[][p109]{White2008}

Since definitions vary, this paper will use the terms community service and volunteering interchangeably to mean actions that benefit others in the community, and participation and community engagement, or merely engagement, as a broader term to mean involvement in the community such as being part of a community organisation. %MY DEFINITIONS - see \cite{Karlberg2005}... talks about service-learning (which I have not yet mentioned here).



With a growing awareness about the importance of one's environment in the opportunities and influences that young people are exposed to, little overlap really exists between community development research and youth engagement research. Much of the scholarly work today on community building overlooks the role particular groups of people within a community can play, for example, young people. As we can see, the role of civically engaged young people and the impact this involvement has on young people has been explored, with a number of youth studies fields having examined how young people experience and develop through engagement in community. From this perspective, we can understand how individual young people gain from this process - engaging in the community has a reciprocal positive impact on individual self-development \citep{Lerner2007} - but little about how the community gains. 




%ADD a section about Youth citizenship (after PYD)

\subsection{Positive Youth Development} %ADD this somewhere: “I would argue that there is a built-in contradiction in this reasoning”

Historically positive development in adolescents was seen in the absence of negative behaviours, indicating assumptions around their tendency towards young people being “problems to be managed” \citep{Lerner2005a}. The view of adolescents as a time of storm and stress is increasingly seen as counter-factual. Instead, the first decade of the current century saw a new vocabulary for the discussion around young people. This vocabulary was firmly based in concepts around young people as assets, capacious and resources to be developed. Such vocabulary included the importance of moral development \citep{Youniss1999a,Colby1999,Damon1997}, well-being \citep{Dolan2010}, civic engagement \citep{Zaff2010,Flanagan2007a}, and thriving \citep{Lerner2007}. 


Emerging from the barrage of deficit models of youth development arose a contradictory model which rested on the premises of youth as resources to be developed  \citep{Lerner2010}. The foremost of these approaches is the Positive Youth Development (PYD) model, which is based on the assumption that all youth have positive qualities and capacities existed, but which likely were in need of development. Positive Youth Development is predicated on the idea that young people have potential for positive development \citep{Lerner2010} and are “to be viewed as resources to be developed, and not as problems to be managed” \citep[][p27]{Lerner2005a}. When viewed in this light, young people engage in society by drawing on both internal and external resources, making a contribution to society while also gaining personally \citep{Sherrod2007}. This model focused on five key character areas which, according to its advocates, formed the basis for thriving \citep{Lerner2010}. These key characteristics are widely known as the 5 C's, and they constituted the qualities of competence, character, confidence, connection and caring. The relevance to this thesis is particularly pertinent in PYD's 6th C, contribution. Contribution, as the theory proposed, would naturally arise as a result of a firm foundation in the first 5 C's. 

There are many studies which report the significant benefits of these qualities. A sense of connection, for instance, has been linked to fewer academic challenges \citep{Youngblade2007}, while self-confidence is an integral aspect of many conceptualisations of empowerment \citep{Murphy-Graham2008}. This model became the foundation for many youth programs including the 4H program which has been the focus of a longitudinal study . Some Positive Youth Development advocates speak of a relative plasticity of the individual which legitimates the proposition that all individuals can successfully and positively develop \citep{Lerner2010}. This is similar to the neurobiology approach discussed in section \ref{Neuro}. This plasticity is, according to \citet{Lerner2010} the essence of adolescent neurobiology and legitimises a view of young people as assets to be developed. 

A focus on individual development is a key interest within much of the positive youth development literature. For example, \citet{Sherrod2007} discussed the 5 C's of development: competence, character, confidence, connection and caring. A sixth element, contribution, is expected to naturally emerge from the first five and is exhibited through community engagement \citep{Sherrod2007}. This suggests a level of causation which may not exist, since in this approach there is no direct education for, or encouragement around community engagement. Whilst some theorists note the unintended consequence of individual empowerment as a result of participating in community empowerment endeavours \citep{Jennings2006}, however arguably this is insufficient for true individual empowerment.



\subsection{Critique of PYD}
While there is growing support for this approach both from academia, as well as youth practitioners and increasingly the wider society, there remains questions about its validity. The 6th C, contribution, is of primary interest in this regard. The theory posits that young people who have effectively developed the initial 5 C's will naturally arise to contribute to the betterment of their community \citep{Sherrod2007}. This suggests a level of causation which is questionable. 

Despite this disjuncture, the concept of two-fold purpose is eloquently described discussions of PYD: “young people and their communities are involved in a bidirectional relationship wherein community assets are both products and producers of the actions of engaged young people” \citep[][[p178]{Lerner2010}. This disjuncture, to me, suggests a disconnect between the desired outcomes (which seem to line up well with the two fold purpose discussed in this thesis) and the methods of achieving those outcomes. This gives rise to questions around whether they see youth as capable of understanding and enacting the two aspects of development required (or the motivation to engage in such a way), or whether they see the focus on the self as a means to achieve these dual goals in a way which subverts the necessity to be forthright about them. %NOT sure if this makes sense? but basically, I am wondering if they are discussing it with youth in terms of the 5 c's (and subsequent 6th C) ect in a way that focuses on individual development because they don't think the youth have the capacity to understand the importance of the 6th C from the outset? For it is only when they discuss both the individual development and the contribution to the community do they use the word thriving 

The importance of contribution, however, can be found in the original 5 C's. When assessing the connection an individual experiences, the third C, specifies reflect how the young person is supported by individuals, institutions and their community, and less about their contribution. Examples of this include “My friends care about me”, “I get a lot of encouragement at school” and “Adults in my city or town make me feel important” \citep{Jelicic2007}. Connection is about positive bonds with others, community and institutions. It advocates that contribution to this relationship needs to come from both parties. This aspect certainly does point to the need for individuals to contribute to the wider community \citep{Jelicic2007}.

The way in which PYD theorists \citep[such as][]{Lerner2010} discuss civic engagement suggests they see the importance of community context - both in its influence on the individual as well as the importance of individuals having a moral obligation to its development. This view however, does not seem to come across in the basic “five C's” of PYD, or in the seeming inevitable outcome of the “sixth C”. Indeed, evaluations of PYD programs found that the vast majority (98\%) of students undertaking PYD programs also undertook a number of other additional programs. This brings into question the previously discussed finding that engaging young people in the 5 C's naturally saw the emergence of the 6th C a year later. Perhaps those who are already community minded are the ones participating in these programs, and the PYD program, whilst valuable, is not the key factor in predicting a young persons willingness to contribute to their community \citep{Lerner2005a}.


In the conclusion of one such article the authors suggest that youth programs think about the who, what and where of implementing PYD \citep{Jelicic2007}. However this suggests that only certain youth will benefit from this experience. Whilst acknowledging that the implementation of such programs will vary depending on context and needs of the social reality, and that some may be more in need of such initiatives than others because of the lack of already existing support structures, this should, in my opinion, in no wise limit the offering of these programs to all youth, irrespective of circumstance. Indeed, while PYD is claimed to be the most empirically tested framework to date, PYD nor any other framework is viewed as universally applicable. Indeed, it is claimed that universality is not possible given the divers and diverse conditions of the world \citep{Lerner2011}. This lack of universality, however, may not be as accurate as these theorists claim, as will be discussed in Chapter \ref{ConceptualFramework}. 



\subsection{The sixth C - Contribution}
Coming back to the sixth C as an expected outcome, the focus of Positive Youth Development, assumes a causality between the “Five C's” - competence, confidence, character, social connection, and caring or compassion - and what is naturally supposed to arise - the “sixth C” of contribution. Contribution is the desire and ability to give back to the communities and institutions with which an individual associates. It assumes that those who receive social benefits will instinctively understand their responsibility in contributing to, and even enhancing, that social system \citep{Lerner2010}. Individuals who have developed according to PYD naturally feel a sense of responsibility to others, and enact that in their lives \citep{ForrestBank2015}. Contribution is conceived as having two aspects - ideological and behavioural. The ideological component is linked to identity development specific to contributions to community of a moral or civic nature \citep{Lerner2005,Jelicic2007}.

In a longitudinal study looking at young people involved in a PYD program, found that students who had high levels of PYD in grade 5, had low levels of depression and risk-taking behaviours, as well as moderate levels of contribution (the sixth C) in grade 6 \citep{Jelicic2007}. The results from wave 1 included limited levels of contribution, indicating that the more an individual displays indicators of PYD, the more likely they were to be contributing to their community \citep{Lerner2005}. The study on PYD primarily measured the sixth C, contribution, at wave 2, which raises questions about whether their contribution already existed in wave 1. The results seem to imply a causation between the existence of the five Cs in wave 1 to the presence of the sixth C in wave 2 \citep{Jelicic2007}. 

However a more recent study actually challenges the causal assumptions between PYD's five C's and contribution. A PYD study looking at a sample of 17 adolescents who attended after school activities identified that it may be more beneficial to conceptualise contribution as more than merely an outcome. Two findings from this study are particularly interesting. Firstly, they found that young people who experience opportunities to contribute prior to entering late adolescence fair better. Second, the importance of the sixth C, contribution, was discussed as integral to the development of the individual, and that it should not be considered to come about only as a result of the first five C's. The authors noted that when this sixth C is integrated into programs, it promotes the proficiency of the other C's \citep{ForrestBank2015}. Opportunities to contribute, provide positive opportunities for young people irrespective of whether they came about as a result of youth programs or challenges in their own life, such as that of taking care of a younger sibling \citep{ForrestBank2015}. This echoes other research which identifies compulsory engagement can have a similar effect on long-term participation when compared to entirely voluntary experiences \citep[see][in section \ref{Mandatory}]{Brown2007,Stukas1999,Henderson2007,Pearce2006}. This finding led them to posit that contribution may in fact be the key to effective `interventions' for disadvantaged youth. This arguably brings about feelings of control in an environment which they would otherwise have no control, develop competence and autonomy \citep{ForrestBank2015}.



\subsection{Environment}

Both personal and environmental relations create a system which both facilitates and constrains opportunities for change \citep{Lerner2010}. Indeed, this interdependence is echoed by  \citet[][p173]{Lerner2010}: “The forefront of contemporary developmental theory and research is associated with ideas stressing that systemic (bidirectional, fused) relations between individuals and contexts provide the bases of human behaviour and developmental change... Within the context of such theories, changes across the life span are seen as propelled by the dynamic relations between individuals and the multiple levels of the ecology of human development (e.g. families, peer groups, schools, communities and culture), all changing interdependently across time...” \citep[][p173]{Lerner2010}. In a study using the Positive Youth Development framework, researchers found that a positive neighbourhood environment, where young people felt connected with others in their community, was positively associated with social competence, while negative neighbourhood environments was associated with externalising behaviours \citep{Youngblade2007}.


Developmental systems theories recognise both the role individual and social contexts play in development, and as such they focus on the relationship between these two when analysing development and development opportunities \citep{Lerner2010}. Understanding both the positive and negative aspects of multiple contexts of a young person's life - family, school, neighbourhood - as well as positive and negative aspects of individual behaviour, provides a more holistic interpretation of development \citep{Youngblade2007}. 








\section{unsorted}
REWORD - “Contemporary researchers (e.g., Youniss et al., 1999) increasingly frame questions about the impact of service activity on the healthy identity development of youth.” \citep{Lerner2010}

%NOT PYD
Adolescence is the period of time when there is significant biological, psychological, cognitive and social change within an individual, changing from child to adult characteristics. When an individual is in this state of change, of liminality, they are considered an adolescent \citep{Lerner2005a}.
%PROBLEMS EXIST DURING ADOLESCENCE
One cannot, however, deny the existence of problems during adolescence, or the value of efforts to avert problems. Yet, what is critical in any approach to curtail these problems is whether these young people are viewed as having deficits or strengths, and the impact those views have on how the individual overcomes those problems \citep{Lerner2005a}




“Youth as resources to be developed” \citep{Lerner2005b} %DEVELOPED for what? To be free to develop their full capacity

Whilst we cannot overlook the importance of viewing young people as resources to be developed, does this approach also rely too heavily on overlooking the negative? Maybe not?

%UNDERSTANDING THE REALITY, BASIS FOR PYD
Approaches to youth programs that promote a PYD approach focus on the strengths of young people, should be developmental in character, and must be flexible to the social circumstance in which these young people find themselves and the strengths and capacities of the young people themselves. When this is achieved “young people may thrive and civic society may prosper” [p15] \citep{Lerner2005b}




Use the word vocabulary instead of “definitions” or “concepts” ??? Then at the end one can talk about the absence of a standard vocabulary which is a key obstacle in the study of adolescence for both basic and applied scholarship


%CONCLUSTION - CROSSOVER BTN YOUTH DEVELOPMENT AND COMMUNITY DEVELOPMENT
The importance of community based youth development programs should not be abandoned for more general community development programs however, for it is through targeted programs such as this in which both individual development and youth contributions to the community are likely to take place. \citep{Lerner2005}
% NON-Parents
Youth development programs provide an ongoing connection with a non-parent adult who can assist them to develop their capacity and provide opportunities to contribute to the community. \citep{Lerner2005}


look up the construct of the 4-H study - does it include actual programs to build PYD, or merely analyse the existence of factors that contribute to it %SEE \cite{Lerner2005,Jelicic2007,Bowers2014,Lerner2007,Lerner2011,Youngblade2007,ForrestBank2015,Busseri2006}

The finding that indices of certain attitudes and behaviours being able to predict later attitudes and behaviours of a different nature support the importance of this research - that earlier engagement opportunities may bode well for greater and more meaningful development of the individual and their contribution to social change later in life. \citep{Jelicic2007}

%FALSE DICHOTOMIES
In this field, particularly as it relates to policy, there can be dichotomous debates based on false assumptions. Prevention versus promotion, problem youth versus capacity to contribute, capacity versus actual contributions. Understanding these as false dichotomies can open pathways for youth who, for example, have an ongoing struggle with risk-taking behaviours to offer meaningfully service to their community. This perhaps is evidence of the need to introduce such factors into the life of an individual who may not have access otherwise. Those who, as a result of their family, community or social context, do not have ready access to the five C's. \citep{Jelicic2007}

% COMMENT on Page 15, on highlighted text “for example, active engaged citizenship (Zaff, Kawashima-Ginsberg, & Lin, 2011)” : Is active engaged citizenship an approach which promotes PYD? ARGH! [Zaff, Kawashima-Ginsberg &amp; Lin 2011] Reply: Democratic Citizenship? \citep{Lerner2011}


Collective identity, as explored by social psychologists, is defined as QUOTE... “individual’s cognitive, moral, and emotional connection with a broader community, category, practice, or institution. It is a perception of shared status or relation . . . and it is distinct from personal identities, though it may form part of a personal identity.” (Polletta and Jasper, 2001, p. 285)



Collective identities are a means to belong and is a key component in health and well-being \cite{Futch2016}

Comment on Page 5, on highlighted text “Similarly, Jones and Deutsch (2012) combine a psychological needs approach (Deci and Ryan, 2000) with stage-environment fit theory (Eccles, Lord, and Midgley, 1991) to understand how after-school programs may serve as group-based settings that meet specific developmental tasks across pre-, early-, and midadolescence.”  % LOOK UP Jones and Deutsch 2012

REWORD FOR MY OWN PURPOSES: “For example, much of the social-psychological work on collective identity has stemmed from research on collective action as it relates to political movements. “

QUOTE: “Such work is important to highlight because it suggests ways that we can move from understanding settings as mere crucibles for youth development to exploring how identity development can occur in conjunction with youth actively shaping their settings. “: This underscores the impact of the environment on youth but ALSO the potential for youth to shape their world

REWORD FOR MY OWN PURPOSES: “Existing literature shows that development of and engagement with collec-tive identity(ies) are important psychological processes that influence how we understand ourselves and interact with our social world. “:

Engaging in youth programs influence thoughts, behaviours and actions beyond the program itself, impacting other contexts of the lives of youth.\cite{Futch2016}

REWORD FOR MY OWN PURPOSES: “The body of research is rich with evidence that theater, in particular, allows an opportunity for personal expression and creative reaction to the conditions in which its participants live. “:

REWORD FOR MY OWN PURPOSES: “Thus, while the findings of this study speak most directly to arts-based programs, the underlying processes that occur are relevant for any youth program that wants to understand its impact and the role of its group in the lives of youth, as I will argue below “:



Reword for my own purposes?: “overall aim being to show how combining these literatures may be mutually generative for both fields.”



Belonging arises out of participation when an individual feels a connection to the group, identification with the group as a means to express feelings of connectedness and that the group forms part of their identity \cite{Futch2016}

Social psychologist Kay Deaux (1991) reminds us that “beyond the question of how identities are defined is the question of the meaning associated with an identity” (p. 83, emphasis added) “:\cite{Futch2016}

Good quote about collective identity

“the individual development of the person in interaction with the environment and/or social condition to be equally as important as the extent to which individuals and settings are mutually informed and reconstituted through this interaction.” \cite{Futch2016}

Understanding the processes of collective identity allows insight into the psychological importance of development opportunities in youth programs, while also recognising that specific activities are likely to nurture this development.\cite{Futch2016}

