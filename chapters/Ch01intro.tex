% NICE INTRO? The world was "contracting into a neighborhood," Shoghi Effendi observed, and one wracked by social upheavals.

%OUTLINE OPTION 2
% Community building, participation, engagement
% Community Building and Youth
% The relationship between individual and social development

% OUTLINE
% 1. Individual Development - Theory
%		Sen's Capability Theory \& ISGP

% 2. Individual Development - Reality
%		Capacity building \& Empowerment... Identity, purpose, morality, meaning of life, critical thinking (including early adolescent brain development), 

%3. Social Development - Theory
%		Fukuyama's \& ISGP

%4. Social Development - Reality
%		Effective community development efforts; Youth engaging in community is not always positive for the community

% 5. Relationship Between Individual and Social Development - Theory
%		Participation, Critical Theory \& Freire, sense of obligation

%6. Relationship Between Individual and Social Development - Reality
%		Youth civic Engagement Literature, 
%		mandatory volunteering 
%					

% HERNAN SAYS - Summary - Homogonisation of youth - neuroscience; recruitment - > best and brightest and place? ; development or betterment for what and whom? ; what does community mean?... US college students often “need” to volunteer -> as part of their ??? ; an idealisation of youth? ; injustice(whose values?) implies a loack of morality -> affirmative action ; careful with the homogenisation of youth experiences (such as they are instrinsically motivated) -> social context (matters gender and SES will impact on capabilities - the idea of functionings); expand on Sen's critics ; like the idea of 'having' is not enough, 'doing' rather than 'having' ; BIG question - development for what? ; youth worling for the community's betterment? Whose betterment? ? this is an individual, rather than collective goal ; I like the idea of 'responsibility' and 'obligation' but to whom? ; Also assumptions that youth may have a sense of obligation -> I find them problematic ; I like understanding youth as a resources -> active youth ; In recruitment -> careful, sometimes schools just give you best and brightest ; are you doing urban, regional or rural? ; Does a neuroscientific definition of early adolescent separate “youth” from its social context -> and what it means to be young? MY THOUGHTS - whilst neuroscientific field have identified the development of critical thinking, how this is actualised in the lives of young people will undoubtedly vary (etc about meaning and social context) ; BIG question - if the idea is to understand how youth think about their capacity to controbute to community - what does “community” mean? ; and also careful to tokenistic youth participation and who gets to volunteer?. 

% ANDREW (Georgraphy) - tension between philosophy and science. Also, tension between philosophy - freire's conceptualisation of class and Sen's concept of class or socioeconomic status. shelly feldman - treatment of Sen's idea of individual capabilities talks about class that sits on top of a functioning society, while Freire critiques assymetiric relations of power within capitalism. Feels like Sen takes for granted those contradictions as they emerge from within capitalism, whereas Freire is saying that those contradictions themselves should be the where our criticisms are directed highlighting the boundaries


\chapter {Introduction}
\label{Intro}

What is the relationship between community building and individual development? Perhaps both hint at the importance of justice. Yet what is the meaning of justice? Is justice about having equal access to goods and services? Is it about equal treatment of individuals under universally applicable laws? Is it about equality of opportunity, which, in addition to equal treatment under the law, provides equal access to education? Or does the existence of a democracy imply a certain social equality such as health and housing or a redistribution of resources? Does social justice impede on the freedoms of the individual, and how can a society ensure a balance? And on what basis should these questions be discussed? Who should be involved in the discussion? And who should be involved in the creation of this just society? Should individuals achieve a certain level of competency to participate? What form of participation will be available to individuals with different capability sets? Whilst many of youth community engagement do not ask these questions, asking such questions gets to the heart of some of the challenges around youth engagement - a significant portion of the literature focuses on what youth gain from the experience and less attention is given to the value of the contributions they make. These are just some of the issues which arise when considerint youth community engagement, and which will be the focus of this thesis.
% FROM \cite{Lerner2005a} Comment on Page 59, on highlighted text " In sum, through developmental science research predicated on developmental systems models, we have a historically unique opportunity to conduct scholarship that will fruitfully address what may be argued to be the “really big” question for science and society, that is “What actions (e.g., actions predicated on the “Big Three”), of what duration, with what youth, in what communities, at what points in ontogenetic and historical time, will result in what features of positive youth development and contributions to self, family, community, and civil society?  Or, more simply, we may answer the question of “How do we foster mutually beneficial relations between healthy youth and a nation marked by social justice, democracy, and liberty? ":


% INCORPORATE THESE PARAGRAPHS INTO INTRODUCTION
Individuals naturally strive towards bettering their lot, evidence of this lies in the billions of dollars spent on personal development products. However there is often a realisation that bettering the situation of oneself is inadequate, and that one must look outward and act in a way that benefits others. This realisation can happen at various points in ones life, and often coincides with life changing events such as the death of a friend of family member or when there is a new baby in one's life \citep{Perry2008}. And, according to \citep[][p30]{Damon2009}, "youth is a time of "ignited passions"", and perhaps another time in ones life when an individual ponders their greater purpose in life. 
If this newly found interest is captured, the individual may find themselves devoting significant amounts of time to efforts of this nature, whether it be donating money, helping out neighbours or friends, or taking part in systematic community building efforts. It is this nexus which will be the focus of this thesis - the development of the individual and of society, and the role of adolescents in this process, in particular early adolescents. Ironically, when young people do participate, they are often faced with hostility or disbelief in the authenticity of their choice of engagement: hostility from politicians when protesting or belief that they were manipulated \citep[][p110]{White2008}. 

It is often said that youth are the future of humanity, the leaders of tomorrow. This is undoubtedly true, however youth also have a significant impact on society \textit{today}. Depending on the age range, youth make up almost 50\% of the worlds population, and how youth understand themselves, how they view their inherent capacities, and what they see as their role in society, has significant social consequences. Hence, to consider the impact youth \textit{can} have of the world is not, perhaps, of central concern. These young people \textit{are} having a significant impact on their communities, their families, their peers and on society. In which case, there needs to be a greater understanding around the desire young people have to contribute to meaningful change, and how those desires (ideally translated into actions) can be supported, strengthened and expanded. 

Adolescence in general is increasingly seen as an important time for individual development, particularly as it relates to civic engagement. Adolescence, here considered the age 12-25, is often broken down into early, middle and late adolescence - 12-15, 15-18, and 18-25 respectively. Early adolescence in particular is also beginning to be understood by neuroscientists as a key developmental stage, yet there is little correlation, so far, between the development of the early adolescent brain and how community engagement and community service facilitate those areas of development. 

% OLD INTRO
Being involved in empowerment programs for over a decade, I have seen first hand the importance of young people's involvement in the community - their energy and ideas are often unmatched by older contributors. I have watched as these youth grow and mature, and realise the joy in community service. Simultaneously I have also seen a number of youth who have not had someone to believe in them, and who have subsequently become consumed by negative social forces. Witnessing both, I began to inquire about effective efforts to develop capacities in these young people. I have watched as many programs offered to youth consist of “busy” activities which aim to prevent crime and antisocial behaviours through meaningless and superficial pursuits, not recognising or utilising their potential. And yet, of the young people I have met, either in the former category or the latter, there is a desire for, and often willingness to contribute to the creation of, a better society. 


What form this desire manifests inevitably varies, however \citet{Isin2007} suggest that, for social development to occur, depth of vision and clarity of pathways are dependent on the active participation of an ever-increasing body of individuals. The challenge of engaging ever-increasing numbers of people in social development initiatives is worthy of investigation. This paper explores the role young people can play in this process. Engaging youth would serve two purposes. First, it would potentially contribute to the existing number of individuals engaged in community building processes and, since those who engage during their adolescence are more likely to engage during adulthood \citep{Zeldin2000}, the pool of future potential human resources could grow exponentially. Second, engaging in the community brings significant advantages for positive individual development. 

Currently, around 30\% of the Australian population that believe in the importance of engaging in their community, and act accordingly \citep{AustralianBureauofStatistics2015}. Many reasons are given for this, but two of the most prominent, particularly by young people, is that of personal development and of the importance of giving back to their community \citep{Perry2008, Ballard2014}. Hence, in the creation of a vibrant, rich community life which continually advances, certain concepts require consideration: the challenge of engaging an ever-increasing number of individuals, particularly young people; how to incorporate the energies and opinions of all in this process; % HERNAN SAYS - this gets to the point of what is a community and who gets to participate?
what meaningful engagement looks like at this stage of an individual’s life; and what motivates young individuals to engage. 

To address these challenges, this research seeks to answer the following questions:
\begin{itemize}
\item How do young people (aged 12-14) understand their own capacity to contribute to their communities?
\begin{itemize}
	\item How do young people, in particular those in early adolescence, view community development? 
	\item What do they understand to be the role of young people in the process of community development?
	\item What do young people see as important in community building processes? What is their vision? How do they envision achieving this?
	\item What do they believe would motivate young people to get involved in community development activities?
\end{itemize}
\end{itemize}

As this paper will explore, youth engagement literature often focuses either on the importance of young people engaging due to a sense of responsibility felt towards their community, or as a means to develop their individual capacity. The research questions hope to probe these two foci, while focusing on an identified gap in the literature around early adolescent youth. Whilst the primary question probes understanding of their own capacity, and hopes to reflect aspects of individual capacity, % HERNAN SAYS - aspirations - check lit around this issue
the sub questions will also provide insight into the relationship between individual and social change and whether they view youth as important to that change. As such, the conceptual framework for this paper will explore the development of social and individual capabilities and the relationship between the two; while the literature review will look at community engagement specific to youth, covering fields and debates, general research about community engagement during adolescence, motivation and early adolescence. The paper will end with methodology and a timetable for the completion and write up of research.

\section{Conceptual Framework} 

At any given moment it is possible to view development from one of two perspectives, simultaneously occurring as a community grows in strength. Both perspectives are equally valid and important, and each give rise to particular patterns of thinking and speaking about development efforts. The first perspective is through the lens of individual development - the educational processes that impact personal growth and opportunities. From this perspective, resources and materials are unique to a particular time of life and developmental needs. Efforts to systematise youth development, for example, are enhanced through relevant knowledge and experience gained. Viewing development as the process of an individual advancing through educational processes available to him allows us to understand growing capacities of the individual, and how those can be further enhanced. 

From another perspective one thinks in terms of social development - the social forces that act within a society which allow for the ability or inability for wider advancement. Viewing development from this perspective - as the development of a society, a community, or a neighbourhood - can help us examine the relationships between individuals and the increased capacity of growth when those relationships are exploited, and often highlights strengths and weaknesses that warrant further development efforts. The question of targeting particular populations moves to the foreground in this light, as does the challenge of institutional capacity and social forces. 

It is particularly relevant for institutions to step back and view development from both of these vantage points, and to understand what is essentially one reality. Doing so avoids fragmentation and makes an analysis of social reality and subsequent planning more suitable to the context of any development initiative. 

% SEE Paragraphs 2, 3 and 4 on page 3 of the 12/12/2011 message from the Universal House of Justice:  COORDINATION - It is especially at the level of coordination that it proves indispensable to step back and view from these two vantage points what is essentially one reality. Doing so makes it possible to analyse accurately, to assess strategically, to allocate wisely, and to avoid fragmentation. At this point, then, early in the execution of the Plan, it seems more vital than ever for attention to be devoted to the issue of coordination. Though the basic elements of an effective organizational scheme are already well understood, the form it should assume under diverse circumstances is in need of articulation. We have asked the International Teaching Centre to follow efforts made in this direction, particularly in the several hundred furthest advanced clusters worldwide, in order to effect the rapid systematization of lessons learned...... In all such clusters, where the demands of large-scale growth are asserting themselves, each stage of the educational process promoted by the training institute must receive added support. The work of the coordinator should be reinforced by assistance from a growing number of experienced individuals, and meetings for the exchange of information and insights become regular and more systematic in approach. So, too, must periodic occasions be created for the three coordinators appointed by the institute—or, where applicable, teams of coordinators concerned with study circles, junior youth groups and children’s classes respectively—to examine together the strength of the educational process as a whole. And they, in turn, should meet on a regular basis with the Area Teaching Committee. Further, if an adequate flow of information, guidance and much-needed funds is to reach the cluster, a parallel set of steps will have to be taken by the board of the institute to enhance the functioning of that agency at the regional level. Where such a mature scheme of coordination is brought into place, the Auxiliary Board members and their assistants will be able to provide support across all areas of action with even greater effectiveness. .... One final point merits reflection in this respect. Nearly all of the several hundred clusters under consideration are associated with one or another of some forty sites for the dissemination of learning established by the Office of Social and Economic Development at the World Centre in response to the overwhelming demand for the junior youth programme experienced throughout the world. Institutes operating in these clusters have already benefited over the past year from knowledge gained through the sites, particularly in relation to coordination of the programme. Without question, the capacity to sustain scores of junior youth groups lent a powerful impetus to the progress of all such clusters and contributed decisively to the subsequent development of study circles and children’s classes. Sites supported by the Office of Social and Economic Development will continue to assist training institutes in addressing the complex set of questions arising out of the implementation of a programme for an age group whose enormous potential must remain the object of ongoing exploration. We look to the institutes themselves, however, to foster the learning process necessary to manage large numbers of children’s classes and study circles, to put in place a scheme at the cluster level that will strengthen coordination across their three defined areas of action, and to open the flow of resources from the regional level into the grassroots—this, to ensure the seamless progression of sizeable contingents from one stage of the educational process to the next and to facilitate the steady unfoldment of cycles of activity so essential to systematic growth.

A significant portion of youth engagement literature focuses on what they gain from the experience. However only a very small portion of this literature looks at what they are trying to achieve, and whether it is actually effective in these efforts. More specifically, there is a lack of understanding around the social benefits of youth engagement. It seems somewhat counter-intuitive to acknowledge that youth engage from a desire to contribute to the development of society, and promote this as a desirable goal, and simultaneously overlook the impact those actions have on the society. This seems to suggest that who they are is more important than what they do. This thesis will look at both aspects, the benefits to the individual and also to society, as two facets of the one reality. 

Individuals cannot segregate themselves from their environments, but rather they are simultaneously affecting and being affected by the world around them \citep{Sen1999, ISGP2010}. This theis will use the work of Fukuyama, Sen ad Freire to explore these issues in more detail. They propose conceptual frameworks which can be used to investigate the dual processes of individual and social transformation which are relevant to a study of young people's involvement in community development activity. Sen's work on development as a means to pursue freedoms which individuals value, along with Fukuyama's discussions about participation from within a community towards economic, social and political change and the importance of development as a continual process of learning, application and dissemination of knowledge, all speak to a process which would benefit from the engagement of an ever-increasing body of individuals, particularly young people. It is in this context that this research will examine development as a dual process for the individual and society. This two-fold purpose forms the basis of this conceptual framework: the development of individual and social capabilities and the inseparability of this relationship. 

[and the logic of the interaction between them]
	
\section{Unsorted}
%INCREASING individualism in the world
Institutional individualism, according to Beck, is the process by which institutional entities produce and affirm individualisation; education, welfare and other institutions focus on the individual. Individuals create, build, manage and juggle their own biographies and identities, their social networks. Gone are the days in which one was born into a particular society and their own selves molded by that inevitability. \citep{Beck2001}. Institutions have come to replace communities in many aspects of an individuals life. Individuals rely on institutional guidelines instead of social traditions to guide the decisions on one's life. \citep{Beck2001} 
This increasingly individualised world sees the individual blamed for failure. Instead of social problems, the individual is responsible for their own failings, and problems such as illness, unemployment, and addiction - which were previously communal issues - are the failure of the individual. \citep{Beck2001} 

Whilst culture was previously thought of as traditions, today it is a source of freedom, of individualisation. In the words of Beck, “...culture is the field in which we assert that we can live together, equal yet different” \citep{Beck2001}. 
Beck comments on the choice of young people to distance themselves of traditional politics, suggesting they are in a quest for “fun”. He does note that the very things they engage with to substitute their political engagement - advertising etc - is merely a different form of political engagement. While Beck is likely correct in his evaluation of the distancing of young people from traditional politics, perhaps this is more a reflection of the systems of institutions in which they find themselves, not the lack of “fun” represented therein. With an increasing contention ever-present in political debates, the criticism that is believed to come as a matter of course, arguably young people dislike the system more than the actual politics \citep{Beck2001}. Beck acknowledges the practice of distancing themselves from poltics is a means to question the system itself. \citep{Beck2001} 
This is evidenced by the desire for personal fulfilment being matched by the motivation to help others and concern for the public interest. These are not - as some might suggest - mutually exclusive. In fact, Beck asserts that these two goals actually reinforce and enrich each other. \citep{Beck2001} 
There is an assumption that those who help others are distinct from those who need help, yet Beck acknowledges that this assumption overlooks many individuals who help others also need and receive help. \citep{Beck2001} 
If we are to avoid conceptualisations of individuals as those of power and those without, of those who are helpers and those who are in need, we are better able to place youth in the reification of the transformation of society. They are, at once, individuals who are trying to better themselves, and able to contribute to the betterment of their society.
%DECLINE OF VALUES
Beck notes that the decline of values is an incorrect description of what is actually happening. Instead, he notes the conflict of values from the younger generation, that instead of embracing the values (and the institutions which uphold those values) of old, the younger generation grapple with a globalising world and the reality that comes with it. \citep{Beck2001}. Beck discusses the two simultaneous processes of individualization and globalisation; these two processes create a framework by which society is able to question and reshape itself. He speaks of the cries of the declining social values as a self righteous act permeated by empirical and historical blindness. \citep{Beck2001} %INTRO %MY THOUGHTS - historical blindness in the sense that individuals do not acknowledge similar processes occuring historically...

% Although giant strides have been made in recent years in the field of …, there remains an open question as to …

% The work described in the following chapters attempts to …
% TIPS - So it’s good to summarise the general principles you have just introduced, state a problem or question that needs an answer (and why it matters in relation to the general aims of your research field), and give a quick hint of how the next chapter will help to answer that question. ... “If man-made nanostructures are to follow a similar path [to nature], exploiting guided self-assembly to rapidly form functional structures, we must study both the physics of structure formation at the nanoscale and the influence of structure on function, specifically optical and electronic properties. Scanning probe techniques provide a versatile means of characterisation of these structures. Specifically, scanning near-field optical microscopy (SNOM) provides a means of optical characterisation with resolutions beyond the classical diffraction limit, in parallel with topographic information. These techniques, along with synchrotron based spectroscopy to probe deeper into the electronic properties of nanostructured assemblies, will be discussed in the following chapters.”

%INSERT THESE SOMEWHERE - Programs aimed at this age group need to enhance ability to percieve social forces and engage in dialogue, while also assisting them to develop capabilities necessary for a life of meaningful engagement in their community. ......... initiating one activity can naturally lead to the emergence of cmplimentary activities. ....... divers and evolving efforts ..... 
% AND from 29 december message - “The impulse to learn through action is, of course, present among the friends from the very start. The introduction of quarterly cycles of activity capitalizes on this emerging capacity and allows it to be steadily reinforced. Although this capacity is specifically associated with the reflection and planning phase of a cycle, especially the reflection gathering that regulates its pulsating heartbeat, it also comes to be exercised at all other points of the cycle by those pursuing related lines of action. We note that, as learning accelerates, the friends grow more capable of overcoming setbacks, whether small or large—diagnosing their root causes, exploring the underlying principles, bringing to bear relevant experience, identifying remedial steps, and assessing progress, until the process of growth has been fully reinvigorated”...... “Just like individuals, the agencies emerging in a cluster need assistance as they take up their duties. “ ... “Those serving at the regional or national level may do much to advise the friends and expand their vision of what can be accomplished, but they would not seek to impose their own expectations on the planning process; rather, they are helping the believers who are labouring in a cluster to gradually enhance their ability to devise and implement a course of action informed by the experience accumulating at the grassroots of the community and familiarity with actual conditions. In order to develop the capacity of cluster agencies to learn and to act systematically, regional and national institutions need to be conscientious and methodical in their own efforts to assist them.”
%AND - Not a linear process..... begins with a few simple lines of action and develops in complexity ... move past difficulties and to turn them into stepping stones for progress (obstacles are inevitable)..... before discussing the aforementioned capacities in more depth.... patient and concerted effort.... the system of support can be quite simple. One or more experienced individuals working...... a culture of mutual support and unity of vision characterise their efforts.... also benefit from assistance in sharpening their reading of reality ... animators benefit from the accumulating body of knowledge being gained .... those initially accompanying do not immediately withdraw, (continuing) to assist the process as it grows... (perhaps assisting in other capacities, furthering the development of their own skills)... chain of activities ....